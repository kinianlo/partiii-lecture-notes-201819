Last time, we considered the effect of gravitational radiation on geodesic deviation.

We found that for $V^a$ the vector connecting two neighboring geodesics,
$\frac{d^2V^a}{ds^2}-\frac{1}{2}\eta^{ab}(\p_0^2 h_{bc}) V^c=0.$
We had $h_{ab}=Ae_{ab}\text{Re}e^{i\omega(t-z)}$
for some characteristic frequency $\omega$. Now, a wave traveling in the $z$ direction will not displace particles in the $z$ direction since there are no nonzero components of $e_{za}$ and therefore no $h_{za}$. However, if we look at particles lying in the $xy$ plane, we can get some nontrivial motion.

Generally, we get
\begin{align*}
    \frac{d^2x}{ds^2}-\frac{1}{2}\ddot h_{xx}x -\frac{1}{2}\ddot h_{xy}y &=0\\
    \frac{d^2y}{ds^2}-\frac{1}{2}\ddot h_{yx}x-\frac{1}{2} \ddot h_{yy}y&=0.
\end{align*}
If we look at the $+$ polarization, $e_{xx}=1,e_{yy}=-1$ gives us
\begin{align*}
    \frac{d^2x}{ds^2}-\frac{1}{2}A\omega^2 \cos(\omega s) x &=0\\
    \frac{d^2y}{ds^2}+\frac{1}{2} A\omega^2 \cos(\omega s) y&=0.
\end{align*}
For the $\times$ polarization, we instead get
\begin{align*}
    \frac{d^2x}{ds^2}-\frac{1}{2}A\omega^2 \cos(\omega s) y &=0\\
    \frac{d^2y}{ds^2}+\frac{1}{2} A\omega^2 \cos(\omega s) x&=0.
\end{align*}
These are known as \term{Hill's equations} and while they're not pleasant to solve, we can intuitively understand how they distort rings of particles.

For the $+$ polarization, an initially circular ring of particles lying in the $xy$ plane is squeezed in the $y$ direction, then in the $x$ direction, and alternates between these. For the $\times$ polarization, the ring is squeezed along $y=x$ and then along $y=-x$.
%add images later?

We see that the displacement is only transverse to the direction of the wave motion, so gravitational waves are transverse waves. This is in analogy to EM waves. However, since the photon is spin $1$, the polarization states are a bit different-- we get ``up-and-down'' \emph{or} ``left-and-right'' motion, but not both at once.\footnote{This is different from taking linear combinations of the polarization states, which we can of course do. Really the difference is between side-to-side wiggles (massless spin 1) and squishing-squeezing wiggles (massless spin 2).} That we get squeezing along different axes is characteristic of the graviton being a spin 2 particle.

Most gravitational waves are produced by cataclysmic astrophysical events where a tremendous amount of energy is released. Recall that moving charges produce electromagnetic radiation with a power given by Larmor's formula,
$$P=\frac{2}{3}q^2 \ddot x^2.$$
Let us try to find an equivalent for classical gravitational radiation. That is, let us consider the simplest case-- the Einstein equations with fluctuations about flat space,
$$R_{ab}-\frac{1}{2}Rg_{ab}=8\pi T_{ab}.$$
We take perturbations of flat spacetime generated by $T_{ab}$, such that
$$g_{ab}=\eta_{ab}+h_{ab}$$ with $h_{ab}$ a small perturbation. Linearizing the Einstein equations gives
$$\Box h_{ab}=-16\pi(T_{ab}-\frac{1}{2}\eta_{ab} T)$$
for transverse tracefree perturbations. Previously, we just looked at the term linear in $h_{ab}$ in $R_{ab}-\frac{1}{2}Rg_{ab}$, but the Ricci tensor in fact involves all possible powers of $h_{ab}$. For note that if $g_{ab}\to \eta_{ab}+h_{ab}$, then the inverse metric goes to $g^{ab}\to \eta^{ab}-h^{ab}+h^{ac}h^b_c +O(h^3)$ Therefore the Christoffel symbols and the Ricci tensor will involve higher powers of the perturbation.

To zeroth order, there is no perturbation. However, we can write
$$[R_{ab}-\frac{1}{2}Rg_{ab}]^{(1)}+[R_{ab}-\frac{1}{2}Rg_{ab}]^{(2)}=8\pi T_{ab},$$
where the superscripts indicate terms of that order in $h$. Let us then rewrite the Einstein equations as
$$[R_{ab}-\frac{1}{2}Rg_{ab}]^{(1)}=8\pi T_{ab}-[R_{ab}-\frac{1}{2}Rg_{ab}]^{(2)},$$
where we interpret the first-order terms on the LHS as responding to the source $T_{ab}$ on the right, and the second-order terms as an effective energy-momentum tensor $t_{ab}$ for gravitational waves, i.e.
$$+8\pi t_{ab}=-[R_{ab}-\frac{1}{2}R g_{ab}]^{(2)}.$$
The effect is that in perturbation theory, gravitational waves carry energy and momentum.

Expanding out to second order in $H$ gives a big mess:\footnote{``As you can see, doing that involves hours of fun.'' --Malcolm Perry}
\begin{multline*}
    R_{ab}=\frac{1}{2}[\p_a h_{cd} \p_b h^{cd}+h^{cd}(\p_a \p_b h_{cd}+\p_c \p_d h_{ab}-\p_c \p_a h_{db}-\p_d \p_b h_{ac})\\
    +\p^c h^d_b(\p_c h_{ad}-\p_dh_{ac})-(\p_d h^{cd}-\frac{1}{2}\p^c h)(\p_b h_{cd}+\p_a h_{cd}-\p_c h_{ab})].
\end{multline*}
In electromagnetism, we often calculate the time average to make our lives easier. In particular terms which are total derivatives time average to zero, e.g. $\ang{\p_a(\quad)}=0$ and similarly $\ang{\p_a(\quad)\p_b(\quad)}=-\ang{(\quad)\p_a\p_b(\quad)}.$

The calculation is messy but the result is okay. We can define $\bar h$ to be the trace-reversed version of $h_{ab}$:
\begin{align*}
    \bar h_{ab}&=h_{ab}-\frac{1}{2}\eta_{ab} h\\
    h_{ab}&=\bar h_{ab}-\frac{1}{2}\eta_{ab}\bar h
\end{align*}
where $h$ is the trace of $h_{ab},$ $h=h_{ab}\eta^{ab}$. In this notation, we get an effective energy-momentum tensor of
$$t_{ab}=\frac{1}{32\pi}\ang{\p_a \bar h_{cd}\p_b \bar h^{cd}-\frac{1}{2}\p_a \bar h \p_b \bar h-2\p_c \bar h_a^c \p_d \bar h_b^d}.$$
In the transverse tracefree gauge we have $\p_c \bar h^c_a=0,\bar h=0$ and therefore
$$t_{ab}=\frac{1}{32\pi}\ang{\p_a \bar h_{cd} \p_b \bar h^{cd}}.$$
The quantity being averaged here is analogous to the the Poynting vector in electromagnetism. Now passing back to the Einstein equations, we have
$$\Box h_{ab}=-16\pi[T_{ab}-\frac{1}{2}\eta_{ab}T]$$
in the transverse tracefree gauge, or equivalently
$$\Box \bar h_{ab}=-16\pi T_{ab}.$$
We can then solve  for $\bar h_{ab}$ with a Green's function:
$$\bar h_{ab}(\vec x,t)=4\int d^3 x'\frac{1}{| x-x'|}T_{ab}(\vec x',t')$$
where $t'=t-|\vec x-\vec x'|$.

What if we consider static sources where $T_{ab}$ is only non-zero in some small region? Moreover, let us sit far away from whatever is radiating so that $r=|\vec x|\gg |\vec x'|=r'$. Then we may expand
$$\frac{1}{|\vec x-\vec x'|}=\frac{1}{(x^2-2\vec x \cdot \vec x' + {x'}^2)^{1/2}}=\frac{1}{r}\left(1+\frac{\vec x \cdot \vec x'}{r^2}-\frac{{x'}^2}{2r^2}+\frac{3}{2}\frac{(\vec x \cdot \vec x')^2}{r^4}+\ldots\right).$$
Let us denote by $\theta$ the angle between $\vec x$ and $\vec x'$ so that $\vec x \cdot \vec x'=r r' \cos\theta$. Then we can rewrite this expression as
$$\frac{1}{r}+\frac{1}{r^2}r' \cos\theta +\frac{1}{2r^3} r'^2(3\cos^2\theta-1)+\ldots.$$
We might\footnote{Well, I wouldn't.} recognize these functions of $\theta$ as the Legendre polynomials $P_1(\cos\theta),P_2(\cos\theta)$. Thus we are led to the solution
$$\bar h_{ab}=\frac{4}{r}\int {r'}^2 dr' \sin\theta d\theta d\phi T_{ab}(x')\left[1+\frac{r'\cos\theta}{r}+\frac{{r'}^2 (3\cos^2\theta-1)}{2r^2}+\ldots\right],$$
which is simply the multipole expansion for gravitational radiation.