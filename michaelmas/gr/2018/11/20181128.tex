A quick correction: on Example Sheet 3, Q1, the metric should be
$$ds^2=-(1+2\Phi)dt^2+(1-2\Phi)(dx^2+dy^2+dz^2).$$

Going on. Last time, we introduced the vierbein fields and wrote down a basis of 1-forms for the Schwarzschild metric.
We found that the one-forms were
\begin{align*}
    E^0 &= W dt\\
    E^1 &= dr/W\\
    E^2 &= rd\theta\\
    E^3 &= r\sin\theta d\phi,
\end{align*}
and the corresponding two-forms were
\begin{align*}
    dE^0 &= = -W' E^0 \wedge E^1\\
    dE^1 &= 0\\
    dE^2 &=\frac{W}{r} E^1 \wedge E^2\\
    dE^3 &=  \frac{W}{r} E^1 \wedge E^3 + \frac{\cot \theta}{r} E^2 \wedge E^3.
\end{align*}

Now we want the connection $1$-form. But we know that $$dE^\mu=-\omega^\mu{}_\nu \wedge E^\nu,$$
and that $\omega_{\mu\nu}=-\omega_{\nu\mu}$, so by direct comparison we see that with
\begin{align*}
    dE^0 &= -\omega^0{}_1 \wedge E^1 -\omega^0{}_2 \wedge E^2 -\omega^0{}_2 \wedge E^3\\
    dE^1 &= -\omega^1{}_0 \wedge E^0 -\omega^1{}_2 \wedge E^2 -\omega^1{}_3 \wedge E^3\\
    dE^2 &= -\omega^2{}_0 \wedge E^0 -\omega^2{}_1 \wedge E^1 -\omega^2{}_3 \wedge E^3.
\end{align*}
By comparison to the two-forms, we find that
\begin{align*}
    \omega^0{}_1 &= W' E^0\\
    \omega^0{}_2 &= 0\\
    \omega^0{}_3 &= 0\\
    \omega^1{}_2 &= -W/r E^2\\
    \omega^1{}_3 &= -W/r E^3\\
    \omega^2{}_3 &= -\cot\theta/r E^3
\end{align*}
and the others are related to these by symmetry.

To compute the curvature 2-form
$$\Omega^\mu{}_\nu=d\omega^\mu{}_\nu+\omega^\mu{}_\rho \wedge \omega^\rho{}_\nu,$$
we can do some direct computation. For instance,
$$\Omega^0{}_1=d\omega^0{}_1+\omega^0{}_1 \wedge \omega^1{}_1 +\omega^0{}_2 \wedge \omega^2{}_1 +\omega^0{}_3 \wedge \omega^3{}_1.$$
But these last two terms vanish by the $\omega$s we explicitly computed and $\omega^1{}_1=0$ by antisymmetry. Therefore
\begin{align*}
    \Omega^0{}_1&= d(W'E^0)\\
    &=W'' dr \wedge E^0 + W'dE^0\\
    &=W'' W E^1\wedge E^0 +W'(-W' E^0 \wedge E^1)\\
    &=-(WW''+{W'}^2)E^0 \wedge E^1.
\end{align*}
A similar computation holds for $\Omega^0{}_2,\Omega^0{}_3$:
$$\Omega^0{}_2 = \omega^0{}_1 \wedge \omega^1{}_2 = -WW'/r E^0 \wedge E^2$$
and
$$\Omega^0{}_3=-WW'/r E^0 \wedge E^3.$$
What about $\Omega^1{}_2$? Here, we have (discarding terms from asymmetry of $\omega$)
\begin{align*}
    \Omega^1{}_2&= d\omega^1{}_2+\omega^1{}_0 \wedge \omega^0{}_2 + \omega^1{}_3 \wedge \omega^3{}_3\\
    &= d(-W/r E^2)\\
    &=(-W'/r+W/r^2) dr \wedge E^2 -W/r dE^2\\
    &=(-WW'/r + W^2/r^2)E^1 \wedge E^2 -W^2/r^2 E^1 \wedge E^2\\
    &=-WW'/r E^1 \wedge E^2.
\end{align*}
where the third term in the first line is like $E^3\wedge E^3$ and so vanishes.
A  similar computation yields
$$\Omega^1{}_2 = -\frac{WW'}{r} E^1 \wedge E^3.$$
For $\Omega^2{}_3$, we have to do a bit more work.
\begin{align*}
    \Omega^2{}_3 &= d\omega^2{}_3 +\omega^2{}_1 \wedge \omega^1{}_3\\
    &=\frac{1}{r\sin^2\theta}d\theta \wedge E^3 + \frac{\cot\theta}{r^2}dr \wedge E^3 -\frac{\cot\theta}{r}\left(\frac{\cot\theta}{r}E^2 \wedge E^2 +\frac{W}{r} E^1 \wedge E^3 +\frac{W^2}{r^2}E^2 \wedge E^3\right)\\
    &=\left(\frac{1}{r^2 \sin^2\theta}-\frac{W^2}{r^2}-\frac{c\to^2\theta}{r^2}\right)E^2\wedge E^3\\
    &=\left(\frac{1}{r^2}-\frac{W^2}{r^2}\right) E^2 \wedge E^3.
\end{align*}

However, recall that the curvature $2$-form has a nice relation to the Riemann tensor:
$$\Omega^\mu{}_\nu=\frac{1}{2}R^\mu{}_{\nu\rho\sigma}E^\rho \wedge E^\sigma,$$
so this object contains all the information about curvature (and is often more convenient to compute than Riemann directly).

Note also that the Ricci tensor is the Riemann contracted over the first and third indices,
$$R_{\rho\sigma}=R^\mu{}_{\rho\mu\sigma},$$
so for instance
$$R_{00}=R^0{}_{000}+R^1{}_{010}+R^2{}_{020}+R^3{}_{030}.$$
But note that the Riemann tensor with an index up inherits the symmetries of the Riemann tensor with all indices down. For instance, $R^0{}_{000}=0$ since Riemann is antisymmetric in the first and last pair of indices. Meanwhile, $R^1{}_{010}\sim R_{1010}=R_{0101}=-R^0{}_{101}=+(WW'' +{W'}^2)$ from comparison to $\Omega^0{}_1$. Similarly, $R^2_{020}\sim -R^0{}_{202}=2WW'/r$, and $R^3{}_{030}\sim -R^0{}_{303}$, so we find that
$$R_{00}=WW'' + {W'}^2+2WW'/r.$$
A similar process yields the other components of Ricci,
\begin{align*}
    R_{11} &= -WW''-W'{}^2 -2WW'/r\\
    R_{22} &= -2WW'/r -W^2/r^2 +1/r^2 =R_{33}.
\end{align*}
It's a good exercise to check that with $W=\left(1-\frac{\text{const}}{r}\right)^{1/2}$ makes $R_{\mu\nu}=0$, i.e. Schwarzschild solves the vacuum Einstein equations with $\Lambda=0$.

\subsection*{Non-examinable: Spinors in curved spacetime} Recall from \emph{Quantum Field Theory} that a spinor is something that transforms like a spinor under the Lorentz group.\footnote{This is a bit tautological. If you prefer, a spinor is an object which satisfies the Lorentz algebra and for which a rotation of $4\pi$ but not $2\pi$ is the identity transformation. It is the correct formalism to describe objects of half-integer spin.}
We know that there are some generators of the Lorentz group $L_{\mu\nu}$ (with $L_{\mu\nu}$ antisymmetric), such that these generators satisfy the Lorentz algebra
$$[L_{\mu\nu},L_{\rho\sigma}]=\eta_{\mu\rho}L_{\nu\sigma} -\eta_{\mu\sigma}L_{\nu\rho}-\eta_{|nu\rho}L_{\mu\sigma} +\eta_{\nu\sigma}L_{\mu\rho}.$$

In particular, spin $1/2$ particles satisfy the anticommutation relations
$$\set{\gamma^\mu,\gamma^\nu}=2\eta^{\mu\nu} I_4,$$
such that the generators are
$$L_{\mu\nu}=\frac{i}{4}[\gamma_\mu, \gamma_\nu].$$
In the context of GR, spinors transform under infinitesimal Lorentz transformations as
$$\psi \to(1+\lambda^{\mu\nu}L_{\mu\nu})\psi,$$ with $\lambda_{\mu\nu}$ antisymmetric. That is, they do not transform like tensors, which complicates the definition of the covariant derivative. As it turns out, we define a covariant derivative on spinors as
$$\nabla_\mu \psi = \p_\mu \psi + \frac{1}{2} \underbrace{\omega_\mu{}^{\rho\sigma}}_{\text{spin connection}} L_{\rho\sigma} \psi$$
and $L$ is a generator of the Lorentz group.

The Dirac equation in curved spacetime now takes the form
$$0=(\gamma^\mu \nabla_\mu +m)\psi =(\gamma^a \nabla_a +m) \psi,$$
with $\nabla_a = e^\mu_a \nabla_\mu$ and $\gamma^a = \gamma^\mu e^a_\mu$ (i.e. written as Lorentz quantities in terms of the vierbein).

Note there's a complication with trying to quantize in curved spacetime. Usually we write down generalized coordinates $q^i$, which become fields $\phi(x)$ in field theory, and similarly generalized momenta $p_i = \P{L}{\dot \phi}=\dot \phi(x)=\pi(x).$
The canonical commutation relations in QM, $[p_i, q^j]=-i \delta^j_i$ become
$[\phi(x),\pi(x')]=i\delta^{(3)}(x-x')$. However, the complication here is that $\cdot=\P{}{t}$ when the underlying space is Minkowski, but this is not so well-defined in curved spacetime. If we're in the Schwarzschild spacetime we could use the regular $t$ coordinate, but we could just as well choose the advanced coordinate $u$ from Eddington-Finkelstein. What the ``right'' time derivative to take is totally unclear, and the problems only compound when we attempt to quantize gravity. This is of course a huge open problem, the details of which are beyond the scope of this course.

%$$u^\nu \nabla_\nu u^\mu=0$$
%$$u^\nu \nabla_\nu (\nabla_\mu u^\mu) =-S-R_{\rho\sigma}u^\rho u^\sigma, S\geq 0.$$
%$$R_{\rho\sigma} u^\rho u^\sigma \geq 0$$
%$$g_{ab}=\Omega \eta_{ab},\quad \Omega(t,x)=\begin{cases}
%1 & |x| \geq 1\\
%(1-1/t^6)x^2+1/t^6 & |x| < 1.
%\end{cases}$$
%$$-\frac{1}{t^6}\left(\frac{dt}{ds}\right)^2=-1\implies \frac{dt}{ds}=t^3\implies \tau_0=\int_0^{\tau_0} ds=\int_1^\infty \frac{dt}{t^3}=\frac{1}{2}.$$