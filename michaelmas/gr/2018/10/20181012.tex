Previously, we defined the exterior derivative, which took a $p$-form to a $p+1$-form. Now we will define the covariant derivative, an operation which in general takes a tensor of type $(r,s)$ to a tensor of type $(r,s+1)$.

Suppose we start with a scalar field $\phi(x)$. The ordinary derivative is just
$$\p_a \phi=\P{\phi}{{x^a}}.$$
Let us change coordinates to ${\tilde x^{a'}}$ some function of the original coordinates. Then this derivative transforms as
$$\p_{a'}\phi= \frac{\p x^a}{\p {\tilde x}^{a'}} \P{}{x^a}\phi = \frac{\p x^a}{\p {\tilde x}^{a'}}\p_a \phi.$$
That is, it transforms in the way we expect an index-down quantity to transform, with the correct factor of $\frac{\p x^a}{\p {\tilde x}^{a'}}.$
We might ask whether the partial derivative of a vector transforms in the same way. The answer is no-- instead, we get something a little different.

\begin{align*}
\p_{b} V^{a}\to \p_{b'} \tilde V^{a'} &= \left(\frac{\p x^b}{\p \tilde x^{b'}} \P{}{x^b}\right) \left( \frac{ \p \tilde x^{a'}}{\p x^a} V^a\right)\\
&=\frac{\p x^b}{\p \tilde x^{b'}} \frac{\p \tilde x^{a'}}{\p x^a} \p_b V^a+ \frac{\p x^b}{\p \tilde x^{b'}}\frac{\p^2 \tilde x^{a'}}{\p x^a \p x^b}V^a.
\end{align*}
This first part is tensorial, but the second part is not (it has a term which is a second derivative of the coordinates). In order to get a tensor, we must add a correction term to the partial derivative. 
\begin{defn}
This motivates us to define the \term{covariant derivative} by
$$\nabla_b V^a \equiv \p_b V^a +\Gamma^a_{bc} V^c$$
where $\Gamma^a_{bc}$ is called a \term{connection}. As the name suggests, a covariant derivative is a derivative which transforms in a tensorial way under arbitrary coordinate transformations.
\end{defn}
We can figure out how $\Gamma$ transforms under coordinate transformations:
$$\tilde \Gamma^{a'}_{b'c'}=\frac{\p \tilde x^{a'}}{\p x^a} \frac{ \p x^b}{\p \tilde x^{b'}}\frac{\p x^c}{\p \tilde x^{c'}} \Gamma^a_{bc}-\frac{\p^2 \tilde x^{a'}}{\p x^b \p x^c}\frac{\p x^b}{\p \tilde x^{b'}}\frac{\p x^c}{\p\tilde x^{c'}}.$$
So $\Gamma$ does \emph{not} transform as a tensor, but that's actually what we want-- this correction term allows us to get a proper tensor when we take the covariant derivative of a vector. Thus
$$\nabla_b V^a \to \nabla_{b'}V^{a'}=\frac{\p x^b}{\p \tilde x^{b'}} \frac{\p \tilde x^{a'}}{\p x^a}\p_b V^a+\frac{\p \tilde x^{a'}}{\p x^a} \frac{ \p x^b}{\p \tilde x^{b'}} \Gamma^a_{bc} V^c.$$
so $\nabla_b V^a$ is an honest tensor. We'd also like $\nabla$ to be linear and obey the Leibniz rule: for two tensors $T,S$ and two real numbers $\alpha,\beta\in \RR$, we should have
$$\nabla(\alpha T+\beta S)= \alpha \nabla T+\beta \nabla S$$
and also
$$\nabla(T\otimes S)=\nabla T \otimes S + T\otimes \nabla S.$$

For a vector $V$ and a one-form $W$, define the scalar $S=V^a W_a$. Then
\begin{align*}
\nabla_a S&= \p_a S\\
&= (\p_a V^b)W_b+ V^b(\p_a W_b)\\
&= (\nabla_a V^b)W_b- \Gamma^b_{ac} V^c W_b+V^b(\p_a W_b)\\
&= (\nabla_a V^b)W_b+ V^b \nabla_a W_b.
\end{align*}
Therefore for the Leibniz rule to hold on the product of a vector and a one-form, it must be that
$$\nabla_b W_a \equiv \p_b W_a - \Gamma^c_{ba}W_c.$$
Note the sign flip from the vector definition! More generally, we can use Leibniz to deduce what the covariant derivative operator is on a general tensor of type $(r,s)$.
$$\nabla_c T^{a_1\ldots a_r}_{b_1\ldots b_s} = \p_c T^{a_1\ldots a_r}_{b_1\ldots b_s}+ 
\Gamma^{a_1}_{cd} T^{da_2 \ldots a_r}+\Gamma^{a_2}_{cd} T^{a_1 d a_3 \ldots a_r}_{b_1\ldots}+\ldots + \Gamma^{a_r}_{cd}T^{a_1a_2\ldots d}_{\ldots} - \Gamma^d_{cb_1}T^{\ldots}_{db_2 \ldots b_s}- \Gamma^d_{cb_2} T^{\ldots}_{b_1 d \ldots b_s}-\ldots - \Gamma^d_{cb_s} T^{\ldots}_{b_1 b_2 \ldots d}.$$

So every upstairs indices we swap out gets a $+\Gamma$ and every downstairs index we swap gets a $-\Gamma$. Let's return to our expression for the transformation of $\Gamma,$
$$\tilde \Gamma^{a'}_{b'c'}=\frac{\p \tilde x^{a'}}{\p x^a} \frac{ \p x^a}{\p \tilde x^{b'}}\frac{\p x^c}{\p \tilde x^c} \Gamma^a_{bc}-\frac{\p^2 \tilde x^{a'}}{\p x^b \p x^c}\frac{\p x^b}{\p \tilde x^{b'}}\frac{\p x^c}{\p\tilde x^{c'}}.$$
Note that the second part is symmetric under the interchange of $b',c'$. Therefore take just the part antisymmetric in $b',c'$:
$$\Gamma^{a'}_{b'c'}-\Gamma^{a'}_{c'b'}=\frac{\p \tilde x^{a'}}{\p x^a}\frac{\p x^b}{\p \tilde x^{b'}}\frac{\p x^c}{\p \tilde x^{c'}}(\Gamma^a_{bc}-\Gamma^a_{cb}).$$
\begin{defn}
The antisymmetric part of $\Gamma$ transforms like a tensor, and so we define the \term{torsion tensor} as
$$T^a_{bc}\equiv\Gamma^a_{bc}-\Gamma^a_{cb}=2\Gamma^a_{[bc]}.$$
Some definitions define this up to a factor of 2 or with different signs.
\end{defn}

Consider an arbitrary scalar $S$.
$$(\nabla_a \nabla_b-\nabla_b\nabla_a)S= \nabla_a \p_b S- \nabla_b \p_a S.$$
If these were just partial derivatives, this commutator would be zero. But working it out explicitly, we see that 
$$(\nabla_a \nabla_b-\nabla_b\nabla_a)S=\p_a \p_b S-\Gamma^c_{ab}\p_c S -\p_b \p_a S + \Gamma^c_{ba}\p_c S = T^c_{ba} \p_c S = T^c_{ba}\nabla_c S.$$
Therefore the torsion measures how much covariant derivatives fail to commute on scalars. We'll see a generalization of this idea when we discuss the Riemann tensor later. In general relativity, the torsion is usually taken to be zero so that $\Gamma$ is symmetric in its lower indices. However, a treatment of fermions naturally requires non-zero torsion, and in local supersymmetry or ``superspace formulations of anything,'' non-zero torsion is essential.

Now, we haven't yet actually found what the connection is in terms of things we actually care about, like say the metric $g_{ab}.$
\begin{defn}
Let us define the \term{metric connection} as the $\Gamma$ such that
$$\nabla_c g_{ab}=0.$$
This will allow us to find a formula for $\Gamma$ in terms of the metric $g$.
\end{defn}

We'll work it out explicitly.
\begin{align*}
\nabla_a g_{bc}&=\p_a g_{bc}-\Gamma^d_{ab}g_{dc}-\Gamma^d_{ac}g_{bd}=0,\\
\nabla_b g_{ca}&=\p_b g_{ca}-\Gamma^d_{bc} g_{da} - \Gamma^d_{ba}g_{cd}=0,\\
\nabla_c g_{ab}&=\p_c g_{ab}-\Gamma^d_{ca} g_{bd}-\Gamma^d_{cb}g_{ad}=0.
\end{align*}
If we add the first two of these and subtract the third, we end up with
$$\p_a g_{bc}+\p_b g_{ca} - \p_c g_{ab} = 2\Gamma^d_{ab} g_{dc},$$
using the fact that $\Gamma^d_{bc}=\Gamma^d_{cb}$ since we want a torsion-free connection.

Now we simply multiply by $g^{ce}$ to find that
$$\frac{1}{2}g^{ce}(-\p_c g_{ab} +\p_a g_{bc} +\p_b g_{ca})=\Gamma^d_{ab} g_{dc} g^{ce} =\Gamma^d_{ab} \delta^e_d = \Gamma^e_{ab}.$$
This gives us explicitly the metric connection, which we sometimes call the Christoffel connection or Christoffel symbols.\footnote{They are a pain to compute by hand, hence why one professor of mine once referred to them as the ``Christ-awful symbols.''} Thus after a quick relabeling of indices we get
$$\Gamma^a_{bc}=\frac{1}{2} g^{ad}(-\p_d g_{bc}+\p_b g_{cd} + \p_c g_{bd}).$$
It is, as desired, symmetric under exchange $b\leftrightarrow c$ since the metric is symmetric, $g_{ab}=g_{ba}$.

So now on scalars,
$$(\nabla_a \nabla_b -\nabla_b \nabla_a)S=0,$$
i.e. covariant derivatives commute on scalars. Moreover using the metric connection if we have $V_a =g_{ab} V^b$, then
$$\nabla_c(V_a)=\nabla_c (g_{ab} V^b)=(\nabla_c g_{ab}) V^b+ g_{ab}(\nabla_c V^b)= g_{ab}\nabla_c V^b,$$
since $\nabla_c g_{ab}=0$. Therefore with the metric connection, the metric commutes with the covariant derivative. This is also true of the inverse metric, which one can prove as an exercise.

\begin{ex}
Prove\footnote{I think we can do this with Leibniz, actually. $\nabla_c(g_{ab}g^{ab})=(\nabla_c g_{ab}) g^{ab}+g_{ab} \nabla_c g^{ab} = 0$ since the trace of the metric is just a constant. If $g_{ab}$ is not identically zero, it must be that $\nabla_c g^{ab}$ vanishes.} that the covariant derivative of the inverse metric is also zero,
$$\nabla_c g^{ab}=0.$$
\end{ex}