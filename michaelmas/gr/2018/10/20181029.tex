Last time, we mentioned in passing that there is an action associated to the matter in our space,
$$I_{matter}=\int g^{1/2} d^4x L_{matter}.$$
For instance, for a scalar field $\phi$ in $\phi^4$ theory we might have
$$I_{matter}=\int g^{1/2} d^4x \left[-\frac{1}{2}g^{ab}\p_a \phi \p_b \phi -\frac{1}{2}m^2\phi^2-\frac{1}{4} \lambda \phi^4\right].$$
Now $\phi$ is invariant under variation of the metric, so
$$\delta I =\int g^{1/2} d^4x \left[\frac{1}{2} h^{ab}\p_a \phi \p_b \phi +\frac{1}{2} g_{ab}(-\frac{1}{2} g^{cd} \p_c \phi \p_d \phi -\frac{1}{2} m^2 \phi^2 -\frac{1}{4} \lambda \phi^4)\right].$$
The corresponding energy-momentum tensor is therefore
$$T_{ab}=\p_a \phi \p_b \phi -\frac{1}{2} g_{ab}(\p_c \phi \p_d g^{cd} +m^2\phi^2 +\frac{1}{2} \lambda \phi^4).$$
The equation for motion for $\phi$ is
$$-\Box \phi+ m^2 +\lambda \phi^3 =0,$$
where the $\Box$ operator is $\nabla_a \nabla^a$, a generalization of the Laplacian. (This is sometimes called the d'Alembertian.)

We can calculate $\nabla_a T^{ab}$ to get
$$\Box \phi \p_b \phi + \nabla_a \phi \nabla^a \nabla_b \phi - \nabla_b \phi \nabla^b \nabla_c \phi - m^2 \phi \nabla_b \phi - \lambda \phi^3 \nabla_b \phi,$$
where the second and third terms cancel,%check this
so we see that
$$\nabla_a T^{ab}=\nabla^b \phi(-\text{Equation of motion})=0.$$
Thus the energy-momentum tensor is conserved by the equations of motion.

Suppose $m^2 <0$ and look for solutions of the field equations that are constant,
$$\phi^2 = -m^2/\lambda.$$ Then $T_{ab}$ takes the form
$$T_{ab}=-\frac{1}{2}(-m^2 \frac{m^2}{\lambda}+\frac{\lambda}{2} \frac{m^2}{\lambda^2})= -\frac{1}{4}g_{ab} \frac{m^4}{\lambda}.$$

This is what happens in the Higgs mechanism. However, note that the stress-energy tensor is proportional to $g_{ab}$ only and is therefore equivalent to a cosmological constant. That is, from the Einstein equations
$$R_{ab}-\frac{1}{2}R g_{ab} +\Lambda g_{ab} = 8\pi G T_{ab},$$ 
we get a contribution to $\Lambda$ of
$$-8\pi G \left(\frac{1}{4} \frac{m^4}{\lambda}\right)$$ However, $\Lambda$ has a length scale of $10^{28}$ cm, whereas $m^4/\lambda$ has a weak interaction scale of $10^{15}$ cm, so there is a 43 orders of magnitude conflict between the value of the vacuum energy predicted by the Higgs mechanism and the value experimentally observed. There is no known solution to this problem.

What about the electromagnetic field? For this field, we have an associated action of
$$I_{em}=-\frac{1}{4}\int d^4x g^{1/2} F_{ab} F^{ab} = -\frac{1}{4}\int d^4x g^{1/2} F_{ab} F_{cd} g^{ac} g^{bd}.$$
Here, the Maxwell tensor is given as usual by $F_{ab}\equiv \p_a A_b -\p_b A_a$ with $A_a$ the electromagnetic four-potential. As the metric is varied, we take $\delta A_a$ to be zero and find the variation
$$\delta I_{em}=-\frac{1}{4} \int d^4x g^{1/2} [-2 h^{ac} g^{bd} F_{ab} F_{cd} + \frac{1}{2} g_{ab} F_{cd} F^{cd} h^{ab}].$$
We find that the corresponding stress-energy tensor is
$$T_{ab}=F_{ac} {F_b}^c - \frac{1}{4} g_{ab} F_{cd} F^{cd}.$$
It's a simple exercise to check that $\nabla_a T^{ab}=0$ if $\nabla_a F_{bc} + \nabla_b F_{ca} + \nabla_c F_{ab}=0$ and $\nabla_a F^{ab}=0$.

Let's now examine some symmetries of the Einstein equation. In the absence of matter, we have
$$R_{ab}-\frac{1}{2} R g_{ab} +\Lambda g_{ab}=0.$$
If $\Lambda=0$ as well, then one solution is the Minkowski spacetime,
$$ds^2 = -dt^2 + dx^2 +dy^2 +dz^2.$$
In Minkowski spacetime, the metric is constant and so all the Christoffel symbols vanish, $\Gamma^a_{bc}=0$, and thus all the nice tensors and scalars we cooked up also vanish,
$${R^a}_{bcd}=R_{ab}=R=0.$$ Note that the Minkowski metric is therefore invariant under translations, spatial rotations, and Lorentz transformations.

What if we consider a spacetime where $\Lambda >0$? Can we still find a maximally symmetric space for such a $\Lambda$? The answer is yes-- it is called de Sitter space, and the solution looks like
$$ds^2=\Omega^2(t) (-dt^2+dx^2+dy^2+dz^2),$$
where $\Omega$ cannot depend on $x,y,z$ to preserve rotational invariance but it may depend on $t$.

Note that the Einstein equations in vacuum take the nice form
\begin{eqnarray*}
0&=& g^{ab}(R_{ab}-\frac{1}{2} R g_{ab}+\Lambda g_{ab})\\
&=&R -\frac{1}{2} R (4) +4 \Lambda\\
&\implies& R=4\Lambda,
\end{eqnarray*}
so the Einstein equations become
$$R_{ab}=\Lambda g_{ab}.$$
Then using a result from the future (see Lecture 12, Eqn. \ref{ricciconformal})\footnote{No, I don't know why he assumed we knew this. We won't prove it for another two lectures.}
$$\Box \Omega = -\frac{2}{3} \Lambda \Omega^3$$
where $\Box$ is the flat space wave operator
$$\Box = -\frac{\p^2}{\p t^2}+\nabla^2.$$
Letting $\cdot =\P{}{t}$, we have
$$\ddot \Omega = \frac{2}{3} \Lambda \Omega^3.$$
We can solve by multiplying by $\dot \Omega$ and integrating,
$$\dot \Omega \ddot \Omega = \frac{2}{3} \Lambda \Omega^3 \dot \Omega \implies \frac{1}{2} \dot \Omega^2 = \frac{1}{2} \Lambda \Omega^4.$$
We conclude that
$$\dot \Omega = \sqrt{\frac{\Lambda}{3}} \Omega^2,$$
and integrating again we write
$$\frac{\dot \Omega}{\Omega^2}=\pm\sqrt{\Lambda/3} \implies \frac{1}{\Omega}=\mp \sqrt{\Lambda/3}t.$$
It follows that $\Omega^2=\frac{3}{\Lambda t^2}$, so our de Sitter metric is
$$ds^2 = \frac{3}{\Lambda t^2}[-dt^2+dx^2 +dy^2 +dz^2].$$
The limit $\Lambda \to 0$ is pathological, but as $t\to 0$ the metric coefficients blow up. We call this a \term{coordinate singularity} (as opposed to an essential singularity, like in a black hole). All this tells us is that this set of coordinates don't work towards $t\to 0$. The only way to tell in general if a divergence is an essential singularity is to find that one of our curvature tensors or scalars diverges, e.g. $R\to \infty.$ Conversely, we can get rid of coordinate singularities by an appropriate change of coordinates.

Here's an example. Suppose we invent the new coordinate $T$ given by
$$t=\sqrt{\frac{3}{\Lambda}} e^{\sqrt{\Lambda/3} T}.$$
Then the differential is
$$dt=dT e^{\sqrt{\Lambda/3}T}.$$
In the new coordinates, the metric takes the form
$$ds^2 = -dT^2 + e^{-2\sqrt{\Lambda/3}T}(dx^2+dy^2+dz^2).$$
Those of you with previous relativity experience might recognize this as a Friedman-Robertson-Walker (FRW) universe-- interpreted physically, this is simply a flat-space universe where space contracts as $T$ increases.

But we could have alternatively defined a coordinate $\tilde T$ such that
$$t=\sqrt{\frac{3}{\Lambda}}e^{-\sqrt{\Lambda/3}\tilde T}.$$
We'd get a similar metric to before, but now space is expanding:
$$ds^2 = -d\tilde T^2 +e^{2\sqrt{\Lambda/3}\tilde T} (dx^2+dy^2 + dz^2),$$
which says that space is expanding as $\tilde T$ increases. What's gone wrong? All that's happened is that our new ``time'' coordinates $T,\tilde T$ don't describe the entire spacetime. Note that as $T\to -\infty, t\to 0$ and as $\tilde T \to +\infty, t\to 0$.

Our two different metrics, one in $T,x,y,z$ and one in $\tilde T,x,y,z$, both represent solutions of Einstein's equations. But both come from the same original metric, so they seem to describe different parts of spacetime. We'd like to have a single set of coordinates which is good everywhere. Can we do this?

Let us define coordinates $X,Y,Z,V+W$ such that
\begin{eqnarray*}
T&=&\sqrt{\frac{3}{\Lambda}}\ln \frac{V+W}{\sqrt{3/\Lambda}}\\
x&=& \sqrt{\frac{3}{\Lambda}} \frac{X}{V+W}\\
Y&=& \sqrt{\frac{3}{\Lambda}} \frac{Y}{V+W}\\
Z&=& \sqrt{\frac{3}{\Lambda}} \frac{Z}{V+W}.
\end{eqnarray*}
We also impose the following constraint:
$$-V^2+W^2+X^2+Y^2 +Z^2=3/\Lambda,$$
which defines a surface in five-dimensional space. Suppose we look at this constraint in $\RR^{1,4}$, flat five-dimensional space with metric
$$ds^2=-dV^2+dW^2+dX^2+dY^2+dZ^2.$$
If we eliminate $W$ between these two (by taking $dW$ in the first equation) we find that the metric in our constrained five-dimensional space is precisely the same as what you get by plugging in
$T,x,y,z$ as functions of $V,W,X,Y,Z$ into
$$ds^2=-dT^2+e^{-2\sqrt{\Lambda/3}T}(dx^2+dy^2+dz^2).$$
The proof of this is in Anthony Zee's \textit{Einstein Gravity in a Nutshell} and also in Schr\"odinger's paper \textit{Properties of expanding universes}.