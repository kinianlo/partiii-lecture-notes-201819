Last time, we wrote down the geodesic equation. That is, we defined curves $x^a(s)$ such that
$$\frac{d^2x^a}{ds^2}+\Gamma^a_{bc}\frac{dx^b}{ds}\frac{dx^c}{ds}=0.$$
In terms of the tangent vector to the curve $V^a = \frac{dx^a}{ds},$
we can equivalently write
$$V^a\nabla_a V^b=0.$$

What happens if we reparametrize the curve? For instance, we could change to some new variables $\tilde s=\tilde s(s).$
Then $$\frac{d}{ds}=\frac{d\tilde s}{ds} \frac{d}{d\tilde s}\text{ and }\frac{d^2}{ds^2}=\frac{d^2 \tilde s}{ds^2}\frac{d}{d\tilde s} + \left(\frac{d\tilde s}{ds}\right)^2\frac{d^2}{d\tilde s^2}.$$

By rewriting in terms of $\tilde s$, we get a new version of the geodesic equation.
$$\left(\frac{d\tilde s}{ds}\right)^2 \frac{d^2}{d\tilde s^2} x^a+ \frac{d^2\tilde s}{ds^2} \frac{d}{d\tilde s} x^a + \left(\frac{d\tilde s}{ds}\right)^2 \Gamma^a_{bc} \frac{dx^b}{d\tilde s}\frac{dx^c}{d\tilde s}=0.$$
A little rearranging reveals that in terms of $\tilde s,$ we get
$$\frac{d^2x^a}{d\tilde s^2}+\Gamma^a_{bc} \frac{dx^b}{d\tilde s} \frac{dx^c}{d\tilde s}=-\frac{\frac{d^2\tilde s^2}{d s^2}}{\left(\frac{d\tilde s}{d s}\right)^2} \frac{dx^a}{d\tilde s}.$$
But $\tilde s$ is arbitrary, so in terms of our new tangent vector $\tilde V^a=dx^a/d\tilde s,$ we get a more general form of the geodesic equation,
$$\tilde V^b \nabla_b \tilde V^a=f(s)\tilde V^a,$$
where $f(s)$ is now some arbitrary function.
\begin{defn}
If $f(s)=0$, we say the geodesic is \term{affinely parametrized.} If a geodesic is affinely parametrized, then it remains so for $\tilde s$ such that $d^2\tilde s/ds^2=0$, e.g. for $\tilde s= as+b,$ $a,b$ constants.
\end{defn}

If a geodesic is affinely parametrized, then
$$V^a \nabla_a (V^b V_b)=2V_b V^a \nabla _a V^b=0$$ and so $V^b V_b$ is constant along the geodesic. This comes from the directional derivative interpretation of the operator $V^a \nabla_a$. That is, if the tangent vector to the geodesic is initially timelike(/spacelike/null) it will remain timelike(/spacelike/null) all along the geodesic.\footnote{We can expand out this calculation a little bit-- write $V^b V_b$ as $V^b V^c g_{bc}$. Now $\nabla_a(V^b V^c g_{bc})=V^b V^c\nabla_a( g_{bc})+V^b g_{bc} \nabla_a (V^c) + V^c g_{bc} \nabla_a (V^b)= 2 V_b \nabla_a V^b$ since $\nabla_a (g_{bc})=0$ by the metric connection condition. Therefore $V^a \nabla_a (V^b V_b)= 2V_b (V^a \nabla_a V^b) = 0$ by the geodesic equation.}

Recalling how the Christoffel symbol transforms under arbitrary changes of coordinates, we find that under $x^a \mapsto \tilde x^{a'}\equiv \tilde x^{a'}(x^b),$ we have
$$\Gamma^{a'}_{b'c'}= \frac{\p \tilde x^{a'}}{\p x^a}\frac{\p x^b}{\p \tilde x^{b'}}\frac{\p x^c}{\p \tilde x^{c'}}\Gamma^a_{bc}-\frac{\p^2 \tilde x^{a'}}{\p  x^{b} \p  x^{c}}\frac{\p x^b}{\p \tilde x^{b'}}\frac{\p x^c}{\p \tilde x^{c'}}.$$

Is it possible to make the new Christoffel symbol vanish in the $\tilde x$ coordinates? This is equivalent to the condition that
$$\frac{\p \tilde x^{a'}}{\p x^a} \Gamma^a_{bc}= \frac{\p^2 \tilde x^{a'}}{\p x^b \p x^c}.$$ Suppose we want it to vanish at $x^a_0$. Let us choose coordinates $\tilde x$ defined by
$$\tilde x^{a'}=(x^a-x^a_0)+\frac{1}{2}\Gamma^a_{bc}(x^b-x_0^b)(x^c-x^c_0).$$
Deriving with respect to the original coordinates yields
$$\frac{\p \tilde x^a}{\p x^e}=\delta^a_e +\Gamma^a_{bc}(x^b-x^b_0)\delta^c_e+\ldots$$ where the $\ldots$ denotes derivatives of the Christoffel symbols and we have only included terms to leading order in $x^b-x^b_0$. Similarly
$$\frac{\p^2 \tilde x^a}{\p x^e \p x^f}=\Gamma^a_{ef}+\text{derivatives of }\Gamma.$$

We therefore see that
$$\frac{\p \tilde x^{a'}}{\p x^a} \Gamma^a_{bc}= \Gamma^a_{bc} (\delta^{a'}_a +\Gamma^{a'}_{b'c'}(x^{b'}-x^{b'}_0)\delta^{c'}_a)=\Gamma^{a'}_{bc}= \frac{\p^2 \tilde x^{a'}}{\p x^b \p x^c}$$
when $x^b=x^b_0$, so we have found coordinates where the Christoffel symbols vanish at a point of our choosing, $\Gamma^{a'}_{b'c'}(x_0)=0$.

\begin{defn}
If we choose coordinates so that $\Gamma^a_{bc}$ vanishes at a point, then those coordinates are called \term{Gaussian normal coordinates.} In normal coordinates, the metric takes the form
$$g_{ab}=C_{ab}+O(x-x_0)^2,$$
where $C_{ab}$ is some set of constants. This choice of coordinates has forced the terms linear in $x-x_0$ to vanish, and by applying the Lorentz transform and rotations (and possibly a scale transformation) we can in general diagonalize $C_{ab}$ so that at $p$,
$$C_{ab}=\eta_{ab}.$$
\end{defn}
What we have learned is that spacetime can always be made to look like Minkowski spacetime at a given point (up to higher-order corrections). We also call such a choice of coordinates \term{inertial coordinates}.

In normal coordinates, the geodesic equation for an affinely parametrized curve is
$$\frac{d^2x^a}{ds^2}=0,$$
the equation of motion for a freely falling particle in Minkowski space. This confirms our intuition that a geodesic is really a generalization of a straight line.

Of course, it's also apparent that this equation is not covariant since it depends on our choice of coordinates $x^a$. In order to be coordinate-independent, we need to write this as a tensorial equation, which is just
$$V^a\nabla_a V^b=0.$$ If we further choose $V^a V_b=-1$ (this is true for the trajectory of a massive particle) then the parametrization $s$ is simply the proper time along the curve.

One can now consider families of curve parametrized by both time $t$ along the curve and another parameter $s$. The tangent vectors to each geodesic are given by
$$T^a(s) \equiv \frac{dx^a(t,s)}{dt}$$
and if we derive with respect to $s$ instead we get a tangent vector relating neighboring geodesics,
$$S^a(t)\equiv\frac{dx^a(t,s)}{ds},$$
sometimes called the deviation vector.
How does $S^a$ change as one moves along the geodesics? If we consider
$$V^b \equiv (\nabla_T S)^b=T^a\nabla_a S^b$$ this is like the ``relative velocity of geodesics'' (where I've used $\nabla_T$ to represent a directional covariant derivative)-- it measures how much the deviation vector $S^a$ points in the direction of the tangent vector $T^a$.

One can equivalently define the ``relative acceleration of geodesics'' as
$$A^b \equiv (\nabla_T V)^b = T^a \nabla_a V^b.$$ We'll revisit this quantity shortly.

\begin{lem}
Consider the following quantity:
$$S^a \nabla_a T^b-T^a \nabla_a S^b.$$ 
Expanding out, we find that $$S^a\nabla_a T^b- T^a \nabla_a S^b= S^a (\p_a T^b +\Gamma^b_{ac} T^c) - T^a (\p_a S^b +\Gamma^b_{ac} S^c).$$
The terms with Christoffel symbols cancel by the symmetry of $\Gamma^b_{ac}$ (since $a$ and $c$ are dummy indices, we can swap them and relabel). But $$S^a \p_a T^b = S^a \frac{\p T^b}{\p x^a} = \frac{\p^2 x^b}{\p s\p t} = T^a \p_a S^b,$$
so our expression
$$S^a \nabla_a T^b-T^a \nabla_a S^b=0.$$
This tells us that we can swap $S^a$ and $T^b$ through the covariant derivative so long as we keep the indices fixed.
\end{lem}

Now what is the acceleration between neighboring geodesics,
$$A^a =\frac{d^2S^a}{dt^2}?$$
Expanding out, we find that it is
\begin{eqnarray*}
\frac{d^2S^a}{dt^2}&=&T^c \nabla_c(T^b \nabla_b S^a)\\
&=&T^c (\nabla_c S^B)(\nabla_b T^a)+T^c S^b \nabla_c \nabla_b T^a\\
&=& T^c(\nabla_c S^b)(\nabla_b T^a)+T^c S^b(\nabla_b \nabla_c T^a +{{R_{cb}}^a}_d T^d).
\end{eqnarray*}
In the next step, we'll move the $T^c$ inside the $\nabla_b$ derivative to get
$$T^c(\nabla_c S^b)(\nabla_b T^a)+S^b\nabla_b(T^c \nabla_c T^a) - S^b(\nabla_b T^c) \nabla_c T^a+T^c S^b{{R_{cb}}^a}_d T^d.$$
But the second term vanishes by the geodesic equation, and the third term can be written as
$-S^c \nabla_c T^b \nabla_b T^a = -T^c \nabla_c S^b \nabla_b T^a$ (by the identity we proved earlier) so the third term cancels the first one.

What remains is a function of the Riemann tensor. Using the symmetries of the Riemann tensor, we can rewrite our acceleration equation as
$$\frac{d^2S^a}{dt^2}={R^a}_{bcd} T^b T^c S^d,$$
which we call the \term{equation of geodesic deviation.} If the Riemann tensor is zero, then the relative velocity between geodesics is constant. If the Riemann tensor $\neq 0$, then we get a ``stretching force'' between neighboring geodesics, sometimes called a tidal force.