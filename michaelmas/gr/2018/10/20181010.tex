\subsection*{A quick admin note} There is no lecture Monday 15 October. In addition, office hours will be Tuesdays at 4 PM in B1.26. Moving on.

Let us recall that we have a multiplication law on one-forms and vectors, 
$$\langle \omega, X\rangle = \omega_a X^a$$
for $\omega$ any one-form, $X$ any vector. That is, we can write this product in terms of the components of $\omega$ and $X$. 

\begin{defn}
With this in mind, we define the \term{differential} of a function $f:M\to \RR$ to be the one-form $df$, such that
$$\langle df, X \rangle = Xf$$
(that is, $X$ as a differential operator acting on $f$).
\end{defn}
\begin{exm}
Non-lectured example: consider the function $f=x+y$ in $\RR^3$ and let $X= \P{}{y}$. (We have chosen a coordinate basis to make the computation clearer.) Then $df=dx+dy$ (a one-form) and now
$$\langle df, X \rangle = Xf = \P{}{y}(x+y)=1.$$
\end{exm}

Recall we have a basis of 1-forms $E^a$ and a basis of vectors $E_b$ with $\langle E^a, E_b \rangle = \delta^a_b$. In a coordinate basis, the basis vectors take the form
$$E_a = \P{}{x^a}\text{ and }E^b= dx^b.$$
Thus
$$\langle dx^a, \P{}{{x^b}}\rangle = \delta^a_b.$$
\begin{defn}
A one-form is \term{exact} if it can be written as $df$ for some scalar $f$. For instance, $dt$ and $dr$ are exact because they are the differentials of $t$ and $r$, but $rd\theta$ is not exact. However, the one-form $r dr$ is exact, since it can be written $d(r^2 /2).$
\end{defn}
In Minkowski space with Cartesian coordinates, the natural basis of one-forms $dt,dx,dy,dz$ forms a coordinate basis since each of these is exact, and the basis of vectors dual to this is $\P{}{t},\P{}{x},\P{}{y},\P{}{z}$. 

However, in spherical coordinates the Minkowski metric looks different. It takes the form
$$ds^2=-dt^2 +dr^2+r^2d\theta^2 +r^2 \sin^2\theta d\phi^2.$$
The basis of one-forms here, 
$$dt, dr, rd\theta, r\sin\theta d\phi$$
is not a coordinate basis because these are not all of the form $df$.
The set of basis vectors dual to the one-forms in spherical coordinates is also kind of bad. They take the form 
$$\P{}{t},\P{}{r},\frac{1}{r}\P{}{\theta},\frac{1}{r\sin\theta}\P{}{\phi},$$ and these are not a coordinate basis because they are not of the form $\P{}{{x^a}}$ (equivalently, they are not dual to exact one-forms).

However, we remark that our defining equation for the product of a one-form and vector produces an ordinary scalar, which must be invariant under coordinate transformations:
$$\langle \omega, X\rangle = \omega_a X^a\text{ in any basis.}$$
This determines how the components of a one-form $\omega_a$ change under coordinate transformations.
In a coordinate basis, we know that the components of a vector transform like coordinate functions:
$$X^a\to \tilde X^{a'}=\frac{\p \tilde x^{a'}}{\p x^a} X^a.$$
Therefore in a coordinate basis, the components of a one-form must transform in the inverse way,
$$\omega_a \to \tilde \omega_{a'} = \frac{\p x^a}{\p \tilde x^{a'}} \omega_a.$$
Note where the primed indices lie and which coordinates are the new coordinates $\tilde x$ versus the old coordinates $x$. A little mnemonic-- to keep the indices straight, just remember that primed indices go with $\tilde x$ coordinates, and we can only contract over pairs of up and down indices. In particular, when $x^a$ is in the denominator\footnote{Okay, technically not the denominator since it's a derivative but you know what I mean.} of a derivative $\P{\tilde x^{a'}}{x^a}$ it acts like an index-down quantity, so it should contract with an index-up object, namely the vector components $X^a$. Similarly, when $x^a$ is in the numerator of the derivative it remains index-up and should therefore contract with the index-down one-form components.%The factor here $\frac{\p x^a}{\p \tilde x^{a'}}$ is analogous to how the Lorentz transformation acts (as the Lorentz transformation is a particular coordinate transformation satisfying certain constraints).

Suppose that $\langle df ,X\rangle = 0$ for some $df$ and $X$ an arbitrary vector. If we are working in $n$ dimensions, this equation gives one constraint on the $n$ components of $X$. Thus, there are still $(n-1)$ different linearly independent choices of $X$ which solve this equation, so the solutions $X$ therefore span an $n-1$-dimensional space. We have put one constraint specified by $f$ on our space of all possible $X$ such that $df$ is the normal to the surface $f=$ constant.

\begin{exm}
Again, a non-lectured concrete example. Let us again work in $\RR^3$ and set $f=x.$ Then a general $X$ can be written as $X^a \P{}{x^a}$ and the condition that $\ang{df,X}=0$ can be computed explicitly as
$$\ang{df,X}=\paren{X^1 \P{}{x}+X^2 \P{}{y}+X^3 \P{}{z}}(x)=X^1(1)=0.$$
Therefore our surface is defined by $X^1=0$ but we may choose $X^2$ and $X^3$ freely (one constraint gives $3-1=2$ free choices). Indeed, we see that $df=dx$ is normal to the surface $f=x=$ constant.
\end{exm}

\begin{defn}
A \term{tensor} is a coordinate-invariant object which generalizes the idea of vectors and covectors. Written in terms of basis one-forms $E^a$ and basis vectors $E_a$, a tensor of type $(r,s)$ takes the form
$$T={T^{a_1\ldots a_r}}_{b_1\ldots b_s} E_{a_1}\otimes E_{a_2}\otimes\ldots \otimes E_{a_r} \otimes E^{b_a}\otimes \ldots \otimes E^{b_s},$$
where $\otimes$ is the tensor product (not just a direct product!).\footnote{Tensor products are more complicated than direct products because their addition structure is multilinear, i.e. linear in each argument individually but not all simultaneously. Where it might make sense to add $(2,1)+(1,2)=(3,3)$ in $\RR\times \RR$, the equivalent tensor product in $\RR \otimes \RR$ would have $2 \otimes 1 + 1\otimes 2 = 2\otimes 1 + 2 \otimes 1= (2+2)\otimes 1 = 4 \otimes 1$. So this is quite a different beast. More info on tensor products and tensors as mathematical constructions can be found at \url{https://jeremykun.com/2014/01/17/how-to-conquer-tensorphobia/}.}
\end{defn}

The tensor $T$ (not the components!) is coordinate invariant, so in a coordinate basis the components of $T$ transform as
$${{\tilde T}^{a_1'\ldots a_r'}}{}_{b_1'\ldots b_s'}=\frac{\p \tilde x^{a_1'}}{\p x^{a_1}} \ldots \frac{\p \tilde x^{a_r'}}{\p x^{a_r}} \frac{\p x^{b_1}}{\tilde x^{b_1'}} \ldots \frac{\p x^{b_s}}{\tilde x^{b_s'}} {T^{a_1\ldots a_n}}_{b_1\ldots b_s}.$$
In a non-coordinate basis, these $\frac{\p \tilde x^{a'}}{\p x^a}$ are replaced by some general functions $\Phi ^{a'}_a$ where $\tilde x^{a'}=\Phi^{a'}_a x^a$.

We can perform the symmetrization operation, denoted by putting indices to be symmetrized in parentheses:
$$X_{(a_1 \ldots a_r)}\equiv \frac{1}{r!}\left[\text{sum of all permutations of }a_1\ldots a_r\right].$$
For example, $X_{(ab)}=\frac{1}{2}\left[X_{ab}+X_{ba}\right]$. Here, the factorial accounts for that the symmetrization of an already-symmetric tensor should just be that tensor (so that if $X_{ab}=X_{ba},X_{(ab)}=\frac{1}{2}[X_{ab}+X_{ba}]=X_{ab}$). 

Similarly we have the antisymmetrization operation, denoted by putting indices to be antisymmetrized in square brackets:
$$X_{[a_1\ldots a_r]}=\frac{1}{r!}\left[\text{sum over all even permutations } - \text { sum of all odd permutations}\right].$$
For example, $X_{[ab]}=\frac{1}{2}[X_{ab}-X_{ba}].$ Having defined symmetrization and antisymmetrization, we now consider a special class of tensor-- the totally antisymmetric $(0,p)$ tensor.

\begin{defn}
A \term{differential $p$-form} is a tensor of type $(0,p)$ which is antisymmetric on all indices, i.e. $A_{a_1\ldots a_p}=A_{[a_1 \ldots a_p]}$. Some familiar $p$-forms include the $2$-form $F_{\mu\nu}$ from electromagnetism and the Levi-Civita symbol $\epsilon_{ijk}$.
\end{defn}
We can describe $A$ in terms of basis vectors $E^a$ using a construction called the wedge product.
\begin{defn}
The \term{wedge product} is a special kind of antisymmetrizing multiplication of a $p$-form and a $q$-form. For a $p$-form $A=A_{a_1\ldots a_p}$ and a $q$-form $B=B_{b_1\ldots b_q}$, the wedge product $A\wedge B$ is given by
$$(A\wedge B)_{a_1\ldots a_p b_1 \ldots b_q}\equiv A_{[a_1\ldots a_p}B_{b_1\ldots b_q]}.$$
For instance $A\wedge B = (-1)^{pq}B \wedge A$ (this is easy to prove-- we simply switch the $q$ indices of $B$ past the $p$ indices of $A$ and pick up the appropriate $pq$ sign flips along the way).
\end{defn}
As an invariant object, the $p$-form $A$ can be written as $$A=A_{a_1\ldots a_p} E^{a_1}\wedge \ldots \wedge E^{a_p},$$ where $A_{a_1\ldots a_p}$ are now the components of the $p$-form $A$.%
    \footnote{It's nice to prove here that $A\wedge B=(-1)^{pq} B\wedge A$ in a basis. We can see that
    \begin{align*}
        (A\wedge B)_{a_1\ldots a_p b_1 \ldots b_q} &= A_{a_1\ldots a_p} B_{b_1\ldots b_q} E^{a_1}\wedge \ldots \wedge E^{a_p} \wedge E^{b_1} \wedge \ldots \wedge E^{b_q}\\
        &= (-1)^{pq} B_{b_1 \ldots b_q} A_{a_1\ldots a_p} E^{b_1} \wedge \ldots \wedge E^{b_q} \wedge E^{a_1} \wedge \ldots \wedge E^{a_p}\\
        &= (-1)^{pq} B\wedge A,
    \end{align*}
    since the components themselves are just numbers and we pick up $pq$ sign flips from commuting the $p$ basis one-forms $E^{a_1}\ldots E^{a_p}$ with the $q$ basis one-forms $E^{b_1}\ldots E^{b_q}$.}

\begin{defn}
We also define the exterior derivative, a generalization of the usual derivative $\p_\mu$:
$$(dA)_{ba_1 \ldots a_p} \equiv \P{}{{x^{[b}}} A_{a_1 \ldots a_p]}=\p_{[b}A_{a_1 \ldots a_p]}$$
defines a $p+1$-form, as it is by definition antisymmetric in its $p+1$ indices.
The exterior derivative of a product follows a variation of the Leibniz rule:
$$d(A\wedge B)=dA\wedge B +(-1)^p A\wedge dB.$$
Note that $ddA=0$, so $d$ is nilpotent (it kills all exact differentials).\footnote{Suppose we compute $ddA$: then we will have two derivatives in our expression $\p_{[\mu} \p_\nu A_{a_1\ldots a_p]}$. But derivatives commute, so to every $\p_\alpha \p_\beta$ term in the antisymmetrization sum there will be a corresponding $-\p_\beta \p_\alpha$ term. These terms cancel no matter what $A$ is, so $ddA=0$ identically.}
\end{defn}

The gradient is a simple example of an exterior derivative of a 0-form (AKA a scalar):
$$(d\phi)_\mu=\p_\mu \phi.$$

We now introduce the metric, a very special rank two (i.e. two-index) symmetric tensor usually denoted $g_{\mu\nu}$.%
    \footnote{So far, we have been using Latin indices everywhere. In most GR contexts, Greek indices $\mu,\nu,\sigma,$ etc. are used to range over $0,1,2,3$ and Latin indices $i,j,k$ over $1,2,3$ (the spatial components). Here, we follow the lecturer's convention of using Latin $a,b,c,d$ for all indices but note that it is nonstandard. There is a sense in which Latin indices are used in an ``abstract tensor notation'' to serve as placeholders and simply indicate the type of the tensor (i.e. a $(p,q)$ tensor) rather than the components, but that is a little different.} 
The metric generalizes the idea of distance from Euclidean geometry to curved spaces. Unlike the Euclidean metric, the scalar which the metric spits out is not guaranteed to be non-negative-- recall from special relativity that a timelike four-vector $V^\mu$ has a ''length'' given by $\eta_{\mu\nu} V^\mu V^\nu <0$. However, it is a scalar invariant and therefore preserved under arbitrary coordinate transformations.

From prior experience with special (or general) relativity, we might have an intuition that the metric also has something to do with gravitation. The line element $ds$ (defined by $ds^2 = g_{ab}dx^a dx^b$) is invariant and is therefore a (symmetric) tensor. In a freely falling frame, the metric of Minkowski space is $$\tilde \eta_{a'b'} = \frac{\p x^a}{\p\tilde x^{a'}}\frac{\p x^b}{\p \tilde x^{b'} }g_{ab}.$$ Do such $\frac{\p x^a}{\p\tilde x^{a'}}$ always exist? The answer turns out to be yes-- $g_{ab}$ is not degenerate, so one may diagonalize it and then rescale the eigenvalues. Sylvester's theorem states that if $g$ has $r$ positive eigenvalues, $s$ negative eigenvalues, then diagonalizing preserves this.%it's super unclear how this all applies, TBH. 

Therefore given a metric $g_{ab}$ that is non-degenerate, the inverse metric $g^{ab}$ can be defined such that 
\begin{equation}
g^{ab}g_{bc}=\delta^a_c,
\end{equation} where $\delta^a_c$ is the Kronecker delta. One may then use the metric and the inverse metric to raise and lower indices:
$V_b=g_{bc}V^c$ and $V^a = g^{ab}V_b$.

\textit{``There are more unknowns than there are knowns.''} A brief summary of this course.