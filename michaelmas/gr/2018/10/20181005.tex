Some course materials include:
\begin{itemize}
\item Sean Carroll, Spacetime and Geometry
\item Misner, Thorne, and Wheeler, Gravitation
\item Wald, General Relativity
\item Zee, Einstein Gravity in a Nutshell
\item Hawking and Ellis, ``The Large Scale Structure of Spacetime''
\end{itemize}

In Minkowski spacetime (flat space) we have spatial coordinates in $\RR^3$, the Cartesian coordinates $(x,y,z)$ and a time coordinate $t$. The line element (spacetime separation) is given by the metric
$$ds^2=-dt^2+dx^2+dy^2+dz^2.$$
$ds$ is the proper distance between $x$ and $x+dx$, $y$ and $y+dy$, $z$ and $z+dz$, and $t$ and $t+dt$. (We work in units where $c=1$. Note that the metric convention here is flipped from my QFT notes-- this is arbitrary.) Using the Einstein summation convention, the metric is usually written as $$ds^2=\eta_{\alpha\beta},$$ with $\eta_{\alpha\beta}$ the Minkowski space metric.

Let's recall from special relativity that we call separations with $ds^2>0$ ``spacelike,'' with $ds^2<0$ ``timelike,'' and $ds^2=0$ null (or occasionally lightlike).
\begin{defn}
The \term{chronological future} of a point $p$ is the set of all points that can be reached from $p$ along future directed timelike lines, and we call this $I^+(p)$. It is the interior of the future-directed light cone. Conversely we have the chronological past of $p$, $I^-(p)$, which is the interior of the past-directed light cone. We also have the \term{causal future} of $p$, which is the set of all points that can be reached from $p$ along future-directed timelike \emph{or} null lines, and we call this $J^+(p)$. Similarly we have the causal past, $J^-(p)$. Thus $J$ is the closure of $I$ and is the interior \emph{plus} the light cone itself.
\end{defn}

Let $x^a(\tau)$ be a curve in spacetime. Then the tangent vector to the curve is $u^a=\frac{dx^a}{d\tau}$. For timelike curves, $u^a u^b \eta_{ab}=-1 \iff \tau$ is the proper time along the curve. $\int_p^q d\tau = \Delta \tau$, so the integral of $d\tau$ along a curve from $p$ to $q$ yields the proper time interval, what a clock actually measures.

We also remark that Minkowski space has some very nice symmetries. Since $x,y,$ and $z$ do not appear explicitly in the metric, our spacetime is invariant under translations. It is also invariant under rotations in $\RR^3$. It would be nice to extend rotations to include the time coordinate $t$ as well-- this is exactly what a Lorentz transformation does.

Lorentz transformations in general involve time-- they are defined by the matrices $\Lambda$ which satisfy
$$\Lambda^T \eta \Lambda= \eta,$$
i.e. they preserve the inner product $\eta$ in Minkowski space, forming the group $O(3,1)$. Lorentz transformations consist of rotations in $\RR^3$ and boosts. This is equivalent to the property $R^T \delta R=\delta$, where $R$ is a rotation and all matrices $R$ satisfying this equation form the group $O(3)$.
The Lorentz boost in the $x$-direction given explicitly is
\begin{eqnarray*}
t'&=&\frac{t-vx}{\sqrt{1-v^2}}\\
x'&=&\frac{x-vt}{\sqrt{1-v^2}}\\
y'&=&y\\
z'&=&z
\end{eqnarray*}
Rather than constructing the (in general complicated) Lorentz transformation, it is often more convenient to rotate one's frame of reference in $\RR^3$ so the boost is in the new $x$-direction, perform the Lorentz boost, and then transform back:
$$R^T \Lambda R= \Lambda_R,$$
where $\Lambda_R$ is a new Lorentz transform.

\begin{defn}
The Lorentz transformations taken together form the \term{Lorentz group}. It satisfies identity, unique inverses (since $\det\Lambda \neq 0$), associativity (from associativity of matrix multiplication), and closure (perhaps good to prove this).\footnote{More precisely, we know that the determinant is nonzero since $-1=\det{\eta}=\det(\Lambda^T)\det(\eta)\det(\Lambda)=(-1)\det(\Lambda)^2\implies \det(\Lambda)=\pm 1$.}
\end{defn}

$\Lambda$ can include reflections in time or space. To avoid such complications, we sometimes refer to the \term{proper orthochronous Lorentz group,} i.e. to exclude space and time reversals. 
\begin{defn}
The Poincar\'e group is then the semidirect product of Lorentz transformations and translations. This is therefore the group of symmetries of Minkowski space.
\end{defn}
We have translations defined as
$$x^a\to {x^a}'=x^a+\Delta x^a$$
and also Lorentz transformations, with the property
$${(\Lambda^T)_a}^c \eta_{cd} {\Lambda^d}_b=\eta_{ab}.$$


\begin{defn}
We also have \term{contravariant vectors} (indices up) written $u^a$ and their corresponding \term{covariant} vectors $$u_a\equiv\eta_{ab}u^b,$$ where we have used the metric to lower an index. We can also raise indices using the inverse metric $\eta^{ab}$ (defined by $\eta^{ab}\eta_{bc}=\delta^a_c$). Thus
$$u^b=\eta^{ba}u_a.$$
\end{defn}

In general the Lorentz transformation of a contravariant vector is given by
$u^a\to {u^a}' = {\Lambda^a}_b u^b,$ where
$${\Lambda^a}_b =
\begin{pmatrix}
\gamma&-\gamma v &0 & 0\\
-\gamma v & \gamma & 0 & 0\\
0&0&1&0\\
0&0&0&1
\end{pmatrix}$$
where $\gamma$ is given in the usual way by $\gamma=\frac{1}{\sqrt{1-v^2}}$. For instance, $x^a$ is an example of a contravariant vector.

\begin{defn}
A scalar is an object which is invariant under a Lorentz transformation. We saw that a covariant vector transforms with right multiplication by the Lorentz transformation, whereas a contravariant vector transforms by left multiplication. More generally a tensor of type $(r,s)$ transforms with $r$ copies of the lorentz transformation on the $r$ up indices and $s$ copies of the lorentz transformation on the $s$ down indices, $\Lambda_1 \ldots \Lambda_r T^{1\to r}_{1\to s} \Lambda^1 \Lambda^s$.
\end{defn}