\usepackage[utf8]{inputenc}
\usepackage[margin=1in]{geometry}
\usepackage[usenames, dvipsnames]{xcolor}
\usepackage{graphicx}
\usepackage{mathtools}
\usepackage{amssymb}
\usepackage{amsthm}
\usepackage{fancyhdr}
\usepackage{adforn}
\usepackage{xparse}
\usepackage{tikz}
\usetikzlibrary{fadings}
%\usetikzlibrary{matrix, positioning, calc}
% Additional math macros that I want in both my notes and my psets
\usepackage[sc, noBBpl]{mathpazo}
\usepackage{mathrsfs}
\usepackage[T1]{fontenc}
\usepackage{calligra}
\usepackage{microtype}
\usepackage[all]{xy}
\usepackage{slashed}
\newcommand{\A}{\mathbb A}
\newcommand{\cat}{\mathsf}
\newcommand{\sC}{\cat C}
\newcommand{\sD}{\cat D}
\newcommand{\sS}{\cat S}
\newcommand{\sA}{\mathscr A}
\newcommand{\sF}{\mathscr F}
\newcommand{\sG}{\mathscr G}
\renewcommand{\P}{\mathbb P}
\newcommand{\cO}{\mathscr O}
\newcommand{\sI}{\mathscr I}
\DeclareMathOperator{\coker}{coker}
\renewcommand{\Im}{\operatorname{Im}}
\newcommand{\pt}{\mathrm{pt}}
\DeclareMathOperator{\Hom}{Hom}
\newcommand{\op}{^{\mathsf{op}}}
\newcommand{\Id}{\mathrm{Id}}
\DeclareMathOperator{\Mat}{Mat}
\newcommand{\m}{\mathfrak m}
%\newcommand{\p}{\mathfrak p}
\newcommand{\q}{\mathfrak q}
\DeclareMathOperator{\MSpec}{MSpec}
\DeclareMathOperator{\Spec}{Spec}
\newcommand{\Top}{\cat{Top}}
\newcommand{\Ring}{\cat{Ring}}
\newcommand{\Mod}{\cat{Mod}}
\DeclareMathOperator{\res}{res}
\newcommand{\Alg}{\cat{Alg}}
\newcommand{\Fun}{\cat{Fun}}
\newcommand{\AffSch}{\cat{AffSch}}
\newcommand{\Ab}{\cat{Ab}}
\DeclareMathOperator{\bl}{--}
\DeclareMathOperator{\Free}{Free}
\DeclareMathOperator{\For}{For}
\newcommand{\Set}{\cat{Set}}
\newcommand{\LocRing}{\cat{LocRing}}
\newcommand{\Grp}{\cat{Grp}}
\newcommand{\Sch}{\cat{Sch}}
\newcommand{\inHom}{\operatorname{\underline{\Hom}}}
\DeclareMathOperator{\Frac}{Frac}
\DeclareMathOperator{\Gal}{Gal}
\DeclareMathOperator{\Nil}{Nil}
\newcommand{\pre}{\sC^{\text{pre}}}
\newcommand{\sh}{_{\text{sh}}}
\newcommand{\G}{\mathbb G}
\DeclareMathOperator{\Proj}{Proj}
\newcommand{\sM}{\mathscr M}
\newcommand{\sV}{\mathscr V}
\newcommand{\fU}{\mathfrak U}
\newcommand{\GL}{\mathrm{GL}}
\DeclareMathOperator{\Sym}{Sym}
% http://tex.stackexchange.com/questions/141434/how-to-type-sheaf-hom
\DeclareMathOperator{\shom}{\mathscr{H}\text{\kern -4pt {\calligra\large om}}\,}
\newcommand{\sL}{\mathscr L}
\DeclareMathOperator{\QC}{QC}
\DeclareMathOperator{\Supp}{Supp}
\newcommand{\sN}{\mathscr N}
\DeclareMathOperator{\Ann}{Ann}
\DeclareMathOperator{\Der}{Der}
\newcommand{\ctcpx}[1]{(#1)^{\text{der}}}
\newcommand{\Dist}{\mathsf{Dist}}
\newcommand{\shdi}{\operatorname{Sh}_{\Dist}}
\DeclareMathOperator{\Sh}{Sh}
\newcommand{\shz}{\mathsf{Sh}_{\text{\rm Zar}}}
\DeclareMathOperator{\Gr}{Gr}
% Source: http://tug.org/pipermail/xy-pic/2001-July/000015.html
\newcommand{\pullbackcorner}[1][dr]{\save*!/#1+1.2pc/#1:(1,-1)@^{|-}\restore}
\newcommand{\pushoutcorner}[1][dr]{\save*!/#1-1.2pc/#1:(-1,1)@^{|-}\restore}
\newcommand{\TDel}{\mathrm{2\Delta}}
\DeclareMathOperator{\Bl}{B\ell}
\newcommand{\cR}{\mathcal R}
\newcommand{\cL}{\mathcal L}
\newcommand{\cH}{\mathcal H}
\newcommand{\refR}{\reflectbox{\(\cR\)}}

\renewcommand{\a}{\alpha}
\renewcommand{\b}{\beta}
%\newcommand{\e}{\epsilon}
\renewcommand{\l}{\lambda}
\renewcommand{\L}{\Lambda}
\newcommand{\g}{\gamma}
\newcommand{\s}{\sigma}
\newcommand{\z}{\zeta}
\newcommand{\RR}{\mathbb{R}}
\newcommand{\NN}{\mathbb{N}}
\newcommand{\QQ}{\mathbb{Q}}
\newcommand{\ZZ}{\mathbb{Z}}
\newcommand{\CC}{\mathbb{C}}
\newcommand{\cC}{\mathcal{C}}
\newcommand{\f}{\frac}
\newcommand{\p}{\partial}
\renewcommand{\P}[3][]{\f{\partial^{#1} #2}{\partial #3 ^{#1}}}
%\newcommand{\avg}[1]{\langle #1 \rangle}
\newcommand{\avg}[1]{\left< #1 \right>}
\newcommand{\?}{\overset{?}{=}}
\newcommand{\Int}{\int_{-\infty}^\infty}
\newcommand{\ket}[1]{\left| #1 \right>} % for Dirac bras
\newcommand{\bra}[1]{\left< #1 \right|} % for Dirac kets
\newcommand{\braket}[2]{\left< #1 \vphantom{#2} \right|
 \left. #2 \vphantom{#1} \right>} % for Dirac brackets
\newcommand{\pv}{\vec{p}}

\newcommand{\grad}[1]{\gv{\nabla} #1} % for gradient
\let\divsymb=\div % rename builtin command \div to \divsymb
\renewcommand{\div}[1]{\gv{\nabla} \cdot #1} % for divergence
\newcommand{\curl}[1]{\gv{\nabla} \times #1} % for curl
\renewcommand{\labelenumi}{(\alph{enumi})}
\let\vaccent=\v % rename builtin command \v{} to \vaccent{}
\renewcommand{\v}[1]{\ensuremath{\mathbf{#1}}}
\newcommand{\uv}[1]{\ensuremath{\mathbf{\hat{#1}}}} % for unit vector
\newcommand{\gv}[1]{\ensuremath{\mbox{\boldmath$ #1 $}}} 
% for vectors of Greek letters
\usepackage{hyperref}
\usepackage{siunitx}

\usepackage{tikz-feynman}
\tikzfeynmanset{compat=1.0.0}

% TODO fiddle with colors
\definecolor{newblue}{HTML}{1F98A6}
\definecolor{newred}{HTML}{D95448}
\definecolor{neworange}{HTML}{F29441}
\hypersetup{
	colorlinks,
	linkcolor=newred,
	citecolor=neworange,
	urlcolor=newblue!80!black,
}
\usepackage[all]{hypcap}
\pagestyle{plain}
\setcounter{tocdepth}{1}


\usepackage{titlesec}
\titleformat{\section}[frame]
  {\normalfont}
  {\filright
   \footnotesize
   \enspace Lecture \arabic{section}.\enspace}
  {8pt}
  {\Large\bfseries\filcenter}
\usepackage[dotinlabels]{titletoc}
\titlecontents{section}[1.5em]{}{\contentslabel{2.3em}}{\hspace*{-2.3em}}{\hfill\contentspage}

\renewcommand{\sectionmark}[1]{\markleft\thesection. #1}

\fancyhf{}
\fancyhead[RO,LE]{\small\thepage}
\fancyhead[LO]{\small\slshape\nouppercase{\leftmark}}
\fancyhead[RE]{\small\slshape Quantum Field Theory Lecture Notes}
\setlength{\headheight}{11.0pt}
\pagestyle{fancy}

\numberwithin{equation}{section}
\newcommand{\orbreak}{
\begin{center}
	\adforn{17}\;\(\cdot\)\;\adforn{18}
	\vspace{0.2cm}
\end{center}
}

\renewcommand{\labelitemi}{\(\circ\)}

% I wanted to allow one to reference parts of a thm/cor/etc. and have it print the thm number too, e.g. 29.2(1),
% but this isn't working right now. Probably the best way to do this would be to play around with enumitem to
% define a new enumerate-like counter and then just use that directly instead of enumerate in comp.

% This feels really wobbly, but so far it's working
\NewDocumentEnvironment{comp}{mm}{%
	\csname #1\endcsname\hfill
	\csname #2\endcsname
}{
	\csname end#2\endcsname
	\csname end#1\endcsname
}

% usage:
% \shortexact[f][g]{A}{B}{C},
%
%			 f    g
% for 0 -> A -> B -> C -> 0,
\DeclareDocumentCommand{\shortexact}{O{} O{} mmmm}{
\xymatrix{
	0\ar[r] & #3\ar[r]^-{#1} & #4\ar[r]^-{#2} & #5\ar[r] & 0#6
}}
% exactly the same, but for 0 -> A -> B -> C
\DeclareDocumentCommand{\leftexact}{O{} O{} mmmm}{
\xymatrix{
	0\ar[r] & #3\ar[r]^-{#1} & #4\ar[r]^-{#2} & #5 #6
}}
% ... and the same, for A -> B -> C -> 0
\DeclareDocumentCommand{\rightexact}{O{} O{} mmmm}{
\xymatrix{
	#3\ar[r]^-{#1} & #4\ar[r]^-{#2} & #5\ar[r] & 0#6
}}



% usage:
% X\dblarrow[r] & Y
%   f
% X => Y
%   g
\DeclareDocumentCommand{\dblarrow}{O{} O{} O{}}{
	\ar@<0.4ex>[#1]^-{#2}\ar@<-0.4ex>[#1]_-{#3}
}
% Note: it would be a useful exercise to figure out how to define this so it can be used as
% \dblarrow[r]^f_g

\everyentry={\displaystyle}

\newcommand{\N}{\mathbb N}
\newcommand{\Z}{\mathbb Z}
\newcommand{\Q}{\mathbb Q}
\newcommand{\R}{\mathbb R}
\newcommand{\C}{\mathbb C}
\newcommand{\F}{\mathbb F}
\newcommand{\vp}{\varphi}
\newcommand{\term}{\emph}
\renewcommand{\vec}[1]{\boldsymbol{\mathbf{#1}}}
\DeclarePairedDelimiter\paren{(}{)}
%\DeclarePairedDelimiter\ang{\langle}{\rangle}
\DeclarePairedDelimiter\abs{\lvert}{\rvert}
\DeclarePairedDelimiter\norm{\lVert}{\rVert}
\DeclarePairedDelimiter\bkt{[}{]}
\DeclarePairedDelimiter\set{\{}{\}}
% Swap paren* and paren, etc., so that the normal version resizes by default.
% Meanwhile, one can use \paren*[\Big]{...} to customize the size easily.
% It would be interesting to wrap this up into a custom \definedelimiter command...
\makeatletter
	\let\oldparen\paren
	\def\paren{\@ifstar{\oldparen}{\oldparen*}}
	\let\oldbkt\bkt
	\def\bkt{\@ifstar{\oldbkt}{\oldbkt*}}
\makeatother
\newcommand{\e}{\varepsilon}
\def\qedsymbol{{\small{\ensuremath{\boxtimes}}}}
\newcommand{\inj}{\hookrightarrow}
\newcommand{\surj}{\twoheadrightarrow}
\DeclareMathOperator{\id}{id}
\newcommand{\ud}{\,\mathrm{d}}
\renewcommand{\d}{\mathrm d}
\newcommand{\dfr}[2]{\frac{\mathrm d #1}{\mathrm d #2}}
\newcommand{\pfr}[2]{\frac{\partial #1}{\partial #2}}

%\catcode`\"=13
%\newcommand{"}[1]{^{(#1)}}
\newtheorem{thm}[equation]{Theorem}
\newtheorem*{thm*}{Theorem}
\newtheorem{lem}[equation]{Lemma}
\newtheorem*{lem*}{Lemma}
\newtheorem{cor}[equation]{Corollary}
\newtheorem{prop}[equation]{Proposition}
\newtheorem{obs}[equation]{Observation}
\theoremstyle{definition}
\newtheorem{ex}[equation]{Exercise}
\newtheorem{exm}[equation]{Example}
\newtheorem{defn}[equation]{Definition}
\newtheorem*{claim}{Claim}
\theoremstyle{remark}
\newtheorem*{rem}{Remark}
\newtheorem*{fct}{Fact}
\newtheorem*{note}{Note}
