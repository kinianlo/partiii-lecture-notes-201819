Today, we'll introduce spinors, the mathematical framework describing the behavior of fermions! We'll start to show explicitly why spin $1/2$ is different than spin $0$.\footnote{It's pretty cool to learn about this in Cambridge, where Dirac actually discovered the behavior of spin $1/2$ particles.}

Now, so far we've only considered scalar fields $\phi$. Under a Lorentz transformation, these transform as
\begin{align*}
    x^\mu & \to {x'}^\mu = {\Lambda^\mu}_\nu x^\nu\\
    \phi(x) & \to \phi'(x)=\phi(\Lambda^{-1}x).
\end{align*}
Most particles have an intrinsic angular momentum, which we call \term{spin}, and fields with spin have a non-trivial Lorentz transformation. For instance, spin $1$ particles (i.e. \term{vector fields}) come with an index $\mu$ and transform as
$$A^\mu(x)\to {A^\mu}'(x)={\Lambda^\mu}_\nu A^\nu(\Lambda^{-1} x).$$
In general fields can transform as $\phi^a \to {D^a}_b(\Lambda) \phi^b(\Lambda^{-1}x),$
where we say the ${D^a}_b$ form a representation of the Lorentz group. These might be familiar from \emph{Symmetries, Fields and Particles}, but to give a quick overview, a representation $D$ of a group $g$ is a map from that group to a space of linear transformations (usually taken to be matrices) which preserves the group multiplication. That is, it satisfies
\begin{gather*}
    D(\Lambda_1\Lambda_2)=D(\Lambda_1)D(\Lambda_2)\\
    D(\Lambda^{-1})=(D(\Lambda))^{-1}\\
    D(I)= I.
\end{gather*}
To find the representations, we look at the Lorentz algebra by considering infinitesimal Lorentz transformations. If we write
$${\Lambda^\mu}_\nu=\delta^\mu_\nu+\epsilon {\omega^\mu}_\nu+O(\epsilon^2),$$
then the property that $\Lambda$ preserves the inner product on four-vectors implies that $\omega_{\mu\nu}$ is a $4\times 4$ antisymmetric matrix. In particular this means it has $\frac{4\times 3}{2}=6$ independent components, corresponding to the three rotations and three Lorentz boosts.

We may introduce a basis of six $4\times 4$ matrices, which we will label by four indices
$$(M^{\rho\sigma})^{\mu\nu}=\eta^{\rho\mu}\eta^{\sigma\nu}-\eta^{\sigma\mu}\eta^{\rho\nu},$$
where these matrices are antisymmetric in $\rho,\sigma$ and in $\mu,\nu$. We take $\rho,\sigma$ to specify which matrix we are looking at. Lowering the index $\nu$, we take $\mu,\nu$ to specify the row and column respectively. Therefore
$$(M^{\rho\sigma})^\mu_\nu=\eta^{\rho\mu}\delta^\sigma_\nu-\eta^{\sigma\mu}\delta^\rho_\nu.$$
\begin{exm}
The basis vector ${(M^{01})^\mu}_\nu$ is given by
$${(M^{01})^\mu}_\nu=
\begin{pmatrix}
0&+1&0&0\\
+1&0&0&0\\
0&0&0&0\\
0&0&0&0
\end{pmatrix}.$$
This generates a boost in the $x^1$ direction (it mixes up $x^1$ and $t$).

Similarly, the basis vector ${(M^{12})^\mu}_\nu$ takes the form
$${(M^{12})^\mu}_\nu=
\begin{pmatrix}
0&0&0&0\\
0&0&-1&0\\
0&1&0&0\\
0&0&0&0
\end{pmatrix}.$$
This generates rotations in the $(x^1-x^2)$ plane.
\end{exm}
Note that when we lower $\nu$ in order to write the generators as matrices, the matrix may not explicitly look antisymmetric! We can now write
$$\omega^{\mu\nu}=\frac{1}{2}(\Omega_{\rho\sigma}M^{\rho\sigma})^\mu_\nu$$
where these $M$s are the generators of the group of Lorentz transformations and $\Omega$ is some set of antisymmetric parameters.
\begin{defn}
The \term{Lorentz algebra} is defined by the bracket
$$[M^{\rho\sigma},M^{\tau\nu}]=\eta^{\sigma\tau}M^{\rho\nu}-\eta^{\rho\tau}M^{\sigma\nu}+\eta^{\rho\nu}M^{\sigma\tau}-\eta^{\sigma\nu}M^{\rho\tau}.$$
\end{defn}

The spinor representation means that we search for other matrices satisfying the Lorentz algebra. 
\begin{defn}
We define the Clifford algebra (in any number of dimensions we like, though four is the most useful for our purposes) as a set of matrices $\gamma^\mu$ such that
$$\set{\gamma^\mu,\gamma^\nu}=2\eta^{\mu\nu} I,$$
where we have defined the anticommutator $\set{\gamma^\mu,\gamma^\nu}\equiv\gamma^\mu \gamma^\nu+\gamma^\nu\gamma^\mu$. In four dimensions, the $\gamma^\mu$ are a set of $4\times 4$ matrices with $\mu=0,1,2,3$.
\end{defn}

We need to find $4$ matrices which anticommute, and such that $(\gamma^i)^2=-I \forall i\in \set{1,2,3}$ and $(\gamma^0)^2=I$. The simplest representation is in terms of $4\times 4$ matrices. A common choice is the \term{chiral} or \term{Weyl representaion}, where we take
$$\gamma^0=\begin{pmatrix}
0&I_2\\
I_2 & 0
\end{pmatrix},
\gamma^i=\begin{pmatrix}
0&\sigma^i\\
-\sigma^i&0
\end{pmatrix},$$
where the $\sigma^i$ are the usual $2\times 2$ Pauli matrices. As a quick refresher, the Pauli matrices are
$$\sigma^1=\begin{pmatrix}
0&1\\1&0
\end{pmatrix},
\sigma^2=\begin{pmatrix}
0&-i\\i&0
\end{pmatrix},
\sigma^3=\begin{pmatrix}
1&0\\0&-1
\end{pmatrix}.$$
They satisfy the commutation and anticommutation relations
$$[\sigma^i,\sigma^j]=2i e^{ijk}\sigma^k\text{ and }\set{\sigma^i,\sigma^j}=2\delta^{ij}I_2.$$
Note that the $\gamma$ matrices under any similarity transformation $U\gamma^\mu U^{-1}$ (where $U$ is an invertible constant matrix) also forms an equally good basis.

We now define
$$S^{\rho\sigma}\equiv \frac{1}{4}[\gamma^\rho,\gamma^\sigma]=\frac{1}{2}\gamma^\rho \gamma^\sigma -\frac{1}{2} \eta^{\rho\sigma}$$
by the Clifford algebra. We'll make the following claims: first,
$$[S^{\mu\nu},\gamma^\rho]=\gamma^\mu \eta^{\nu\rho}-\gamma^\nu \eta^{\rho\mu}.$$
Second, using the previous claim and the definition of $S$, we can prove (e.g. on the example sheet) that $S$ satisfies the commutation relation
$$[S^{\rho\sigma},S^{\tau\nu}]=\eta^{\sigma\tau}S^{\rho\nu}-\eta^{\rho\tau}S^{\sigma\nu}+\eta^{\rho\nu}S^{\sigma\tau}-\eta^{\sigma\nu}S^{\rho\tau}.$$
But this is precisely the relations that the Lorentz group generators satisfy, and so $S$ provides a representation of the Lorentz algebra.\footnote{At this point, Professor Allanach made a slight digression to read from an interview with Dirac conducted by an USAmerican journalist. It's entertaining reading and can be found here: \url{http://sites.math.rutgers.edu/~greenfie/mill_courses/math421/int.html}}

We now introduce a four-component \term{Dirac spinor} $\psi_\alpha(x),\alpha\in\set{1,2,3,4}.$ The spinor then transforms under Lorentz transformations as
$$\psi^\alpha(x) \to {S[\Lambda]^\alpha}_\beta \psi^\beta(\Lambda^{-1}x).$$
Here,
$$S[\Lambda]=\exp \left(\frac{1}{2}\Omega_{\rho\sigma}S^{\rho\sigma}\right)\text{ and }\Lambda=\exp\left(\frac{1}{2}\Omega_{\rho\sigma} M^{\rho\sigma}\right)$$
are both $4\times 4$ matrices.

Is the spinor representation equivalent to the usual vector representation? No-- one can look at specific Lorentz transformations to see this. For instance, the rotations $i,j\in\set{1,2,3}$ give
\begin{align*}
    S^{ij}&= \frac{1}{4}\left[\gamma^i,\gamma^j\right]\\
    &=\left[\begin{pmatrix}
        0&\sigma^i\\
        -\sigma^i&0
    \end{pmatrix},
    \begin{pmatrix}
        0&\sigma^j\\
        -\sigma^j&0
    \end{pmatrix}\right]\\
    &=\frac{-i}{2} \epsilon^{ijk}\begin{pmatrix}
        \sigma^k&0\\
        0&\sigma^k
    \end{pmatrix}.
\end{align*}
If we write $\Omega_{ij}=-\epsilon_{ijk}\phi^k,$ where $\phi^k$ is a vector specifying a rotation axis, e.g. $\Omega_{12}=-\phi^3$.
Then
$$S[\Lambda]=\exp\left(\frac{1}{2}\Omega_{\rho\sigma}S^{\rho\sigma}\right)=\begin{pmatrix}
e^{i\gv\phi \cdot \gv \sigma/2}& 0\\
0&e^{i\gv\phi \cdot \gv \sigma/2}
\end{pmatrix}.$$
Therefore a rotation about the $x^3$ axis can be written as $\phi=(0,0,2\pi)$, and then 
$$S[\Lambda]=\begin{pmatrix}
e^{i \sigma^3 \pi}& 0\\
0&e^{i\sigma^3 \pi}
\end{pmatrix}.=-I_4.$$
Therefore a rotation of $2\pi$ takes $\psi_\alpha(x)\to -\psi_\alpha(x).$ This is different from the vector representation, where 
$$\Lambda=\exp\left(\frac{1}{2}\Omega_{\rho\sigma}M^{\rho\sigma}\right)=\exp \begin{pmatrix}
0&0&0&0\\0&0&2\pi&0\\
0&-2\pi&0&0\\
0&0&0&0
\end{pmatrix}=I_4,$$
as expected. So indeed spinors see the full $SU(2)$ rotational symmetry, and not just the $SO(3)$ symmetry of the ordinary Lorentz group.

What about boosts? Let us take
$$S^{0i}=\frac{1}{2}\begin{pmatrix}
-\sigma^i & 0\\
0&\sigma^i
\end{pmatrix}$$
and write our boost parameter $\Omega_{0i}=-\Omega_{i0}\equiv \chi_i.$ Then
$$S[\Lambda]=\begin{pmatrix}
e^{-\gv \chi \cdot \gv\sigma/2} & 0\\
0&e^{-\gv \chi \cdot \gv\sigma/2}
\end{pmatrix}.$$
For rotations, $S[\Lambda]$ is unitary since $S[\Lambda]S[\Lambda]^\dagger =I,$ but for boosts this is \emph{not} the case.

It turns out there are no finite-dimensional unitary representations of the Lorentz group: this is because the matrices
$$S[\Lambda]=\exp\left[\frac{1}{2}\Omega_{\rho\sigma} S^{\rho\sigma}\right]$$ are only unitary if the $S^{\mu\nu}$ are anti-hermitian, $(S^{\mu\nu})^\dagger=-S^{\mu\nu}.$ But
$$(S^{\mu\nu})^\dagger=-\frac{1}{4}[{\gamma^\mu}^\dagger,{\gamma^\nu}^\dagger]$$
can be anti-hermitian if all the $\gamma^\mu$s are either all hermitian or all anti-hermitian. However, this can't be arranged, since
$\gamma^0)^2=I\implies \gamma^0$ has real eigenvalues (and cannot be anti-hermitian), whereas $(\gamma^i)^2=-I\implies \gamma^i$ has purely imaginary eigenvalues, and therefore cannot be hermitian.