The make-up lecture that was supposed to take place on Monday was cancelled, so we'll have an extra lecture instead on Thursday at the regular time.

Last time, we started discussing Lorenz gauge, where
$$\p_\mu A^\mu=0.$$
We wrote down a new Lagrangian
$$\cL=-\frac{1}{4}F_{\mu\nu}F^{\mu\nu}-\frac{1}{2\alpha}(\p_\mu A^\mu)^2,$$
noting that this new theory does not have the gauge symmetry of the original, and additionally, $A_0, A_i$ are now all dynamical. Thus
$$\pi^0=\P{\cL}{\dot A_0}=-\frac{\p_\mu A^\mu}{\alpha}$$
(which is now nonzero) and
$$\pi^i = \P{\cL}{\dot A_i}=-\dot A^i +\p^i A^0$$
as before.

If we now apply the commutation relations
$$[A_\mu(\vec x),A_\nu (\vec y)]=[\pi_\mu(\vec x),\pi_\nu (\vec y)]=0$$
and
$$[A_\mu(\vec x),\pi_\nu(\vec y)]=-i \delta^3(\vec x- \vec y)\eta_{\mu\nu},$$
we can expand out the field in terms of polarization vectors $\epsilon_\mu^{(\lambda)}$ and creation and annihilation operators $a_{\vec p}^\lambda, a_{\vec p}^{\lambda\dagger}.$
$A_\mu$ takes the form
$$A_\mu(\vec x) =\int \frac{d^3p}{(2\pi)^3} \frac{1}{\sqrt{2|\vec p|}} \sum_{\lambda=0}^3 \left(\epsilon_\mu^{(\lambda)}(\vec p) a_{\vec p}^\lambda e^{i \vec p \cdot \vec x}+\epsilon_\mu^{(\lambda)}{}^* (\vec p) a_{\vec p}^{\lambda \dagger} e^{-i\vec p \cdot \vec x}\right).$$

Note that the polarization vectors have a Lorentz index $\mu$ since $A_\mu$ must transform like a vector field. WLOG we may pick $\epsilon_\mu^{(0)}$ to be timelike and $\epsilon_\mu^{(i)}$ to be spacelike, with normalization
$$\epsilon^\lambda \cdot \epsilon^{\lambda'}=\eta^{\lambda \lambda'}.$$
Moreover we shall choose $\epsilon_\mu^{(1)},\epsilon_\mu^{(2)}$ to be transverse polarizations, i.e. $\epsilon^{(1)}\cdot p = 0$ and $\epsilon^{(2)}\cdot p = 0$. We now choose $\epsilon^{(3)}$ to be the longitudinal polarization. For a photon traveling in the $x^3$ direction, the momentum is simply $p^\mu=|\vec p|(1,0,0,1)$, so the polarization vectors take the simple form
$$\epsilon^{(0)}=\begin{pmatrix}
1\\0\\0\\0\end{pmatrix},\epsilon^{(1)}=\begin{pmatrix}
0\\1\\0\\0\end{pmatrix},\epsilon^{(2)}=\begin{pmatrix}
0\\0\\1\\0\end{pmatrix},\epsilon^{(3)}=\begin{pmatrix}
0\\0\\0\\1\end{pmatrix}$$
and we can Lorentz boost or rotate to get photons traveling in other directions.

We also write the conjugate momentum in terms of creating and annihilation operators.
$$\pi^\mu(\vec x) =\int \frac{d^3p}{(2\pi)^3} \sqrt{\frac{|\vec p|}{2}} \sum_{\lambda=0}^3 \left((\epsilon^{(\lambda)}(\vec p))^{\mu} a_{\vec p}^\lambda e^{i \vec p \cdot \vec x}-(\epsilon^{(\lambda)} (\vec p))^{*\mu} a_{\vec p}^{\lambda \dagger} e^{-i\vec p \cdot \vec x}\right).$$
Now we get the following commutation relations:
$$[a_{\vec p}^\lambda,a_{\vec q}^{\lambda'}]=[a_{\vec p}^{\lambda \dagger},a_{\vec q}^{\lambda' \dagger}]=0$$
and
$$[a_{\vec p}^\lambda, a_{\vec q}^{\lambda' \dagger}]=-\eta^{\lambda \lambda'} (2\pi)^3 \delta^3(\vec p-\vec q).$$
The minus sign in front of the delta function may seem okay for for $\lambda=1,2,3$ but a bit strange for the timelike polarization. Somehow, timelike $\gamma$s are different.

We define a ground state $\ket{0}$ by
$$a_{\vec p}^\lambda \ket{0}=0$$ as usual, and then the various momentum states are
$$\ket{\vec p,\lambda}=a_{\vec p}^{\lambda \dagger}\ket{0}.$$
Now, this is totally fine for $\lambda=1,2,3$ but if we take $\lambda=0$, we get
$$\braket{p,\lambda=0}{q,\lambda=0}=\bra{0}a_{\vec p}^0 a_{\vec q}^{0\dagger} \ket{0}=-(2\pi)^3 \delta^3(\vec p-q),$$
which appears to be a state of negative norm. Now, a Hilbert space with negative norm states is usually problematic-- in particular, the probabilistic interpretation of QM goes out the window. In our case, the constraint equation comes to the rescue.

Let us switch to the Heisenberg picture and see what becomes of this polarization.
\begin{enumerate}
    \item Initially, we might think we could just impose the gauge condition. But note that $\p_\mu A^\mu=0$ doesn't work since $\pi^0 = -\p_\mu A^\mu$, so the commutation relationships cannot be obeyed (i.e. if $\pi^0$ vanishes then all its commutators are zero).
    \item We could impose a condition on Hilbert space, such that for physical states $\ket{\psi}$ we have $\p_\mu A^\mu\ket{\psi}=0$. But this is too strong. For suppose we split up $A_\mu$ into to parts,
    $$A_\mu^+(\vec x)=\int \frac{d^3p}{(2\pi)^3} \frac{1}{\sqrt{2|\vec p|}} \sum_{\lambda=0}^3 \epsilon_\mu^{(\lambda)}(\vec p) a_{\vec p}^\lambda e^{i \vec p \cdot \vec x}$$
    and
    $$A_\mu^-=\int \frac{d^3p}{(2\pi)^3} \frac{1}{\sqrt{2|\vec p|}} \sum_{\lambda=0}^3 \epsilon_\mu^{(\lambda)}{}^* (\vec p) a_{\vec p}^{\lambda \dagger} e^{-i\vec p \cdot \vec x}.$$
    Then we see that $\p_\mu A^{\mu+}\ket{0}=0$, which is okay, but $\p_\mu A^{\mu-}\ket{0}\neq 0,$ which tells us that $\ket{0}$ is not physical. Therefore this doesn't work.
    \item Finally, let us say that physical states $\ket{\psi}$ are defined by
    \begin{equation}\label{gbcondition}
        \p_\mu A^{+\mu}(\vec x)\ket{\psi}=0
    \end{equation}
    so that
    $$\bra{\psi'}\p_\mu A^\mu \ket{\psi}=0$$
    for all physical states $\ket{\psi},\ket{\psi'}$. Eqn. \ref{gbcondition} is known as the \term{Gupta-Bleuler condition}.
\end{enumerate}
By the linearity of \ref{gbcondition}, we see that the physical states $\set{\ket{\psi}}$ span a Hilbert space. Moreover, we can decompose a generic state into its transverse components $\ket{\psi_T}$ and its timelike (and/or longitudinal) components $\ket{\phi}$:
$$\ket{\psi}=\ket{\psi_T}\ket{\phi}.$$
Then
$$\p_\mu A^{+\mu}\ket{\psi}=0 \iff (a_{\vec k}^3- a_{\vec k}^0)\ket{\phi}=0$$
Check this! This means that physical states contain timelike/longitudinal pairs only. That is,
$$\ket{\phi}=\sum_{n=0}^\infty C_n \ket{\phi_n}$$
where $n$ indexes over $n$ timelike/longitudinal pairs in general.%maybe check this

Therefore
$$\braket{\phi_m}{\phi_n}=\delta_{m0}\delta_{n0},$$ so states with transverse-longitudinal (TL) pairs have zero norm. These zero norm states are treated as an equivalence class-- two states which differ only in the TL pairs are treated as physically equivalent.

This only makes sense if observables don't depend on $\ket{\phi_n}$, e.g. if our Hamiltonian is
$$H=\int \frac{d^3p}{(2\pi)^3} |\vec p| \left(\sum_{i=1}^3 a_{\vec p}^{i\dagger} a_{\vec p}^i - a_{\vec p}^{0\dagger} a_{\vec p}^0\right),$$
then since $$(a_{\vec k}^3 -a_{\vec k}^0)\ket{\psi}=0 \implies \bra{\psi} a_{\vec p}^{3\dagger} a_{\vec p}^3 \ket{\psi}=\bra{\psi} a_{\vec p}^{0\dagger} a_{\vec p}^0 \ket{\psi}.$$
That is, as long as timelike and longitudinal pieces cancel in the Hamiltonian $H$, we only get physical contributions from transverse states.
In general this cancellation works for any gauge-invariant operator evaluated on physical states.

Now it's time for us to write down the photon propagator! It is
$$\bra{0}T[A_\mu(\vec x)A_\nu (\vec y)]\ket{0}= \int \frac{d^4p}{(2\pi)^4} \frac{-i}{p^2+i\epsilon} \left[\eta_{\mu\nu}+(\alpha-1)\frac{p_\mu p_\nu}{p^2}\right]e^{-ip\cdot (x-y)}$$
for a general gauge $\alpha$. Note that in Feynman gauge ($\alpha=1$), the propagator takes a particularly simple form-- we just get the $\frac{-i \eta_{\mu\nu}}{p^2+i\epsilon}$ term. One can check that if we do any physical calculation in full generality leaving the $\alpha$ in, there are no $\alpha$s in the final result.

Can we introduce interactions and sources into our theory? Sure we can. Let's first write down the Maxwell Lagrangian with a source,
$$\cL=-\frac{1}{4}F_{\mu\nu}F^{\mu\nu}-j^\mu A_\mu.$$
The equations of motion are $\p_\mu F^{\mu\nu}=j^\nu$, and we see that this implies
$$\p_\nu j^\nu =\p_\nu \p_\mu F^{\mu\nu}=0,$$
so $j^\nu$ is a conserved current. Now the Dirac Lagrangian tells us that our theory of spin $1/2$ fermions,
$$\cL_D=\bar \psi(i\slashed{\p}-m)\psi,$$
has an internal symmetry $\psi\to e^{-i\alpha}\psi, \bar \psi \to e^{i\alpha}\bar \psi$ with $\alpha\in \RR$. This yields a current $j^\mu=\bar \psi \gamma^\mu \psi$. So we take this current from the Dirac Lagrangian and put it straight into the Maxwell Lagrangian to couple our photon to fermions.

Doing so yields the Lagrangian for quantum electrodynamics,
$$L_{QED}=-\frac{1}{4}F_{\mu\nu}F^{\mu\nu}+\bar \psi(i\slashed{\p}-m)\psi - e\bar \psi_\alpha \gamma^\mu_{\alpha\beta} A_\mu \psi_\beta.$$
Here, $\alpha,\beta$ are spinor indices and $\mu$ is a Lorentz index. $e$ is a coupling constant determining the strength of the coupling of the photon to our fermion (e.g. an electron). We therefore have the kinetic terms describing a massless spin one particle in $F_{\mu\nu}$, the Dirac kinetic terms for a massive spin $1/2$ particle in $\bar \psi(i\slashed{\p}-m)\psi$, and a coupling term which tells us that (as we are well aware) photons and electrons can interact.