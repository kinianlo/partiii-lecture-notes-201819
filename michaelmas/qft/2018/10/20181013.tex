We previously found that we could write the field momentum operator (not the conjugate momentum!) as
$$\vec{P}=-\int \pi(x) \grad \phi(\vec{x}) d^3x = \int \frac{d^3p}{(2\pi)^3} \vec{ p} a_{\vec p}^\dagger a_{\vec p}.$$
We could also act on our momentum eigenstates with the equivalent of the angular momentum operator $J^i$, and what we find is that 
$$J^i\ket{\vec{p}}=0,$$
so the scalar field theory represents a spin $0$ (scalar) particle.

In general we could imagine cooking up the multi-particle state
$$\ket{\pv_1,\pv_2,\ldots,\pv_n}=a_{\pv_1}^\dagger a_{\pv_2}^\dagger \ldots a_{\pv_n}^\dagger \ket{0}.$$
But it follows that
$$\ket{\pv, \vec{q}}=\ket{\vec{q},\vec p},$$
since the creation operators for different momenta commute, $[a_{\pv_1}^\dagger, a_{\pv_2}^\dagger]=0$. So our states are symmetric under interchange, which means these particles are bosons. The full Hilbert space is spanned by
$$\ket{0},a_{\pv}^\dagger \ket{0}, a_{\pv_1}^\dagger a_{\pv_2}^\dagger\ket{0}, \ldots$$
and this space of states is called \term{Fock space}.

If we use the number operator
$$N\equiv \int \frac{d^3p}{(2\pi)^3} a_{\pv}^\dagger a_{\pv}$$
which counts the number of particles in a state, we find%
    \footnote{The calculation is brief. Consider $N\ket{\pv_1, \ldots ,\pv_n}$. We rewrite the multi-particle state as $a_{\pv_1}^\dagger a_{\pv_2}^\dagger \ldots a_{\pv_n}^\dagger \ket{0}$, and then
    %
    \begin{align*}
        N\ket{\pv_1, \ldots ,\pv_n}&=\int \frac{d^3p}{(2\pi)^3} a_{\vec p}^\dagger a_{\vec p} a_{\pv_1}^\dagger a_{\pv_2}^\dagger \ldots a_{\pv_n}^\dagger \ket{0}\\
        &=\int \frac{d^3p}{(2\pi)^3} a_{\vec p}^\dagger (2\pi^3 \delta(\vec p - \vec p_1)+a_{\vec p_1} a_{\vec p})a_{\pv_2}^\dagger \ldots a_{\pv_n}^\dagger \ket{0}\\
        &=\ket{\vec p_1, \ldots,\vec p_n}+\int \frac{d^3p}{(2\pi)^3} a_{\vec p}^\dagger  a_{\pv_1}^\dagger a_{\vec p} a_{\pv_2}^\dagger \ldots a_{\pv_n}^\dagger \ket{0}\\
        &=\ket{\vec p_1, \ldots,\vec p_n}+a_{\pv_1}^\dagger N \ket{\vec p_2, \ldots, \vec p_n},
    \end{align*}
    %
    so proceeding by induction we see that for each $a_{\vec p_i}^\dagger$ we commute through, we pick up a copy of $\ket{\vec p_1,\ldots, \vec p_n}$. When the $a_{\vec p}$ has commuted all the way to the vacuum state $\ket{0}$, it simply annihilates it, leaving behind $n$ copies of the initial state $\ket{\vec p_1,\ldots, \vec p_n}$.}
$$N\ket{\pv_1, \ldots ,\pv_n}=n \ket{\pv_1, \ldots ,\pv_n}.$$
But using the commutation relations, it's easy to check that%
    \footnote{Leaving out the integral and normalization factors, the Hamiltonian is $a_{\vec p} a_{\vec p}^\dagger$, and our number operator is similarly $a_{\vec q}^\dagger a_{\vec q}.$ It follows that 
    \begin{align*}
        [N,H]&\sim a_{\vec q}^\dagger a_{\vec q}a_{\vec p} a_{\vec p}^\dagger -a_{\vec p} a_{\vec p}^\dagger a_{\vec q}^\dagger a_{\vec q}\\
        &=[a_{\vec q}^\dagger, a_{\vec p}]a_{\vec q} a_{\vec p}^\dagger+a_{\vec p} a_{\vec q}^\dagger a_{\vec q} a_{\vec p}^\dagger -a_{\vec p} a_{\vec p}^\dagger a_{\vec q}^\dagger a_{\vec q}\\
        &=(-(2\pi)^3 \delta^3(\vec p-\vec q))a_{\vec q} a_{\vec p}^\dagger +a_{\vec p} a_{\vec q}^\dagger [a_{\vec q},a_{\vec p}^\dagger] +a_{\vec p} a_{\vec p}^\dagger a_{\vec q}^\dagger a_{\vec q}-a_{\vec p} a_{\vec p}^\dagger a_{\vec q}^\dagger a_{\vec q}\\
        &=-(2\pi)^3 \delta^3(\vec p-\vec q) a_{\vec q} a_{\vec p}^\dagger+(2\pi)^3 \delta^3(\vec q-\vec p) a_{\vec p} a_{\vec q}^\dagger.
    \end{align*}
    If we put the integrals back in and integrate over $d^3q$, the delta functions set $\vec q = \vec p$ and therefore $[N,H]\sim -a_{\vec p} a_{\vec p}^\dagger +a_{\vec p} a_{\vec p}^\dagger =0$.}
$$[N,H]=0,$$
which means that the number of particles in a given state is conserved in the free theory. Crucially, this is not true once we add interactions.

Let's also note that our momentum eigenstates are \emph{not} localized in space. We can describe a spatially localized state by a Fourier transform,
$$\ket{\vec x} = \int \frac{d^3 p}{(2\pi)^3} e^{-i \vec p \cdot \vec x} \ket{\vec p}.$$
More generally we describe a wavepacket partially localized in position and momentum space by a Fourier integral of the form
$$\ket{\psi} = \int\frac{d^3p}{(2\pi)^3} e^{-i\vec p \cdot \vec x} \psi(\vec p)\ket{\pv}.$$
Note that neither $\ket{\vec x}$ nor $\ket{\psi}$ are eigenstates of the Hamiltonian like in QM.

We consider now relativistic normalization. We define the vacuum such that $\braket{0}{0}=1$, which certainly must be Lorentz invariant ($1$ is just a number). So in general our momentum eigenstates have the inner product
$$\braket{\vec p}{\vec q}=\bra{0}[a_{\pv},a_{\vec q}^\dagger] \ket{0}=(2\pi)^3 \delta^3(\vec p - \vec q).$$
Is this quantity Lorentz invariant? Under the Lorentz transformation, four-momenta transform as $$p^\mu \to {p'}^\mu=\Lambda^\mu{}_\nu p^\nu.$$
We want the momentum eigenstates $\ket{\vec p}, \ket{\vec p'}$ to be related by a unitary transformation so that the inner product $\braket{\vec p}{\vec q}$ is Lorentz invariant (i.e. $\braket{\vec p}{\vec q}\to \braket{\vec p'}{\vec q'}=\bra{\pv}U(\Lambda)^\dagger U(\Lambda)\ket{\vec q})=\braket{\vec p}{\vec q}$ by unitarity). It turns out the normalization we've chosen is not quite right.

%To derive the correctly normalized states, we'll need to look at a Lorentz invariant object, e.g. the identity operator on 1-particle momentum eigenstates.
%$$1=\int \frac{d^3p}{(2\pi)^3} \ket{\pv}\bra{\pv}.$$
%Either half of this (i.e. the $d^3p$ part and the $\ket{\pv}\bra{\pv}$ part) is not LI, but the whole thing is, since it's equal to a scalar.

%How do we prove this? We start by claiming that
Let us begin by claiming that
$$\int \frac{d^3p}{2E_{\pv}}$$ is Lorentz invariant. 
\begin{proof}
First note that the integration measure $\int d^4p$ is Lorentz invariant, since $\Lambda \in SO(1,3)$ (i.e. $\det \Lambda=1$). Therefore the factor of $\det \Lambda$ we would normally pick up from doing the coordinate transformation is just $1$, so the four-volume element is Lorentz invariant, $\int d^4p = \int d^4 p'$. It's also true that the quantity $p_0^2= \vec{p}^2+m^2$ is Lorentz invariant (in particular, it expresses the length of a four-vector $p_\mu p^\mu = m^2$). The solutions for $p_0$ have two branches, positive and negative:
$$p_0 = \pm \sqrt{\pv^2+m^2}.$$
But our choice of branch is also Lorentz invariant (we can't go from the positive to negative solutions via Lorentz transformation). This means that $p_0^2-\vec p^2 -m^2, p_0>0$ will be a Lorentz invariant quantity, and will remain so even if we put it inside, say, a delta function. Combining the last few facts, we find that
$$\int d^4 p \delta(p_0^2-\pv^2-m^2)|_{p_0 >0} = \int \left.\frac{d^3 p}{2p_0}\right|_{p_0=E_p}=\int \frac{d^3p}{2E_p}$$
is Lorentz invariant, where we have used the fact that $$\delta(g(x))=\sum_{x_i\text{ roots of g}} \frac{\delta(x-x_i)}{|g'(x_i)|}$$
to perform the $dp_0$ integral.%\footnote{To see why this is true, consider Taylor expanding $\delta(g(x))$ around the roots of $g(x)$ to leading order. For instance, if $g(x_0)=0$ then near $x_0$, we have $g(x)\sim g'(x_0)(x-x_0)+O(x^2)$. Thus $\int dx \delta(g(x)) =\int dx \delta(g'(x_0)x -g'(x_0) x_0)=\int dx' \frac{\delta(x'-g'(x_0)x_0)}{g'(x_0)}$}
\end{proof}

We make the next claim: $2E_p \delta^3(\pv - \vec q)$ is the Lorentz invariant version of a $\delta$-function. \begin{proof}
As we just showed, $\int d^3p/2E_p$ is Lorentz invariant. It's also trivial to compute that
$$\int \frac{d^3 p}{2E_p} 2E_p \delta^3(\pv - \vec q)=1.$$
But since $\int d^3p/2E_p$ is Lorentz invariant and the RHS of the equation is certainly Lorentz invariant, it follows that $2E_p\delta^3(\pv-\vec q)$ must also be Lorentz invariant.
\end{proof}

We therefore learn that the correctly normalized states are
$$\ket{p}\equiv \sqrt{2E_p}\ket{\pv} = \sqrt{2 E_p}a_{\pv}^\dagger \ket{0},$$
(where $p$ is now the four-vector $p$, not the three-vector $\pv$) so that these momentum states have the Lorentz invariant inner product
$$\braket{p}{q}=(2\pi)^3 2\sqrt{E_p E_q} \delta^3(\pv - \vec q).$$
Note that in the basis of the old three-momentum eigenstates, we could have written the one-particle identity operator as an integral,
$$1=\int \frac{d^3p}{(2\pi)^3} \ket{\pv}\bra{\pv}.$$
We can now rewrite the 1-particle identity operator%
    \footnote{To see this really is the identity, let's act on the normalized $\ket{q}$. It's basically a one-liner:
    $$\int \frac{d^3p}{2E_p(2\pi)^3}\ket{p}\bra{p}\ket{q}=\int\frac{d^3p}{2E_p(2\pi)^3}\ket{p}\left[(2\pi)^3 2\sqrt{E_pE_q}\delta^3(\vec p-\vec q)\right]=\ket{q},$$ since the delta function makes the integral trivial by setting $\vec p = \vec q$.}
as an integral over the normalized states,
$$1=\int \frac{d^3p}{2E_p(2\pi)^3} \ket{p}\bra{p}.$$


\subsection*{Free $\CC$ scalar field} We could also look at a free complex scalar field $\psi$, with Lagrangian
$$\cL = \p_\mu \psi^* \p^\mu \psi - \mu^2 \psi^* \psi.$$
We can compute the Euler-Lagrange equations varying $\psi,\psi^*$ separately to find
$$\p_\mu \p^\mu \psi + \mu^2 \psi=0\text{ and }\p_\mu \p^\mu \psi^*+\mu^2 \psi^*=0,$$
where the second equation is simply the complex conjugate of the first.
Now we ought to write our field as a sum of two \emph{different} creation and annihilation operators:
$$\psi(\vec x)=\int \frac{d^3p}{(2\pi)^3}\frac{1}{\sqrt{2E_p}} (b_{\pv} e^{i \pv \cdot \vec x} + c_{\pv}^\dagger e^{-i \pv \cdot x})$$
and similarly
$$\psi^\dagger (\vec x)=\int \frac{d^3p}{(2\pi)^3} \frac{1}{\sqrt{2E_p}} (b_{\pv}^\dagger e^{-i \pv \cdot \vec x} + c_{\pv} e^{+i \pv \cdot x})$$
so that the conjugate momentum to the field $\psi$ is%
    \footnote{To actually derive this expression, note that the classical conjugate momentum to $\psi$ is $\P{\cL}{\dot \psi}=\dot \psi^*$. These fields as we've defined them only depend on space through $\vec x$, but when we add back in time dependence, a time derivative of a field will bring down factors of $\pm iE_{\vec p}$. This is more obvious when we write our fields as integrals over $e^{\pm ip \cdot x}$, where $p$ and $x$ are four-vectors with $p_0=E_{\vec p}$ and $x_0=t$. In the next lecture, we'll show that (for example) the field $\psi$ can be rewritten as
    \begin{equation*}
        \psi(\vec x,t) = \int \frac{d^3p}{(2\pi)^3} \frac{1}{\sqrt{2E_p}} \left( b_{\pv} e^{-i p\cdot x}+c_{\pv}^\dagger e^{+i p\cdot x}\right),
    \end{equation*}
    so time derivatives will as promised produce a factor of $-iE_p$ for the $b_{\vec p}$ term and a factor of $+iE_p$ for the $c_{\vec p}^\dagger$ term.
    The signs in the exponents are correct here, since we're working in the mostly minus convention. For now, we assert that this is the correct conjugate momentum by fiat.}
$$\pi(\vec x)=\int \frac{d^3p}{(2\pi)^3} i\sqrt{\frac{E_p}{2}} (b_{\pv}^\dagger e^{-i \pv \cdot \vec x}-c_{\pv} e^{i \pv \cdot \vec x}).$$
The conjugate momentum to $\psi^\dagger$ is equivalently $\pi^\dagger$. The commutation relations are then%
    \footnote{Stated as an exercise in class. This computation is longer, see non-lectured aside.}
$$[\psi(\vec x), \pi(\vec y)]=i\delta^3(\vec x - \vec y), \quad 
[\psi(\vec x),\psi^\dagger(\vec y)]=0$$
$$\implies [b_{\pv},b_{\vec q}^\dagger]=(2\pi)^3 \delta^3 (\pv - \vec q)=[c_{\pv},c_{\vec q}^\dagger].$$
The interpretation of these equations is that different types of particle are created by the $b_{\pv}^\dagger$ and $c_{\pv}^\dagger$ operators. They are both spin $0$ and of mass $\mu$, so we should interpret them as a particle-antiparticle pair. This doesn't work for electrons, which have spin $1/2$ and therefore require a more sophisticated spinor treatment, but it would describe something like a charged pion.

Indeed, if we compute the conserved charges in this theory by applying Noether's theorem, we get a conserved charge of the form
$Q=i\int d^3 x \, (\dot \psi^* \psi - \psi^* \dot \psi)$ or equivalently in terms of the conjugate momentum (since $\pi = \P{\cL}{\dot \psi}=\dot \psi^*$)
$$Q=i\int d^3 x [\pi \psi - \psi^\dagger \pi^\dagger].$$
After normal ordering (exercise) one can write
$$Q=\int \frac{d^3 p}{(2\pi)^3} (c_{\pv}^\dagger c_{\pv}- b_{\pv}^\dagger b_{\pv})=N_c-N_b,$$
which shows that our conserved quantity has the interpretation of particle number (counting antiparticles as $-1$).

Since there are two real scalar fields in this theory, the Hamiltonian for this theory takes the form
$$H=\int \frac{d^3p}{(2\pi)^3} E_p(b_{\pv}^\dagger b_{\pv}+c_{\pv}^\dagger c_{\pv}).$$
As an exercise one can check that $[Q,H]=0$ using the commutation relations,%
    \footnote{Leaving off the integrals, we have
    \begin{equation*}
        [Q,H]\sim [c_{\vec p}^\dagger c_{\vec p}, b_{\vec q}^\dagger b_{\vec q}]+[c_{\vec p}^\dagger c_{\vec p}, c_{\vec q}^\dagger c_{\vec q}]-[b_{\vec p}^\dagger b_{\vec p},b_{\vec q}^\dagger b_{\vec q}]- [b_{\vec p}^\dagger b_{\vec p}, c_{\vec q}^\dagger c_{\vec q}].
    \end{equation*}
    The commutators of $b$s and $c$s are zero since $b$s and $c$s always commute. The other two terms with only $b$s or only $c$s must cancel since $b$ and $c$ have the same commutation relations, so any commutators of $c$s and $c^\dagger$s will be equal to those same commutators with $c$s replaced by $b$s and $c^\dagger$s replaced by $b^\dagger$s everywhere. We conclude that $[Q,H]=0.$
    }
and therefore $Q$ is conserved. This is also true in the interacting theory. $N_c,N_b$ are individually conserved in the free theory, but in the interacting theory they aren't-- instead, they can be created and destroyed in particle-antiparticle pairs so that $N_c-N_b$ is constant.

\subsection*{Non-lectured aside: commutation relations and normal ordering}
First, let's derive the commutation relations for our new creation and annihilation operators. From the field commutation relations, we know that
\begin{equation*}
    [\psi(\vec x),\pi(\vec y)]=i\delta^3(\vec x-\vec y),
\end{equation*}
and if we write out this commutator explicitly in terms of the creation and annihilation operators, we find that it is
\begin{multline}\label{psipicommutator}
    [\psi(\vec x),\pi(\vec y)]=\frac{i}{2} \int \frac{d^3p\, d^3q}{(2\pi)^6} 
    \left([b_{\vec p}, b_{\vec q}^\dagger] e^{i(\vec p \cdot \vec x -\vec q \cdot \vec y)} - [b_{\vec p},c_{\vec q}] e^{i(\vec p \cdot \vec x + \vec q \cdot \vec y)}\right.\\
    \left.+[c_{\vec p}^\dagger, b_{\vec q}^\dagger] e^{-i(\vec p \cdot \vec x + \vec q \cdot \vec y)} - [c_{\vec p}^\dagger,c_{\vec q}] e^{-i(\vec p\cdot \vec x - \vec q \cdot \vec y)}\right).
\end{multline}
Proving that the field relations hold given the creation and annihilation commutation relations is easy-- if we know that $[b_{\vec p},b_{\vec q}^\dagger]=[c_{\vec p},c_{\vec q}^\dagger]=(2\pi)^3\delta^3(\vec p-\vec q)$ and all other commutators are zero, then Eqn. \ref{psipicommutator} reduces to
\begin{align*}
    [\psi(\vec x),\pi(\vec y)]&=\frac{i}{2} \int \frac{d^3p\, d^3q}{(2\pi)^3} 
    \left(\delta^3(\vec p-\vec q) e^{i(\vec p \cdot \vec x -\vec q \cdot \vec y)}
    +\delta^3(\vec q-\vec p) e^{-i(\vec p\cdot \vec x - \vec q \cdot \vec y)}\right)\\
    &=i \int\frac{d^3p}{(2\pi)^3} e^{i\vec p\cdot(\vec x -\vec y)}\\
    &=i \delta^3(\vec x- \vec y).
\end{align*}
Proving the other direction ($b,c$ commutation relations given the field relations) takes a little more work. We also know that the commutator of two different fields vanishes,
\begin{equation*}
    [\psi(\vec x),\psi^\dagger(\vec y)]=0.
\end{equation*}
Computing this commutator, we find that
\begin{multline}
    [\psi(\vec x),\psi^\dagger(\vec y)]=\int \frac{d^3p\,d^3q}{(2\pi)^6}\frac{1}{2\sqrt{E_{\vec p} E_{\vec q}}}
    \left([b_{\vec p}, b_{\vec q}^\dagger] e^{i(\vec p \cdot \vec x -\vec q \cdot \vec y)} + [b_{\vec p},c_{\vec q}] e^{i(\vec p \cdot \vec x + \vec q \cdot \vec y)}\right.\\
    \left.+ [c_{\vec p}^\dagger, b_{\vec q}^\dagger] e^{-i(\vec p \cdot \vec x + \vec q \cdot \vec y)} + [c_{\vec p}^\dagger,c_{\vec q}] e^{-i(\vec p\cdot \vec x - \vec q \cdot \vec y)}\right).
\end{multline}
Since this integral is identically zero, the integrand must vanish. The factor $\frac{1}{2\sqrt{E_{\vec p} E_{\vec q}}}$ is nonzero for any finite values of $E_{\vec p},E_{\vec q}$, so we learn that 
\begin{equation}\label{complexfieldcommutator}
    0=[b_{\vec p}, b_{\vec q}^\dagger] e^{i(\vec p \cdot \vec x -\vec q \cdot \vec y)} + [b_{\vec p},c_{\vec q}] e^{i(\vec p \cdot \vec x + \vec q \cdot \vec y)}
    + [c_{\vec p}^\dagger, b_{\vec q}^\dagger] e^{-i(\vec p \cdot \vec x + \vec q \cdot \vec y)} + [c_{\vec p}^\dagger,c_{\vec q}] e^{-i(\vec p\cdot \vec x - \vec q \cdot \vec y)}.
\end{equation}
This is a useful combination, since we can for instance add it to the commutator in Eqn. \ref{psipicommutator} to find that
\begin{equation}
    [\psi(\vec x),\pi(\vec y)]=i \int \frac{d^3p\, d^3q}{(2\pi)^6} 
    \left([b_{\vec p}, b_{\vec q}^\dagger] e^{i(\vec p \cdot \vec x -\vec q \cdot \vec y)} +[c_{\vec p}^\dagger, b_{\vec q}^\dagger] e^{-i(\vec p \cdot \vec x + \vec q \cdot \vec y)}\right),
\end{equation}
and since we know from the field relations that the left side is a delta function $i\delta^3(\vec x- \vec y)$, we can pass back to the integral form of the delta function to find
\begin{equation*}
    \int \frac{d^3p}{(2\pi)^3} e^{i \vec p\cdot (\vec x - \vec y)}=\int \frac{d^3p\, d^3q}{(2\pi)^6} 
    \left([b_{\vec p}, b_{\vec q}^\dagger] e^{i(\vec p \cdot \vec x -\vec q \cdot \vec y)} +[c_{\vec p}^\dagger, b_{\vec q}^\dagger] e^{-i(\vec p \cdot \vec x + \vec q \cdot \vec y)}\right),
\end{equation*}
or with a little forethought,
\begin{equation*}
    \int \frac{d^3p\, d^3 q}{(2\pi)^3} \delta^3(\vec p-\vec q) 
    e^{i (\vec p\cdot \vec x - \vec q \cdot\vec y)}
    =\int \frac{d^3p\, d^3q}{(2\pi)^6} 
    \left([b_{\vec p}, b_{\vec q}^\dagger] e^{i(\vec p \cdot \vec x -\vec q \cdot \vec y)} +[c_{\vec p}^\dagger, b_{\vec q}^\dagger] e^{-i(\vec p \cdot \vec x + \vec q \cdot \vec y)}\right).
\end{equation*}
Matching terms on left and right (which we can do since Fourier modes are orthogonal), we find that
\begin{equation*}
    [b_{\vec p},b_{\vec q}^\dagger]=(2\pi)^3 \delta^3(\vec p -\vec q), \quad [c_{\vec p}^\dagger,b_{\vec q}^\dagger]=0.
\end{equation*}
A basically identical calculation (subtracting Eqn. \ref{complexfieldcommutator} rather than adding it) yields the equivalent result for $[c_{\vec p},c_{\vec q}^\dagger]$ and $[b_{\vec p},c_{\vec q}]$. We find that
\begin{equation*}
    [b_{\vec p},b_{\vec q}^\dagger]=[c_{\vec p},c_{\vec q}^\dagger]=(2\pi)^3 \delta^3(\vec p -\vec q)
\end{equation*}
and all other commutators vanish. \qed

Next we'll show some properties of the particle number operator $Q$. First, normal ordering. Explicitly, we can write $Q$ in terms of creation and annihilation operators as
\begin{align*}
    Q&= i\int d^3x [\pi(\vec x)\psi(\vec x)-\psi^\dagger (\vec x) \pi^\dagger (\vec x)]\\
    &=i\int d^3x \frac{d^3p \, d^3q}{(2\pi)^6} \frac{i}{2} 
    \left[(b_{\vec p}^\dagger e^{-i\vec p \cdot x}-c_{\vec p}e^{i\vec p\cdot \vec x})(b_{\vec q} e^{i\vec q \cdot \vec x}+c_{\vec q}^\dagger e^{-i\vec q \cdot \vec x})\right.\\
    &{}\qquad\left.
    -(b_{\vec p}^\dagger e^{-i\vec p \cdot x}+c_{\vec p}e^{i\vec p\cdot \vec x})(-b_{\vec q} e^{i\vec q \cdot \vec x}+c_{\vec q}^\dagger e^{-i\vec q \cdot \vec x})\right]\\
    &= -\int d^3x \frac{d^3p \, d^3q}{(2\pi)^6} 
    [b_{\vec p}^\dagger b_{\vec q} e^{-i(\vec p - \vec q) \cdot \vec x} - c_{\vec p} c_{\vec q}^\dagger e^{i(\vec p -\vec q) \cdot \vec x)}]\\
    &= -\int \frac{d^3p \, d^3q}{(2\pi)^3}\delta^3(\vec p -\vec q) 
    [b_{\vec p}^\dagger b_{\vec q} - c_{\vec p} c_{\vec q}^\dagger]\\
    &= \int \frac{d^3p}{(2\pi)^3} [c_{\vec p} c_{\vec p}^\dagger-b_{\vec p}^\dagger b_{\vec p}].
\end{align*}
Applying normal ordering simply switches the $c$ and $c^\dagger$ so that
\begin{equation*}
    :Q:=\int \frac{d^3 p}{(2\pi)^3} (c_{\vec p}^\dagger c_{\vec p}- b_{\vec p}^\dagger b_{\vec p})=N_c-N_b,
\end{equation*}
as desired. \qed