Today, we'll take our first look at interacting theories in detail! Let's first complete our description of the interaction picture. Operators in the interaction picture evolve in time as
$$O_I(t) \equiv e^{iH_0t}O_S e^{-iH_0t},$$ with $O_S$ the Schr\"odinger picture operator. In the context of quantum field theory,
$$\phi_I(x)=e^{iH_0t}\phi(\vec x)e^{-iHt}$$
so that $\phi_I$ obeys the Klein-Gordon equation
$$(\p^2+m^2)\phi_I=0,$$ with solution
$$\phi_I(x)=\int \frac{d^3p}{(2\pi)^3 \sqrt{2E_p}} (a_{\vec p} e^{-ip \cdot x}+ a_{\vec p}^\dagger e^{ip \cdot x}.$$
Here, note that we're taking the four-vector inner product as in the Heisenberg picture, with $p^0=E_p$ and $x^0=t$. We also see that
$\phi_I(t=0,\vec x)= \phi_S(\vec x),$ so the fields at $t=0$ agree with the Schr\"odinger picture fields.

As before,
$$[a_{\vec p},a_{\vec p}^\dagger]=(2\pi)^3 \delta^3(\vec p - \vec p'),$$ with other brackets vanishing. Note that $\ket{0}$ satisfies $a_{\vec p}\ket{0}=0$ is the vacuum of the free theory, not the interacting theory. Interaction picture fields are related to the Heisenberg picture ones by
$$\phi_H(t,\vec x)=e^{iHt} e^{-iH_0t} \phi_I(x) e^{iH_0t} e^{-iHt},$$
where $e^{-iH_0t} \phi_I(x) e^{iH_0t}=\phi_S(\vec x).$ We can also regroup the operators here to write
$$\phi_H(t,\vec x)\equiv U(t,0)^\dagger \phi_I(t,\vec x) U(t,0)$$
where $U(t,t_0)\equiv e^{iH_0(t-t_0)}e^{-iH(t-t_0)}$ is a unitary time evolution operator such that
$$U(t_1,t_2)U(t_2,t_3)=U(t_1,t_3)$$ and $U(t,t)=1$. Equivalently,
$$\ket{\psi(t)}_I=U(t,t')\ket{\psi(t')}_I.$$ That is, $U$ evolves interaction picture states in time.

By differentiating with respect to time, we see that
\begin{eqnarray*}
i\frac{dU(t,0)}{dt}&=&i\left[iH_0 e^{iH_0 t}e^{-iHt}+e^{iH_0t}(-iH) e^{-iHt}\right]\\
&=&e^{iH_0t} (H-H_0)e^{-iHt}\\
&=&e^{iH_0t} (H_{int})_S e^{-iH_0t} e^{iH_0t} e^{-iHt}\\
&=&(H_{int})_I U(t,0).
\end{eqnarray*}

If $(H_{int})_I=H_I$ were just a function, we could solve this by $U=\exp[-i \int_{t_0}^t H_i(t')dt'],$ but because it is an operator we have ordering ambiguities,
$$[H_i(t'),H_I(t'')]\neq 0 \text{ for }t'\neq t''.$$
However, our differential equation for $U$ tells us that $U(t,t_0)$ satisfies
$$U(t,t_0)=1+(-i)\int_{t_0}^t dt' H_I(t')U(t',t_0)$$
(you can check this explicitly in one line). Therefore we can substitute this back into itself to get the infinite series
$$U(t,t_0)=1+(-i)\int_{t_0}^t dt' H_I(t')+(-i)^2 \int_{t_0}^t dt' \int_{t_0}^{t'} dt'' H_I(t')H_I(t'')+\ldots.$$
From the ranges of integration, it's clear that the $H_I$s are automatically time-ordered-- for instance, $H_I(t'')$ always takes place at $t''\geq t'$. In general we can exploit symmetry to write the second term as
$$\frac{(-i)^2}{2!}\int_{t_0}^t dt' \int_{t_0}^t dt'' T(H_I(t')H_I(t'')$$ (where we've picked up a $2$ as a symmetry factor), and in general we get a symmetry factor of $n!$ to a term with $n$ copies of $H_I(t')$.
\begin{defn}
We find that $U$ can be written compactly as
$$U(t,t_0)=T \exp \set{-i\int_{t_0}^t dt' H_I(t')},$$
which we call \term{Dyson's formula.} (Note that $U(t,t_0=T \exp \set{+i\int_{t_0}^t dt' L_I(t')}$, in terms of the Lagrangian.) This is a formal result, but we usually just expand to some finite order in terms of the coupling constants which live in the interacting Hamiltonian $H_I$.
\end{defn}
This is the last bit of machinery we need to start computing scattering amplitudes in quantum field theory!

\begin{defn}
The time evolution used in scattering theory is called the \term{$S$-matrix} (S for scattering). The $S$ matrix is defined to be
$$S=\lim_{t\to \infty,t_0\to -\infty}U(t,t_0).$$
\end{defn}
We will consider interactions where the final state $\ket{f}$ and the initial state $\ket{i}$ are well-separated from each other and are far away from the interaction. Therefore, the initial and final states $\ket{i},\ket{f}$ behave like free particles, i.e. they are eigenstates of $H_0$.\footnote{This is a heuristically useful description but a little slippery in the details. A  priori, there's no reason that eigenstates of the free Hamiltonian should be eigenstates of the interacting Hamiltonian. If you prefer, you can think of the scattering amplitude as the overlap (as measured by the inner product) between initial free particle states and final free particle states, with the possibility for some interaction in between. Even if we started with free particle eigenstates, our interaction is sure to evolve these states to some new ones, but we can look at the overlap between the time evolved versions of the free particle states $U(t,t_0)\ket{i}$ and the final free particle states we're interested in, $\bra{f}.$}

This should seem at least plausible: at late/early times, the particles are well-separated and don't feel the effect of each other. As they approach, they may interact before going their separate ways. The scattering amplitude is then
$$\lim_{t\to\infty, t_0\to -\infty}\bra{f}U(t,t_0)\ket{i}=\bra{f}S \ket{i}.$$

Note that there are some cases that need to be treated differently, like bound states. For instance, a proton and an electron at low energies could interact, $p+e^-\to$ the bound state $(H)$. Here, the assumption that the particles end up well-separated is violated. It turns out that such solutions appear as poles in the $S$-matrix, but this is a more advanced topic and we won't discuss it further here.

Let's return to scalar Yukawa theory. Now, we'll drop the $I$ subscripts and assume uniformly that we are in the interaction picture. We have the Hamiltonian
$$\cH = g \psi^* \psi \phi,$$
where $\psi$ and $\psi^*$ are (anti-)nucleons (e.g. a proton or neutron), and $\phi$ is a meson.
\begin{itemize}
    \item $\phi$ has $a$ and $a^\dagger$ terms which destroy and create mesons, respectively.
    \item $\psi$ has $b$ and $c^\dagger$ terms, where $b$ destroys a nucleon and $c^\dagger$ creates an anti-nucleon.
    \item $\psi^*$ has $b^\dagger$ and $c$ terms, where $b^\dagger$ creates a nucleon and $c$ destroys an anti-nucleon.
\end{itemize}
Looking at the possible terms in the Hamiltonian, we can already see interesting behavior-- we'll have terms where nucleon-anti-nucleon pairs are created and destroyed, e.g. $b^\dagger c^\dagger a$ which destroys a meson and produces a nucleon-anti-nucleon pair. This contributes to meson decay, $\phi\to \psi\bar \psi$.\footnote{When we talk about the fields, we use $*$, but when we denote antiparticles, we usually use the bar notation, e.g. an anti-$\psi$ is a $\bar \psi$.} What we recover is the leading order in $g$ term in $U(t,t_0)$.

At second order, we have more complicated terms like
$$g^2(b^\dagger c^\dagger a)(a^\dagger c b),$$
which describes nucleon-anti-nucleon scattering. We can draw a nice diagram for this process.

Returning to the case of meson decay, we have some $\phi$ meson going in with some defined momentum $\vec p$ as our initial state, and similarly we have $\psi,\bar \psi$ going out with some momenta $\vec q_1,\vec q_2.$ We can write these states as
$$\ket{i}=\sqrt{2E_p}a_{\vec p}^\dagger \ket{0}$$
and
$$\ket{f}=\sqrt{4E_{q_1}E_{q_2}}b_{\vec q_1}^\dagger c_{\vec q_2}^\dagger \ket{0}.$$
To zeroth order there is no interaction and the scattering amplitude is zero. To first order, we have
$$\bra{f}S\ket{i}=-ig \bra{f}\int d^4 x \psi^* (x) \psi(x) \phi(x) \ket{i} + O(g^2).$$
We'll compute this exactly next time and argue that the $O(g^2)$ corrections are relatively small, arriving at our first quantum field theory scattering amplitude.