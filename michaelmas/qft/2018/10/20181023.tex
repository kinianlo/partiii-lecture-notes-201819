Today, we'll introduce Wick's formula and contractions, calculate some more scattering amplitudes, and maybe see our first Feynman diagrams for calculating amplitudes in a more convenient way.

First, we complete the calculation of the order $g$ scattering amplitude from last time. We were interested in meson decay, where we prepared initial and final states
\begin{equation}
  \ket{i}=\sqrt{2E_{\vec p}}a_{\vec p}^\dagger \ket{0}  
\end{equation}
and
\begin{equation}
    \ket{f}=\sqrt{4E_{\vec q_1}E_{\vec q_2}}b_{\vec q_1}^\dagger c_{\vec q_2}^\dagger \ket{0},
\end{equation}
and we were interested in the scattering amplitude $\bra{f}S \ket{i}$. To leading order, we found that
\begin{equation}
    \bra{f}S\ket{i}=-ig \bra{f}\int d^4 x \psi^* (x) \psi(x) \phi(x) \ket{i} + O(g^2),
\end{equation}
and we'll now demonstrate how to compute this.

We know how to expand each of these fields in terms of their respective creation and annihilation operators, and we want to make sure that the initial state and final state are indeed proportional to each other so that this QFT amplitude will be reduced to a c-function of the four-momenta (i.e. it is just a number). 

Note that when we put fields in, the creation and annihilation operators have to precisely cancel out the particles in the initial and final states. For instance, in the field $\phi$ we have both $a_{\vec p}^\dagger$ and $a_{\vec p}$ terms, but our initial state $\ket{i}$ already has an $a^\dagger$ in it. So the $a^\dagger$ bit of $\phi$ acting on $\ket{i}$ will produce a two-meson state proportional to $a_{\vec p'}^\dagger a_{\vec p}^\dagger \ket{0}$ which the $\psi$s won't touch, and this means that the inner product of this two-meson state with $\bra{0}$ will be zero. Alternately, you can think of the $a^\dagger$ from $\phi$ as acting on the $\bra{0}$ on the left, since we can freely commute it through the $\psi$s. That is, $a_{\vec k}\ket{0}=0\implies \bra{0} a_{\vec k}^\dagger =0$, so in general any state with particles in it is going to be orthogonal to the vacuum. Our problem is therefore reduced to matching the operators in our fields with the operators in the initial and final states $\ket{i}$ and $\ket{f}$.

If we expand out the field $\phi$, our matrix element now takes the form
\begin{equation}
    \bra{f}S\ket{i}=-ig \bra{f} \int d^4x \psi^*(x) \psi(x)\int \frac{d^3k}{(2\pi)^3}\frac{\sqrt{2E_{\vec p}}}{\sqrt{2E_{\vec k}}}(a_{\vec k} a_{\vec p}^\dagger e^{-ik\cdot x}+a_{\vec k}^\dagger a_{\vec p}^\dagger e^{ik\cdot x})\ket{0}.
\end{equation}
But as we've just argued, this second term is zero, and we can switch the $a_{\vec k}, a_{\vec p}^\dagger$ at the cost of a delta function $(2\pi)^3 \delta^3(\vec k -\vec p)$, which allows us to do the integral over $d^3k$.%
    \footnote{That is, we write $a_{\vec k} a_{\vec p}^\dagger \ket{0}=(a_{\vec p}^\dagger a_{\vec k} +(2\pi)^3 \delta^3(\vec k-\vec p))\ket{0}=(2\pi)^3 \delta^3(\vec k-\vec p)\ket{0}$ since $a_{\vec k}$ kills the vacuum.}
    
Now expanding the fields $\psi^*, \psi$, we get
\begin{align*}
    \bra{f}S\ket{i} &= -ig\bra{f} \int d^4x \psi^* \psi(x)e^{-ip\cdot x}\ket{0}\\
    &= -ig \bra{0} \int \frac{d^4x}{(2\pi)^6} 
        \frac{d^3k_1 d^3k_2}{\sqrt{4E_{\vec k_1}E_{\vec k_2}}} \sqrt{4E_{\vec q_1}E_{\vec q_2}} c_{\vec q_2}b_{\vec q_1} 
        (b_{\vec k_1}^\dagger e^{ik_1\cdot x}+c_{\vec k_1} e^{-ik_1 \cdot x})\\
    &{} \qquad \times(b_{\vec k_1} e^{-ik_2\cdot x}+c_{\vec k_2}^\dagger e^{ik_2 \cdot x})e^{-ip\cdot x} \ket{0}.
\end{align*}
From the fields $\psi^*$ and $\psi$, only the $b^\dagger$ and $c^\dagger$ terms give a nonzero contribution, and taking the appropriate commutators gives us delta functions over the momenta (i.e. setting $\vec q_1 = \vec k_1,\vec q_2=\vec k_2$). Using these delta functions to compute the $k_1,k_2$ integrals we find that
\begin{align*}
\bra{f}S\ket{i}&=-ig\bra{0}\int d^4x e^{i(q_1+q_2-p)\cdot x}\ket{0}\\
&=-ig[(2\pi)^4\delta^4(q_1+q_2-p)],
\end{align*}
where this delta function simply imposes overall 4-momentum conservation. Note that this is a ``matrix element,'' and not a probability yet. To actually turn this into a measurable probability (e.g. for an experiment), we must take the mod squared and integrate over the possible outgoing momenta $q_1,q_2$-- we'll discuss this more later.%
    \footnote{See Lecture 13 and 14 if you're impatient, or a would-be experimentalist. Or both.}

\subsection*{Wick's theorem} We'll now discuss \term{Wick's theorem} for a real scalar field. When we are working out $S$-matrix elements for more than one interaction, we will often need to compute quantities like
$$\bra{f}T\set{H_I(x_1)\ldots H_I(x_n)}\ket{i},$$
the amplitude of some time-ordered product-- remember that Dyson says we ought to be evolving our states in time with time-ordered products. Our lives would be easier if we could work in terms of normal-ordered products instead, where the $a$s are on the RHS and the $a^\dagger$s are on the LHS. This would let us easily see which terms contribute to the final amplitude (e.g. which $a$s precisely cancel particles created by $a^\dagger$s in the initial state $\ket{i}$, and vice versa for the outgoing state $\ket{f}$). In fact we can do this! Wick's theorem relates time-ordered products to normal-ordered products in a reasonably nice way.

Let's compute a simple example first.
For a real scalar field $\phi$, what is the time-ordered product $T\{\phi(x)\phi(y)\}$?
    
To do this computation, we write our scalar field as
$$\phi(x)\equiv \phi^+(x)+\phi^-(x)$$
where
$$\phi^+ (x)\equiv \int \frac{d^3p}{(2\pi)^3}\frac{1}{\sqrt{2E_{\vec p}}} a_{\vec p} e^{-ip\cdot x}$$
is the annihilation part of the field $\phi$ and
$$\phi^-(x)\equiv \int \frac{d^3p}{(2\pi)^3}\frac{1}{\sqrt{2E_{\vec p}}} a_{\vec p}^\dagger e^{+ip\cdot x}$$
is the creation part of $\phi$.%
    \footnote{The signs here are an unfortunate convention having to do with these being ``positive frequency'' and ``negative frequency'' operators.}
%
Now if we first consider the case $x^0>y^0$, then $T\{\phi(x)\phi(y)\}$ takes the form
\begin{align*}
    T\set{\phi(x)\phi(y)}&=\phi(x)\phi(y)\\
    &=(\phi^+(x)+\phi^-(x))(\phi^+(y)+\phi^-(y))\\
    &=\phi^+(x)\phi^+(y)+\phi^-(x)\phi^-(y)+\phi^-(y)\phi^+(x)+\phi^-(x)\phi^+(y)+[\phi^+(x),\phi^-(y)]\\
    &=:\phi(x)\phi(y):+D(x-y)
\end{align*}
where we've collected terms with all the $\phi^+$ terms to the right of the $\phi^-$ terms into the normal-ordered product $:\phi(x)\phi(y):$. If we had $y^0>x^0$ instead, we would get
$$T\set{\phi(x)\phi(y)}=:\phi(x)\phi(y):+D(y-x).$$
Putting these together, we see that
\begin{equation}
    T\set{\phi(x)\phi(y)}=:\phi(x)\phi(y):+\Delta_F(x-y), 
\end{equation}
where $\Delta_F$ is simply the Feynman propagator. It's important to note that while the time-ordered and normal-ordered products are both operators, their difference is $\Delta_F$, a c-function.

\begin{defn}
We define a \term{contraction} of a pair of fields in a string (denoted by a square bracket between the two fields, or a curly overbrace here because of the limitations of my \LaTeX{} formatting) to mean replacing the two contracted fields by their Feynman propagator. That is, $\overbrace{\phi(x)\phi(y)} \equiv \Delta_F(x-y).$

For instance, we saw that
$$T\set{\phi(x)\phi(y)}=:\phi(x)\phi(y):+\overbrace{\phi(x)\phi(y)}.$$
\end{defn}
\begin{thm}[Wick's theorem]
Time-ordered products of fields are related to normal-ordered products in the following way: %Wick's theorem states that
\begin{equation}
    T\set{\phi(x_1)\ldots \phi(x_n)}=:\phi(x_1)\ldots \phi(x_n):+:\text{\emph{all possible contractions}}:
\end{equation}
\end{thm}

Note that ``all possible contractions'' here includes combinations of fields that are not fully contracted. For instance, the product $T\{\phi(x_1)\phi(x_2)\phi(x_3)\}$ will include terms like $\Delta_F(x_1-x_2)\phi(x_3)$. Similarly $T\{\phi(x_1)\phi(x_2)\phi(x_3)\phi(x_4)\}$ includes $\Delta_F(x_1-x_2):\phi(x_3)\phi(x_4):$ as well as the totally contracted $\Delta_F(x_1-x_2)\Delta_F(x_3-x_4)$.

\begin{exm}
Since all normal ordered terms kill the vacuum state, Wick's theorem allows us to immediately compute amplitudes like
$$\bra{0}T\set{\phi_1\ldots \phi_4}\ket{0}=\Delta_F (x_1-x_2)\Delta_F(x_3-x_4)+\Delta_F(x_1-x_3)\Delta_F(x_2-x_4)+\Delta_F(x_1-x_4)\Delta_F(x_2-x_3).$$
\end{exm}

The proof of Wick's theorem is by induction. Suppose it holds for $T\set{\phi_2\ldots \phi_n}$. Then (see textbooks for detail)
$$T\set{\phi_1 \phi_2\ldots \phi_n}=(\phi_1^+ +\phi_1^-)(:\phi_2\ldots \phi_n+:\text{all contractions of }\phi_2\ldots \phi_n:).$$
The $\phi_1^-$ is okay where it is, while the $\phi_1^+$ must be commuted to the RHS of the $\phi_2\ldots \phi_n$ terms. Each commutator past the $x_k$ term in $\phi_2\ldots \phi_n$ gives us a $D(x_1-x_k)$, which is equivalent to a contraction between $\phi_1$ and $\phi_k$.

Wick's theorem has some immediate consequences. For instance,
$$\bra{0}T \set{\phi_1\ldots \phi_n}\ket{0}=0$$ if $n$ is odd (since one $\phi$ is always left out of the contractions) and it is
$$\sum_{i_1,\ldots,i_n}\Delta_F (x_{i_1}-x_{i_2})\Delta_F(x_{i_3}-x_{i_4})\ldots \Delta_F(x_{i_{n-1}}-x_{i_n})$$
if $n$ is even, where the sum is taken over symmetric permutations of $i_1,\ldots,i_n$.

Note that Wick's theorem also has a generalization to complex fields $\psi\in \CC$, e.g.
$$T(\psi(x)\psi^*(y))=:\psi(x)\psi^*(y):+\Delta_F(x-y)$$
where the contraction of a $\psi$ and $\psi^*$ is a propagator, $\overbrace{\psi(x)\psi^*(y)}\equiv \Delta_F(x-y)$, and the contractions of two $\psi$s or two $\psi^*$s is zero, $\overbrace{\psi(x)\psi(y)}=\overbrace{\psi(x)^*\psi^*(y)}=0$.