Last time, we discussed the tensor product of two representations labeled by their highest weights $\Lambda,\Lambda'$. What we said is that we can decompose the tensor product space into a direct sum, even if the weights $\lambda$ live in a higher-dimensional space (cf. $L(SU(2))$, which had a 1-dimensional weight lattice):
$$R_\Lambda \otimes R_{\Lambda'} =\bigoplus \mathcal{N}_{\Lambda,\Lambda'}^{\Lambda''}R_{\Lambda''}.$$

For instance, we worked out with our algorithm that for $\fg=A_2$,
$$S_{(1,0)}=\set{\omega_1,-\omega_1+\omega_2,-\omega_2}.$$
The 9 elements of the direct sum come from taking the sums of pairs of elements, e.g. $\omega_1+\omega_1, \omega_1+(-\omega_1+\omega_2),$ etc. We would like to work out the representations in the direct sum, and we find that
$$\underbrace{R_{(1,0)}\otimes R_{(1,0)}}_{3\times 3}=\underbrace{R_{(2,0)}\oplus R_{(0,1)}}_{6+3},$$
so the dimension of the tensor product representation is equivalent to the dimension of the direct sum on the RHS as we expect.
We can play the same highest-weight-and-take-the-remainder game as we did for the angular momentum representations. That is, since adding the highest weight elements $\omega_1+\omega_1$ gives us $2\omega_1$ as the new highest weight of the tensor product representation with multiplicity $1$, we must have $R_{(2,0)}$ in the direct sum. After we remove all the elements of $S_{(2,0)}$ from $S_{(1,0)}\otimes S_{(1,0)}$, what remains is a $3$-dimensional representation which is exactly $R_{(0,1)}$-- see the footnote for details.\footnote{At the end of the last lecture, I wrote that the weight set of the tensor product was $$S_{(1,0)\otimes (1,0)}=\set{2\omega_1,\omega_2,\omega_1-\omega_2,\omega_2,2\omega_2-2\omega_1,-\omega_1,\omega_1-\omega_2,-\omega_1,-2\omega_2}$$
or in coordinate notation,
$$S_{(1,0)\otimes (1,0)}=\set{(2,0),(0,1),(1,-1),(0,1),(-2,2),(-1,0),(1,-1),(-1,0),(0,-2)}.$$
Running our algorithm for $R_{(2,0)}$ gives the weight set $$S_{(2,0)}=\set{(2,0),(0,1),(-2,2),(1,-1),(-1,0),(0,-2)}.$$
Comparing these, we see that what remains is
$$\set{(0,1),(1,-1),(-1,0)}=S_{(0,1)}.$$}

\subsection*{Symmetries in QM} Consider a quantum mechanical system with energy levels $E_n,n\geq 0$. The states of this system live in a Hilbert space $\cH$  which we can write as the direct sum of some eigenspaces $\cH_n$:
$$\cH=\bigoplus_{n\geq 0} \cH_n$$
such that for the Hamiltonian $\hat H$,
$$\hat H\ket{\psi}=E_n \ket{\psi}\forall \ket{\psi}\in \cH_n.$$

Now by a symmetry transformation we mean a transformation taking states $\ket{\psi}\to \ket{\psi'}=\hat U \ket{\psi}$.
A symmetry transformation is generated by a unitary operator $\hat U:\cH\to \cH$ such that $\hat U^\dagger \hat U = \hat I$, and such that
$$\hat U \hat H \hat U^\dagger = \hat H.$$

The benefit of such a definition is that the inner product is preserved,
\begin{align*}
    \braket{\psi_1'}{\psi_2'}&=\braket{\hat U \psi_1}{\hat U \psi_2}\\
    &=\braket{\hat U^\dagger \hat U \psi_1}{\psi_2}\\
    &=\braket{\psi_1}{\psi_2}
\end{align*}
by unitarity. It also preserves the eigenspaces $\cH_n$ (i.e. the energy levels):
$\ket{\psi} \in \cH_n \iff \ket{\psi'} \in \cH_n.$

Conserved quantities in our theory correspond to observables (Hermitian operators) $\hat \p = \hat \p^\dagger$, such that
$$[\hat \p, \hat H]=0.$$
This is in analogy to the Poisson brackets vanishing in classical mechanics. Then we say that the exponential
$$\hat U=\exp(is \hat \p)$$
generates a symmetry $\forall s\in \RR$.

It will be useful for us to consider a maximal set of conserved quantities
$$\set{\hat \p^a:[\hat \p^a,\hat H]=0, a=1,\ldots,d}.$$
But note that we have a natural bracket on operators (the regular commutator). This bracket obeys the Jacobi identity and the set is closed under the bracket,
\footnote{Since the bracket obeys the Jacobi identity (it is an ordinary commutator), it follows that if $[\hat p^1,\hat H]=0$ and $[\hat p^2, \hat H]=0$ then
\begin{equation*}
    [\hat p^1,[\hat p^2,H]]+[\hat p^2,[H,\hat p^1]]+[\hat H,[\hat p^1, \hat p^2]]=[\hat H,[\hat p^1, \hat p^2]]=0
\end{equation*}
by the Jacobi identity. Therefore if our set of conserved quantities is $P$, then $\forall \hat p^1, \hat p^2 \in P, [[\hat p^1,\hat p^2],H]=0 \implies [\hat p^1, \hat p^2]\in P$, i.e. the set is closed under the bracket. \qed}
so this leads us to naturally think of the set as a Lie algebra:
$$\fg_\RR =\text{Span}_\RR \set{i \hat \p^a, a=1,\ldots,d}$$
equipped with a bracket which we define to be the commutator of operators. As it is a real Lie algebra,
$$\dim[\fg_\RR]=d.$$

Our symmetry transformations generated by
$$\hat U = \exp(\hat X), \hat X \in \fg_\RR,$$
therefore form a \term{compact Lie group} $G$ such that $$\fg_\RR= L(G).$$
As $[\hat X,\hat H]=0 \quad \forall X\in \fg_\RR$, the energy eigenspaces $\cH_n$ are invariant, so each $\cH_n$ carries a representation $D_n$ of the Lie group $G$ and the associated representation $d_n$ of the corresponding Lie algebra $\fg_\RR$. These representations are labeled by the energy levels $n$. Thus
$$D_n(\hat U)=\exp(d_n(\hat X))\in \text{Mat}_{N_n}\set{\CC}$$ where $N_n=\dim\set{\cH_n}$ and $\hat U= \exp(\hat X)\in G, \hat X \in \fg_\RR$.

Moreover, the representation of the Lie group must be unitary, so the representation of the Lie algebra is anti-hermitian:
$$D_n^{-1}(\hat U)=D_n(\hat U)^\dagger \forall \hat U \in G \iff d_n(\hat X)=-d_n (\hat X)^\dagger \forall \hat X \in \fg_\RR.$$
We also state the following fact-- every complex simple finite-dimensional Lie algebra $\fg$ has a real form of compact type. That is, with $\fg_\RR=L(G), G$ a compact Lie group, we get the results in the table.
\begin{table}[]
    \centering
    \begin{tabular}{c|c}
        $\fg$ & $G$ \\\hline
         $A_n$ & $SU(n+1)$  \\
         $B_n$ & $SO(2n+1)$ \\
         $C_n$ & $Sp(2n)$ \\
         $D_n$ & $SO(2n)$
    \end{tabular}
    \caption{The four families of infinite dimensional semi-simple Lie algebras and their corresponding Lie groups.}
    \label{tab:my_label}
\end{table}

What we conclude is that each of the irreps $R_\Lambda$ of a simple Lie algebra $\fg$ provides a unitary irrep of some real Lie algebra $\fg_\RR$, which we can take to be the symmetries of some physical system. This is because unitarity is closely related to the invariance of the inner product on the representation spaces, as we showed explicitly. We have
$$R_\Lambda(X)^\dagger = - R_\Lambda(X) \forall X\in \fg_\RR.$$

\subsection*{Gauge theories} Let us recall that we describe a \term{gauge symmetry} as a (nonphysical) redundancy in the description of a system. For instance, we can freely set the phase of a wavefunction in QM-- it is only differences in phase which might have physical significance.

In classical EM, we had the scalar and vector potentials
$$\Phi(\vec x,t),\vec A(\vec x,t).$$
Since the physical fields are 
\begin{align*}
    \vec E&=-\grad \Phi+\P{\vec A}{t}\\
    \vec B&= \curl \vec A,
\end{align*} we see that the fields are preserved under the gauge transformation
\begin{align*}
    \phi&\to \Phi+\P{\alpha}{t}\\
    \vec A&\to \vec A +\grad \alpha
\end{align*}
with $\alpha=\alpha(\vec x,t)$ an arbitrary function. Promoting our theory to a relativistic theory, we say that the transformation
$$\a_\mu \to a_\mu + \p_\mu \alpha$$
is a gauge transformation, where $a_\mu=(\Phi,\vec A).$ Moreover we can assemble the fields into a gauge-invariant field strength tensor
$$f_{\mu\nu}=\p_\mu a_\nu -\p_\nu a_\mu.$$
The corresponding Lagrangian is
$$\cL_{EM}=-\frac{1}{4g^2}f_{\mu\nu}f^{\mu\nu},$$
where we are using the mostly-minus convention for the Minkowski metric, $\eta=\set{+1,-1,-1,-1}.$
As we just saw in \emph{Quantum Field Theory}, our gauge freedom precisely gets rid of one of the extra degrees of freedom in $a_\mu$ so that when we quantize the field, we get the expected two degrees of freedom for massless spin $1$ particles.\footnote{Actually, $a_\mu$ na\"ively has four degrees of freedom-- gauge freedom only gets rid of one. It turns out that $a_0$ is non-dynamical for Maxwell's equations in vacuum-- it is just a number and its time derivatives do not appear in the fields. So $a_0$ doesn't really count as a proper degree of freedom, and thus $4-2=2$ polarization states, as expected.}

We also make the slight redefinitions:
\begin{align*}
    A_\mu &= -i a_\mu \in i\RR\\
    F_{\mu\nu}&= -i f_{\mu\nu}.
\end{align*}
We do this in order to make the generalization to other Lie algebras clearer (here, we consider $SU(2)$). For instance, consider a complex scalar field,
$$\cL_\phi = \p_\mu \phi^* \p^\mu \phi - W(\phi^* \phi).$$
This Lagrangian is invariant under a $U(1)$ global symmetry $\phi \to g \phi, \phi^* \to g^{-1} \phi^*$ such that $g = \exp(i\delta) \in U(1), \delta \in [0,2\pi)$.

We might be interested infinitesimal transformations
$\fg = \exp(\epsilon X)\approx (1+\epsilon X),$
where $\epsilon \ll 1,$ and thus $X\in \cL(U(1))= i\RR$. That is, the elements of the Lie algebra are generators of the infinitesimal transformations. For our complex scalar field, our transformation acts on the field by
$$\phi \to \phi+\delta_X \phi$$
where
$$\delta_X \phi = \epsilon X \phi.$$
Similarly we have
$$\phi^*\to \phi^*+\delta_X \phi^*$$
where now
$$\delta_X \phi^* = -\epsilon X \phi^*.$$
The story which we'll finish next time is that if the overall variation of the Lagrangian with respect to the symmetry transformation generated by $X$ vanishes,
$$\delta_X \cL_\phi =0,$$
then by Noether's theorem, there is some conserved charge associated to the system.