Today we'll complete our initial discussion of Killing forms and begin the Cartan classification of finite-dimensional simple complex Lie algebras.

Last time, we showed that if $\fg$ is not semi-simple, then some of the structure constants must vanish. Let's see what how this implies that $\kappa$ is degenerate (i.e. there exists some $v\in V$ such that its Killing form vanishes, $i(v,w)=0$ for all $w\in V$).
\begin{thm}[Cartan's theorem]
For a Lie algebra $\fg$, if its Killing form $\kappa$ is nondegenerate, then $\fg$ is semi-simple.
\end{thm}
\begin{proof}
We started by separating the basis vectors $T^a$ into a set $\set{T^i,i=1,\ldots,d}$ spanning the ideal $\mathfrak{j}$ and the rest of the basis vectors $\set{T^\alpha, \alpha=d+1,\ldots, D}$. Since $\mathfrak{j}$ is abelian, we found that
\begin{equation}\label{ijazero}
[T^i, T^j]=0\quad \forall i,j=1,\ldots,d \implies f^{ij}_a=0,
\end{equation}
and since $\mathfrak{j}$ is an ideal
\begin{equation}\label{ajbzero}
[T^\alpha, T^j]=f^{\alpha j}_k T^k\in \mathfrak{j} \implies f^{\alpha j}_\beta=0.
\end{equation}
Now consider a general element of the Lie algebra, $$X=X_a T^a \in \mathfrak{g}$$ and a general element of the ideal,
$$Y=Y_i T^i \in \mathfrak{j}.$$
Then
$$\kappa(X,Y)=\kappa^{ai}X_a Y_i,\text{ with }
\kappa^{ai}\equiv f^{ad}_c f^{ic}_d.$$
Let's take this carefully.
\begin{align*}
\kappa^{ai}&=f^{ae}_c f^{ic}_e\text{ by definition}\\
&=f^{ae}_\alpha f^{i\alpha}_e\text{ by }\ref{ijazero}\\
&=f^{aj}_\alpha f^{i\alpha}_j\text{ by }\ref{ajbzero}.
\end{align*}
To go from the first line to the second, we have used the fact that if $c=1,\ldots,d$, then $f^{ic}_e$ vanishes, so $f^{ic}_e=f^{i\alpha}_e$. To go from the second line to the third, we have then used the fact that if $e=d+1,\ldots,D$ then $f^{ae}_{\alpha}$ vanishes, so $f^{ae}_\alpha=f^{aj}_\alpha.$
Now separate the sum over $a=1,\ldots,D$ into $k=1,\ldots,d$ and $\beta=d+1,\ldots,D$.
Thus
$$\kappa^{ai}=\underbrace{f^{\beta j}_\alpha}_{\text{zero by } f^{\alpha j}_\beta=0} f^{i\alpha}_j + \underbrace{f^{kj}_\alpha}_{\text{zero by}f^{kj}_a=0} f^{i\alpha}_k=0.$$
Therefore $$\kappa[X,Y]=0 \quad\forall Y\in \mathfrak{j},\forall X \in \fg\implies \kappa\text{ is degenerate.}$$ Taking the contrapositive, we conclude that $\kappa$ is nondegenerate $\implies \fg$ is semi-simple.
\end{proof}

In Hugh Osborn's notes, he proves the other direction, so this turns out to be an if and only if. That is, $\kappa$ is nondegenerate $\iff \fg$ is semi-simple.

\subsection*{Cartan classification} Cartan proved in 1894 that one can fully classify all finite dimensional, simple, complex Lie algebras. Happily, these are often the ones which are of most use to us in physics. Simple Lie groups come with non-degenerate inner products (a fortiori, since simple implies semi-simple), which is a nice property. Moreover we will often look at complex Lie algebras since the field $\CC$ is \term{algebraically closed}-- polynomials with complex coefficients have in general complex solutions, whereas the same is not true for polynomials with real coefficients (which can have complex solutions).

Recall that when we did the representation theory of $L(SU(2))_\CC,$ we defined a Cartan-Weyl basis,
$$\set{H,E_\pm}$$ where $H$ is diagonal and $E_\pm$ moves us between eigenvectors. The brackets turned out to be
$$[H,H]=0,\quad [H,E_\pm]=\pm 2 E_{\pm}$$
What this tells us is that the ad map $\text{ad}_H$ (defined by $\text{ad}_H(X)=[H,X]$, in case you forgot) is diagonal, and it has eigenvalues $0,\pm 2$. We would now like to generalize this principle.
\begin{defn}
We say that $X\in \fg$ is \term{ad-diagonalizable} (AD) if
$$\text{ad}_X:\fg \to \fg$$
is diagonalizable.
\end{defn}
For a matrix, this meant that we could write the map as a diagonal matrix by a similarity transformation. More generally, a map is diagonal if we can construct a complete basis for the space it acts on out of eigenvectors of that map.

\begin{defn}
A \term{Cartan subalgebra} $\mathfrak{h}$ of $\fg$ is a maximal abelian subalgebra containing only AD elements.
\end{defn}
Unpacking this definition, a Cartan subalgebra therefore has the following properties.
\begin{itemize}
    \item[i)] $H\in \mathfrak{h}\implies H$ is AD.
    \item[ii)] $H,H'\in h \implies [H,H']=0.$
    \item[iii)] If $X\in \fg$ and $[X,H]=0 \forall H \in \mathfrak{h}$ then $X\in \mathfrak{h}$ (this is what we mean by maximal).
\end{itemize}
\begin{defn}
The dimension of the Cartan subalgebra,
$$r\equiv \dim[\mathfrak{h}],$$
is known as the \term{rank}. It turns out that all possible Cartan subalgebras $\mathfrak{h}\subset \fg$ have the same dimension, so it makes sense to say that $r$ is the rank of $\fg$.
\end{defn}

\begin{exm}
In $L_\CC(SU(2)),$ we have $H=\sigma_3$ and $E_{\pm}=\frac{1}{2}(\sigma_1\pm i\sigma_2)$. We explicitly wrote down the eigenvalues and eigenvectors of the ad map of $H$, so $H$ is ad-diagonalizable. We may choose
$\mathfrak{h}=\text{span}_\CC \set{H}.$
We could have chosen $\sigma_1$ or $\sigma_2$ as our element of the Cartan subalgebra (cooking up combinations of the other two Pauli matrices so that $\text{ad}_{\sigma_1}$ or $\text{ad}_{\sigma_2}$ is diagonal). However, \emph{we could not have chosen $E_+$ as the element of our Cartan subalgebra.} This is apparent when we write down $E_+$ as a matrix:
$$E_+=\begin{pmatrix}
0&1\\0&0
\end{pmatrix}$$
is clearly not diagonalizable and therefore not ad-diagonalizable.
\end{exm}
\begin{exm}
Consider $\fg=L_\CC (SU(n)),$ the set of traceless complex $n\times n$ matrices. A natural basis set is the pairs of diagonal elements
$$(H^i)_{\alpha,\beta}=\delta_{\alpha i} \delta_{\beta i}-\delta_{\alpha (i+1)}\delta_{\beta(i+1)}.$$
For instance, $H^1$ looks like
$$H^1=\begin{pmatrix}
1 & 0& &\\
0 & -1 & &\\
& & 0 &\\
& & & \ddots
\end{pmatrix}.$$
We now claim that the diagonal elements $H^i, i=1,\ldots, n-1$ are generators of the Cartan subalgebra, i.e.
$$\mathfrak{h}=\text{span}_\CC\set{H^i, i=1,\ldots n-1}.$$
It's clear that the diagonal elements $H^i$ commute and therefore have vanishing bracket,
$$[H^i,H^j]=0 \quad \forall i,j=1,\ldots,r.$$
But passing to the adjoint representation, this means that
$$(\text{ad}_{H^i}\circ \text{ad}_{H^j} - \text{ad}_{H^j}\circ \text{ad}_{H^i})=0,$$
so our basis elements $H^i$ naturally define $r$ linear maps
$$\text{ad}_{H^i}:\fg \to \fg$$
which are simultaneously diagonalizable (i.e. we can find a single set of eigenvectors which are compatible with all the linear maps).
Therefore $\fg$ is spanned by the simultaneous eigenvectors of $\text{ad}_{H^i}.$ What may we conclude from this? Well, the ad map $\text{ad}_{H^i}$ has some zero eigenvalues:
$$\text{ad}_{H^i}(H^j)=[H^i,H^j]=0\forall i,j =1,\ldots,r$$
so the ad map has $r$ zero eigenvalues.

The map also has non-zero eigenvalues which correspond to some set of eigenvectors
$$\set{E^\alpha, \alpha \in \Phi}$$
with $\Phi$ some set of eigenvalues. The ad map acts on these $E^\alpha$ by
$$\text{ad}_{H^i}(E^\alpha)=[H^i, E^\alpha]=\alpha^i E^\alpha$$
where the $\alpha^i\in \CC, i=1,\ldots, r$ are not all zero. (If they were all zero, $E^\alpha$ would be in $\mathfrak{h}$ by the maximality condition.) In $L(SU(2))_\CC$, these eigenvectors were just the elements $E_\pm.$
\end{exm}
\begin{defn}
These values $\alpha^i$ define a \term{root} $\alpha$ of $\fg.$ That is, a root $\alpha$ can be thought of as an abstract label on the eigenvectors $E^\alpha$ defining its eigenvalues under the ad map $\ad_{H^i}$. We'll see another way to think of roots shortly, as objects in their own right (namely, linear maps) which act on the elements $H^i \in \fh$. More on this next time.
\end{defn}

