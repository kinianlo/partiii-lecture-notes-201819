So far, we have discussed global symmetries.
\begin{itemize}
\item Spacetime symmetries:
\begin{itemize}
\item Rotation, $SO(3)$.
\item Lorentz transformations, $SO(3,1)$. (Rotations in $\RR^3$ plus boosts.)
\item The Poincar\'e group (not a simple Lie group, so does not fit Cartan classifications)
\item Supersymmetry? (i.e. a symmetry between fermions and bosons, described by ``super'' Lie algebra)
\end{itemize}
\item Internal symmetries:
\begin{itemize}
\item Electric charge
\item Flavor, $SU(3)$ in hadrons
\item Baryon number
\end{itemize}
\end{itemize}
But we also have gauge symmetry.
\begin{defn}
A \term{gauge symmetry} is a redundancy in our mathematical description of physics. For instance, the phase of the wavefunction in quantum mechanics has no physical meaning:
\begin{equation}
\psi\to e^{i\delta}\psi
\end{equation}
leaves all the physics unchanged ($\delta\in \RR$).\footnote{However, differences in phase can have significant effects-- see for instance the \href{https://en.wikipedia.org/wiki/Aharonov\%E2\%80\%93Bohm_effect}{Aharanov-Bohm effect}.}
\end{defn}

\begin{exm}
Another gauge symmetry familiar to us is the gauge transformation in electrodynamics,
$$\vec{A}(\vec{x})\to \vec{A}(\vec{x})+\grad \chi(\vec{x}).$$
By adding the gradient of some scalar function $\chi$ of $\vec{x}$, this leaves $\vec{B}=\curl \vec{A}$ unchanged (since $\curl{\grad F}=0$) and so the fields corresponding to the vector potential produce the same physics. Gauge invariance turns out to be key to our ability to quantize the spin-1 field corresponding to the photon.
\end{exm}
\begin{exm}
Another example (maybe less familiar in the exact details) is the Standard Model of particle physics.\footnote{We'll unpack the Standard Model more in next term's Standard Model class.} The Standard Model is a non-abelian gauge theory based on the Lie group
$$G_{SM}=SU(3)\times SU(2)\times U(1).$$
\end{exm}

We started to describe Lie groups last time. Let us repeat the definition here: a Lie group $G$ is a group which is also a (smooth) manifold. Informally, a manifold is a space which locally looks like $\RR^n$-- for every point on the manifold, there is a smooth map from an open set of $\RR^n$ to the manifold (that patch ``looks flat''), and these maps are compatible. For cute wordplay reasons, the collection of such maps is known as an atlas.

Sometimes it is useful to consider a manifold as embedded in an ambient space, e.g. $S^2$ embedded in $\RR^3$: $\vec{x}(x,y,z)\in \RR^3$ such that $x^2+y^2+z^2=r^2, r>0$.

\begin{defn}
We may define $\vec{x}=(x_1,x_2,\ldots,x_{n+m})\in \RR^{n+m}$, such that for a continuous, differentiable function $F^\alpha(\vec{x}): \RR^{n+m}\to \RR, \alpha=1,\ldots,m$, a space $M$ is defined by all such $\vec{x}$ satisfying $F^\alpha(\vec{x})=0,\a=,1\ldots,m$.
\end{defn}

\begin{thm}\label{manifoldcondition}
$M$ is a smooth manifold of dimension $n$ if the Jacobian matrix $J$ has rank $m$, with the Jacobian defined
$$J^\alpha_i =\P{F^\alpha}{x_i}.$$
In words, all this says is that $M$ is a manifold if $F^\alpha$ imposes a nice independent set of $m$ constraints on our $n+m$ variables, leaving us with a manifold of dimension $n$.
\end{thm}

\begin{exm}
For the sphere $S^2$, we have $m=1,n=2$ and we have the constraint $F^1(\vec{x})=x^2+y^2+z^2-r^2$ for some $r$. Then the Jacobian is simply $J=(\P{F^1}{x},\P{F^1}{y},\P{F^1}{z})=2(x,y,z)$, and this matrix indeed has rank $1$ unless $x=y=z=0$.
\end{exm}

Group operations (multiplication, inverses) define smooth maps on the manifold. The \term{dimension} of $G$, denoted $\dim(G)$, is the dimension of the group manifold $M(G)$. We may introduce coordinates $\{\theta^i\}, i=1,\ldots, D=\dim(G)$ in some local coordinate patch $P$ containing the identity $e\in G$. Then the group elements depend continuously on $\{\theta^i\}$, such that $g=g(\theta)\in G$ (the manifold structure is compatible with group elements). Set $g(0)=e$. 

Thus if we choose two points $\theta,\theta'$ on the manifold $M$, group multiplication,
$$g(\theta)g(\theta')=g(\phi)\in G,$$
corresponds to (induces) a smooth map $\phi: G\times G \to G$
which can be expressed in coordinates
$$\phi^i=\phi^i(\theta,\theta'), i=1,\ldots,D$$
such that $g(0)=e \implies$
$$\phi^i(\theta,0)=\theta^i, \phi^i(0,\theta')={\theta'}^i.$$
We ought to be a little careful that our group multiplication doesn't take us out of the coordinate patch we've defined our cooridnates on, but in practice this shouldn't cause us too many problems.

Similarly, group inversion defines a smooth map, $G\to G$. This map can be written as follows:
$$\forall g(\theta)\in G, \exists g^{-1}(\theta)=g(\tilde \theta)\in G$$
such that $$g(\theta)g(\tilde\theta)=g(\tilde\theta)g(\theta)=e.$$
In coordinates, the map
$$\tilde \theta^i = \tilde\theta^i(\theta), i =1,\ldots, D$$
is continuous and differentiable.

\begin{exm}
Take the Lie group $G=(\RR^D,+)$ (Euclidean $D$-dimensional space with addition as the group operation). Then the map defined by group multiplication is simply
$$\vec{x}''=\vec{x}+\vec{x}' \forall \vec{x},\vec{x}'\in \RR^D$$
and similarly the map defined by group inversion is
$$\vec{x}^{-1}=-\vec{x} \forall \vec{x}\in \RR^D.$$
This is a bit boring since the group multiplication law is commutative, so we'll next look at some important non-abelian groups-- namely, the matrix groups.
\end{exm}

\subsection*{Matrix groups} Let $\text{Mat}_n(F)$ denote the set of $n\times n$ matrices with entries in a field $F=\RR$ or $\CC$. These satisfy some of the group axioms-- matrix multiplication is closed and associative, and there is an obvious unit element, $e=I_n \in \text{Mat}_n(F)$ (with $I_n$ the $n\times n$ unit matrix). However, $\text{Mat}_n(F)$ is not a (multiplicative) group because not all matrices are invertible (e.g. with $\det M=0$). (Since it is not a group, it is also not a Lie group, though it does have a manifold structure, that of $\RR^{n^2}$.) Thus, we define the \term{general linear groups.}
\begin{defn}
The general linear group $GL(n,F)$ is the set of matrices defined by
\begin{equation}
GL(n,F)\equiv \set{M\in \text{Mat}_n(F): \det M \neq 0}.
\end{equation}
\end{defn}

\begin{defn}
We also define the \term{special linear groups} $SL(n,F)$ as follows:
\begin{equation}
SL(n,F)\equiv \set{M\in GL(n,F): \det M = 1.}
\end{equation}
Here, closure follows from the fact that determinants multiply nicely, $\forall M_1,M_2\in GL(n,F),\det(M_1M_2)=\det(M_1)\det(M_2)=\pm 1$ ($=1$ for $SL(n,F)$), and existence of inverses follows from the defining condition that $\det M\neq 0$.
\end{defn}

It's less obvious that $GL(n,F)$ and $SL(n,F)$ are also Lie groups. In fact, our theorem (Thm. \ref{manifoldcondition}) applies here: the condition that $\det M=\pm 1$ corresponds to a nice $F(\vec{x})=\det M -1, \vec{x}\in \RR^{n^2}$, which is sufficently nice as to define a manifold. The same is true for $SL(n,F)$, so these are indeed Lie groups. Note the dimensions of these sets are as follows.
\begin{center}
\begin{tabular}{c c}
$\dim(GL(n,\RR))=n^2$ & $\dim(GL(n,\CC))=2n^2$\\
$\dim(GL(n,\RR))=n^2-1$ & $\dim(SL(n,\CC))=2n^2-2$
\end{tabular}
\end{center}
Let me unpack this a little. In $\text{Mat}_n(F)$, we have our free choice of any numbers we like in $F$ for the $n^2$ elements of our matrix. It turns out that imposing $\det M\neq 0$ is not too strong a constraint-- it eliminates a set of zero measure from the space of possible $n\times n$ matrices, so we have our choice of $n^2$ real numbers in $GL(n,\RR)$ and $n^2$ complex numbers, so $2n^2$ real numbers, in $GL(n,\CC)$. Imposing the restriction that $\det M =1$ is now a stronger algebraic condition-- it reduces our choice of values by 1, since if we have picked $n^2-1$ values of the matrix, the last value is completely determined by $\det M=1$. Thus the dimension of $SL(n,\RR)$ is $n^2-1$. Since we get to pick $n^2-1$ complex numbers in $SL(n,\CC)$ (equivalently there are two real constraints, one on the real components and one on the imaginary ones), that amounts to $2(n^2-1)=2n^2-2$ real numbers. Hence, dimension $2n^2-2$.

\begin{defn}
A \term{subgroup} $H$ of a group $G$ is a subset ($H\subseteq G$) which is also a group. We write it as $H\leq G$. If $H$ is also a smooth submanifold of $G$, we call $H$ a \term{Lie subgroup} of $G$.
\end{defn}

This is where we ended for today. //

We'll next discuss some important subgroups of $GL(n,\RR)$. First, the \term{orthogonal groups.}
\begin{defn}
Orthogonal groups $O(n)$ are the matrix groups which preserve the Euclidean inner product,
\begin{equation}
O(n)=\set{M\in GL(n,\RR): M^T M = I_N}.
\end{equation}
Their elements correspond to orthogonal transformations, so that for $\vec{v}\in \RR^n$, an orthogonal matrix $M$ acts on $\vec{v}$ by matrix multiplication,
$$\vec{v}'=M\cdot \vec{v}$$
and so in particular
$$|\vec{v}'|^2={\vec{v}'}^T \cdot \vec{v}' = \vec{v}^T \cdot M^T M \cdot \vec{v}= \vec{v}^T \cdot \vec{v}=|\vec{v}|^2.$$
It also follows that $\forall M\in O(n), \det(M^TM)=\det(M)^2 = \det(I_n) = 1 \implies \det M =\pm 1$.
\end{defn}
The orthogonal group $O(n)$ has two connected components.
\begin{defn}
The \term{special orthogonal groups} $SO(n)$ are the subset of orthogonal groups which also preserve orientation (i.e. no reflections).
$$SO(n)\equiv \set{M\in O(n): \det M = 1}.$$
That is, elements of $SO(n)$ preserve the sign of the volume element in $\RR^n$,
$$\Omega= \epsilon^{i_1 i_2 \ldots i_n} v_1^{i_1}v_2^{i_2}\ldots v_n^{i_n}.$$
\end{defn}

\begin{ex}
Check the group axioms for $SO(n)$.\footnote{As usual, we need to check closure and inverses. The identity matrix $I$ satisfies $I^TI=I$ and $\det I=1$, and associativity follows from standard matrix multiplication. Inverses: if $M\in SO(n)$, then $M^{-1}$ is defined by $MM^{-1}=I$. But $\det(MM^{-1})=\det(M)\det(M^{-1})=(1)\det(M^{-1})=\det I = 1$, so $\det(M^{-1})=1$. Closure: $\forall M,N \in SO(n), \det(MN)=\det(M)\det(N)=(1)(1)=1$, so $MN\in SO(n)$. \qed} %we also need to check the transpose property, that these are also orthogonal
Show that $\dim(O(n))=\dim(SO(n))=\frac{1}{2} n(n-1)$.
\end{ex}