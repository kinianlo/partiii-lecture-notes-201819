Today, we'll consider the consequences of some specific representations and their structures.

\begin{defn}
Two representations $R_1$ and $R_2$ are isomorphic if $\exists$ a matrix $S$ such that
$$R_2(X)=S R_1(X)S^{-1} \forall X\in \fg.$$
Note this must be the same matrix $S$: that is, $R_2$ and $R_1$ are related by a change of basis. If so, we denote this as $$R_1\cong R_2.$$
\end{defn}
\begin{defn}
A representation $R$ with representation space $V$ has an \term{invariant subspace} $U\subset V$ if
$$R(X)u\in U \forall X\in \fg, u\in U.$$
(This is equivalent to our ideals in Lie algebras and normal subgroups in group theory.)

Any representation has two trivial invariant subspaces: they are the vector $U=\set{0}$ and $U=V$ the whole representation space.
\end{defn}
\begin{defn}
An \term{irreducible representation} (irrep) of a Lie algebra has no non-trivial invariant subspaces.
\end{defn}

With these definitions in hand, let's look at the representation theory of $L(SU(2))$. It's useful to us to write down a basis for the Lie algebra $L(SU(2))$:
$$\set{T^a=-\frac{1}{2}i\sigma_a,a=1,2,3}$$
with $\sigma_a$ the Pauli matrices. We calculated the structure constants:
$$[T^a,T^b]=f^{ab}_c T^c$$ with $f^{ab}_c=\epsilon_{abc}$ (the alternating tensor/symbol) and $a,b,c=1,2,3$. Let's do something kind of strange-- we'll write a new complex basis,
\begin{eqnarray*}
H&\equiv &\sigma_3 =\begin{pmatrix}
1&0\\0&-1
\end{pmatrix}\\
E_+ &\equiv& \frac{1}{2}(\sigma_1+i\sigma_2) =\begin{pmatrix}
0&1\\0&0
\end{pmatrix}\\
E_-&\equiv&\frac{1}{2}(\sigma_1 - \sigma_2)=\begin{pmatrix}
0&0\\1&0
\end{pmatrix}.
\end{eqnarray*}
This is really a basis for a somewhat bigger space, the complexified Lie algebra
$$L_\CC (SU(2))=\text{Span}_\CC\set{T^a, a=1,2,3}.$$

For now, we'll simply note that for $X\in L(SU(2))$, we can certainly rewrite $X$ as
$$X=X_H H + X_+ E^++ X_- E^-,$$
where $X_H\in i\RR$ and $X_+=(\bar X_-).$ This is called the Cartan-Weyl basis for $L(SU(2))$.\footnote{We've been writing $L(G)$ to distinguish the Lie algebra from the corresponding Lie group $G$, but other texts may use the convention of writing $su(2)$ using lowercase letters or the Fraktur script $\mathfrak{su}(2)$ for the Lie algebra. Just a convention to be aware of.}
In this basis, a general element takes the form
$$X=\begin{pmatrix}
X_H & X_+\\
X_-&-X_H
\end{pmatrix}.$$
This is certainly traceless, and $X\in L(SU(2))\iff$ $X$ is antihermitian, i.e. $X_H\in i\RR, X_+ = -(\bar X_-)$.

This basis has some nice properties. For instance, we see that
\begin{eqnarray*}
[H,E_\pm] &=& \pm 2 E_\pm\\
\text{ and } [E_+,E_-] &=& H.
\end{eqnarray*}
Hence the ad map takes a very simple form:
$ad_H(E_\pm)=\pm 2 E_\pm, ad_H(H)=0.$
We also have $ad_H(X)=[H,X]\forall X \in L_\CC(SU(2))$. This describes a general $X$, but note that in this basis, our basis vectors $\set{E_+,E_-,H}$ are eigenvectors of
$$ad_H:L(SU(2))\to L(SU(2)).$$
That is, we have chosen a basis that diagonalizes the ad map, and its eigenvalues $\set{+2,-2,0}$ are called \term{roots}.

\begin{defn}
Consider a representation $R$ of $L(SU(2))$ with a representation space $V$. We \emph{assume} that $R(H)$ is also diagonalizable. Then the representation space $V$ is spanned by eigenvectors of $R(H)$, with
$$R(H)v_\lambda = \lambda v_\lambda: \lambda \in \CC.$$
The eigenvalues $\lambda$ are called \term{weights} of the representation $R$.
\end{defn}
\begin{defn}
For such a representation, we call $E_\pm$ the \term{step operators} (cf. the ladder operators from quantum mechanics).
\end{defn}
In particular,
\begin{eqnarray*}
R(H)R(E_\pm)v_\lambda &=& (R(E_\pm)R(H) + [R(H),R(E_\pm)]) v_\lambda\\
&=&(\lambda \pm 2) R(E_\pm)v_\lambda.
\end{eqnarray*}
We see that the vectors $R(E_\pm)v_\lambda$ we got from acting on eigenvectors of $R(H)$ with the step operators are also eigenvectors of $R(H)$ with new eigenvalues $\lambda \pm 2$.

Note that a finite dimensional representation $R$ of $L(SU(2))$ must have a highest weight $\Lambda \in \CC$, or else we could just keep acting with the raising operator $E_+$ to get more linearly independent vectors. (We can play a similar trick assuming only a lowest weight-- this is what led us to the ladder of harmonic oscillator states.)
If there is a highest state, we have
\begin{eqnarray*}
R(H)v_\Lambda&=&\Lambda v_\Lambda\\
R(E_+)v_\Lambda &=&0
\end{eqnarray*}
If $R$ is irreducible, then all the remaining basis vectors of $V$ can be generated by acting with $R(E_-)$ (that is, there is only one ladder of states to construct). We get
$$V_{\Lambda-2n}=(R(E_-))^n v_\Lambda, n\in \NN.$$
What happens if we now try to raise the lowered states back up? The result is as nice as we could have hoped-- we will get back our old states, up to some normalization.
\begin{eqnarray*}
R(E_+)v_{\Lambda-2n} &=&R(E_+) R(E_-)v_{\Lambda-2n+2}\\
&=&(R(E_-)R(E_+)+[R(E_+),R(E_-)]) v_{\Lambda -2n + 2}\\
&=& R(E_-)R(E_+)v_{\Lambda -2n+2} + (\Lambda -2n+2) v_{\Lambda-2n+2}.
\end{eqnarray*}
where we have used the fact that the representation preserves the bracket structure.

Looking at the lowest-$n$ cases, we can now take $n=1$ to find
$$R(E_+)v_{\Lambda-2}=\Lambda v_\Lambda$$
and then for $n=2$,
\begin{eqnarray*}
R(E_+)v_{\Lambda-4}&=&R(E_-)R(E_+)v_{\Lambda-2}+(\Lambda-2)v_{\Lambda-2}\\
&=&\Lambda R(E_-)v_\Lambda +(\Lambda-2)v_{\Lambda-2}\\
&=&(2\Lambda-2) V_{\Lambda-2}.
\end{eqnarray*}
Proceeding by induction, we find that we can always use the relations for lower $n$ to eliminate the $R(E_+)$s at any $n$ we like and write the final result in terms of the next state up. That is,
$$R(E_+)v_{\Lambda -2n}= r_n v_{\Lambda-2n+2}.$$
Plugging this into our general equation for $R(E_+)v_{\Lambda-2n},$ we get a recurrence relation\footnote{Clearly, the left side of our original recurrence relation just becomes $r_n v_{\Lambda-2n+2}$. On the right side, we've left out a few steps. $R(E_-)R(E_+)v_{\Lambda-2n+2}=R(E_-)R(E_+)v_{\Lambda-2(n-1)}=R(E_-)r_{n-1}v_{\Lambda -2n+4} = r_{n-1}v_{\Lambda-2n+2}.$ Pull out the $v_{\Lambda-2n+2}$s everywhere and you're left with the recurrence relation.}:
$$r_n = r_{n-1}+\Lambda -2n+2$$
with the single boundary condtion that $R(E_+)v_\Lambda =0$. This implies that $r_0=0,$ so we use this to find that our recurrence relation takes the form
$$r_n=(\Lambda+1-n)n.$$

In addition, a finite-dimensional representation must also have a lowest weight $\Lambda-2N$ (recall $N$ is the dimension of the representation). That is, we have some lowest weight vector $v_{\Lambda-2N}\neq 0$ such that
$$R(E_-)v_{\Lambda-2N}=0\implies v_{\Lambda-2N-2}=0 \implies r_{N+1}=0.$$
But that vanishing means that $$(\Lambda - N)(N+1)=0 \implies \Lambda= N \in \ZZ_{\geq 0}.$$
This completes the characterization of the representation theory of $L(SU(2))$. We conclude that a finite dimensional irrep $R_\Lambda$ of $L(SU(2))$ can be described totally by a highest weight $\Lambda \in \ZZ_{\geq 0}$ and it comes with a remaining set of weights
$$S_{R_\Lambda}=\set{-\Lambda, -\Lambda+2,\ldots \Lambda-2,\Lambda}\subset \ZZ,$$
where $$\dim(R_\Lambda)=\Lambda+1.$$

\begin{exm}
Let's take some explicit cases. $R_0$ has dimension $1$ ($d_0$, the trivial representation), $R_1$ has dimension $2$ ($d_f$, the fundamental representation), and $R_2$ has dimension 3 ($d_{Adj}$, the adjoint representation).
\end{exm}

This is precisely equivalent to the theory of angular momentum in quantum mechanics but with a different normalization-- in QM, our spin states had single integer steps but with $j_{max}=n/2, n\in\NN.$ This happens because the angular momentum operators obey the same bracket structure (i.e. fail to commute) in exactly the same way as the basis elements of the Lie algebra $L(SU(2))$. 