Last time, we considered both the image and kernel of the operator $\bar Q$. We remarked that some functions will have non-trivial correlators from $O\in H_{\bar Q}$, the $\bar Q$ cohomology. That is, we are interested in functions that are in the kernel of $\bar Q$ ($\bar Q$-closed) but not in its image ($\bar Q$-exact).

For example, the transform $\bar \delta \bar \psi_i=\bar \epsilon_i \p W$ shows that $\p W$ is itself $\bar Q$ of something, i.e. in the image. Thus if our operators $O_i$ contain $\p W$ as a factor, their correlator vanishes, e.g. if 
\begin{equation}
    W(z)=\frac{z^{n+1}}{n+1}-az, \p W =z^n - a
\end{equation}, then we have non-trivial $\bar Q$-invariant operators that are polynomials subject to the condition that $z^n=a$. This tells us that these operators form a ring generated by the set of functions $\set{1,z,z^2,\ldots, z^{n-1}}.$ The ring of non-trivial SUSY operators is often called the \term{chiral ring} (chiral because we've made a choice of $\bar Q$ or $Q$).

\subsection*{Supersymmetric quantum mechanics}
There are (at least) two perspectives on QM: the canonical framework (with operators, states, wavefunctions) and the path integral framework. Today we will stay in the canonical framework and see what SUSY can teach us about quantum mechanics.

Take a worldline theory of a single bosonic field $x(t)$ and a single complex fermion $\psi(t)$ (plus its conjugate $\bar \psi$). We choose the action
\begin{equation}
    S[x,\psi,\bar \psi]=\int \bkt{\frac{1}{2} \dot x^2 + \frac{i}{2}(\bar \psi \dot \psi - \dot {\bar \psi} \psi)-\frac{1}{2} (\p h)^2 - \bar \psi \psi \p^2 h}dt,
\end{equation}
with $h=h(x(t))$ some potential function along the worldline as before. Now that we have one dimension, we have some kinetic terms in our action.

Now, this action $S[x,\psi,\bar \psi]$ is invariant under SUSY transformations
\begin{align}
    \delta x &= \epsilon \bar \psi - \bar \epsilon \psi\\
    \delta \psi &= \epsilon(i\dot x + \p h)\\
    \delta \bar \psi &= \bar \epsilon(-i\dot x + \p h).
\end{align}
We'll defer a discussion of where these transformations actually come from to when we talk about superfields. For now, we'll just take it for granted that we can write down such transformations, and note that we should check explicitly the action is indeed invariant under this set of variations.

By the Noether procedure, promoting $\epsilon\to \epsilon(t)$, we find that
\begin{equation}
    \delta S = -i \int(\dot \epsilon Q +\dot{\bar \epsilon} \bar Q)dt,
\end{equation}
where the charges
\begin{equation}
    Q=\bar \psi (i\dot x +\p h),\quad \bar Q = \psi(-i\dot x + \p h)
\end{equation}
obey the following algebra:
\begin{align*}
    \set{Q,\bar Q}x &= (Q \bar Q + \bar Q Q)x\\
    &= -Q \psi + \bar Q \bar \psi\\
    &= -(i\dot x + \p h) +(-i \dot x + \p h)\\
    &=-2i \dot x,
\end{align*}
and
\begin{align}
    \set{Q,\bar Q}\psi &= \bar Q(i\dot x + \p h)\\
    &=-i \dot \psi -\psi \p^2 h\\
    &\simeq -2i \dot \psi
\end{align}
after applying the equation of motion $\dot \psi = -i\psi \p^2 h.$
Similarly,
\begin{equation}
    \set{Q,\bar Q}=\bar \psi \simeq -2i \dot{\bar \psi}.
\end{equation}
Thus up to the fermionic equations of motion, the anticommutator of the supercharges generates time translations and so must be $\propto H$ the Hamiltonian.

To canonically quantize, we have
\begin{equation}
    p=\frac{\delta L}{\delta \dot x}=\dot x, \quad \pi = \frac{\delta L}{\delta \dot \psi}=i \bar \psi.
\end{equation}
Making the appropriate substitutions, we have a Hamiltonian
\begin{equation}
    H=p \dot x + \pi \dot \psi - L =\frac{1}{2} p^2 + (\p h)^2 + \frac{1}{2} \p^2 h (\bar \psi \psi - \psi \bar \psi).
\end{equation}
Note that classically, we could have just anticommuted $\bar \psi$ and $\psi$ to get rid of the factor of $1/2$, but after quantization we will have to be more careful about ordering ambiguities. Upon quantization (in units where $\hbar =1$), we impose canonical commutation relations,
\begin{equation}
    [x,p]= i, \quad \set{\psi,\bar \psi}=1.
\end{equation}
For $x$, as usual we shall take it to lie in the Hilbert space $\cH= L^2(\RR,dx)$, the space of square-integrable functions of a real variable, in which case
\begin{equation*}
    \hat x \Psi(x)=x \Psi(x)
\end{equation*}
and
\begin{equation*}
    \hat p \Psi(x)=-i\P{\Psi}{x}.
\end{equation*}
The relations $\set{\hat \psi,\hat{\bar \psi}}=1$ are now reminiscent of $[a,a^\dagger]=1$, the relation for the raising and lowering operators of the harmonic oscillator. In analogy to the harmonic oscillator, let's therefore define a fermionic number operator
\begin{equation}
    \hat F = \hat{\bar \psi}\hat \psi.
\end{equation}
Since $\hat F$ is the product of two fermionic operators, it is a bosonic operator, and we can then compute immediately that
\begin{equation}
    [\hat F,\hat \psi]=-\hat \psi,\quad [\hat F,\hat{\bar \psi}]=+\hat{\bar \psi}.
\end{equation}
We also let the vacuum of the fermionic system be the state $\ket{0}$, defined by
\begin{equation}
    \hat \psi \ket{0}=0.
\end{equation}
The first excited state is naturally
\begin{equation*}
    \hat{\bar \psi}\ket{0}=\ket{1}.
\end{equation*}
However, since $\set{\hat {\bar \psi},\hat {\bar \psi}}=0$ (i.e. $\hat {\bar \psi}^2=0$), there are no further excited states. Hence the Hilbert space of the fermionic states is limited to these two states (up to a phase, of course) and the entire Hilbert space of the system is therefore
\begin{equation}
    \cH = L^2(\RR,dx)\ket{0} \oplus L^2(\RR, dx)\ket{1}.
\end{equation}
We can equivalently write this as the sum 
\begin{equation*}
    \cH = \cH_B \oplus \cH_F,
\end{equation*}
a bosonic part and a fermionic part. In the quantum theory, our SUSY operators become
\begin{equation}
    \hat Q = \hat{\bar \psi}(i\hat p + \p h), \quad \hat{\bar Q}=\hat \psi (-i \hat p + \p h).
\end{equation}
The quantum Hamiltonian is
\begin{equation}
    \hat H =\frac{1}{2} \hat p^2 +(\p h)^2 +\frac{1}{2}\p^2 h(\hat{\bar \psi}\hat \psi - \hat \psi \hat{\bar \psi}).
\end{equation}
Dropping hats (so that everything is an operator by assumption), we have immediately
\begin{equation}
    \set{Q,Q}=\set{\bar Q,\bar Q}=0.
\end{equation}
However, the anticommutator of $Q$ and $\bar Q$ is nontrivial. In fact,
\begin{equation}
    \set{Q,\bar Q}=2H,
\end{equation}
which we leave as an exercise. This is why we made the choice of the particular ordering in defining $H$. This is the quantum analogue of the statement that the anticommutator of $Q$ and $\bar Q$ gave us time translation in our classical variables. In the quantum theory, they yield the Hamiltonian.

\subsection*{Supersymmetric ground states} 
As before, $\bra{\Psi}H \ket{\Psi}\geq 0$, with equality iff $Q\ket{\Psi}=0$ and $\bar Q \ket{\Psi}=0$. Therefore a state of zero energy in super-quantum mechanics (SQM) must be SUSY invariant and will then be a ground state.

If we represent the fermionic vacuum $\ket{0}\to \begin{pmatrix}1\\0\end{pmatrix}$ and the excited state $\ket{1}\to \begin{pmatrix}0\\1\end{pmatrix}$, then we can write the conditions on the operators $Q\ket{\Psi}=0,\bar Q\ket{\Psi}=0$ in a matrix representation,
\begin{equation}
    \begin{pmatrix}
        0 & 0\\
        \frac{d}{dx} + \p h & 0
    \end{pmatrix}
    \begin{pmatrix}
        f(x)\\ g(x)
    \end{pmatrix}
    =0,
    \quad
    \begin{pmatrix}
        0 & -\frac{d}{dx} + \p h\\
        0 & 0
    \end{pmatrix}
    \begin{pmatrix}
        f(x)\\ g(x)
    \end{pmatrix}
    =0.
\end{equation}
Solving, we learn that our ground state must be of the form
\begin{equation}
    \ket{\Psi}=\begin{pmatrix}
        A e^{-h(x)}\\
        B e^{+h(x)}
    \end{pmatrix}.
\end{equation}
We want a normalizable solution, so we must set at least one of $A,B$ to zero. This depends on the asyptotic behavior of $h(x)$:
\begin{itemize}
    \item if $h(x)\to_{|x|\to \infty}+\infty$ then the SUSY ground state is $\begin{pmatrix}
        A e^{-h(x)}\\
        0
    \end{pmatrix}.$
    \item if $h(x)\to_{|x|\to \infty}-\infty$ then the SUSY ground state is $\begin{pmatrix}
        0\\
        B e^{+h(x)}
    \end{pmatrix}.$
    \item if $h(x) \to_{x\to +\infty} \pm \infty$ and $h(x)\to_{x\to -\infty} \mp \infty$, then neither solution will be square-integrable, so there is no zero energy state. The ground state will have nonzero energy and SUSY is \term{spontaneously broken}. 
\end{itemize}