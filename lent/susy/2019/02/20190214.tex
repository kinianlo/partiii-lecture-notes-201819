We've been looking at supersymmetric nonlinear sigma models. Previously, our fields were maps from $x:M\to N$ where $M$ was a worldline and $N$ was some target space, a Riemannian manifold with a metric $g$. But it's clear that $M$ could be some bigger manifold, in general ``our universe.''

We said the Hilbert space for our theory was
\begin{equation}
    \cH = \cH_x \otimes \cH_\psi = \Omega^\cdot (N,\CC),
\end{equation}
the space of differential forms up to $p$-forms on $N$ equipped with inner product
\begin{equation}
    \braket{\alpha}{\beta} = \int_N \bar \alpha \wedge * \beta,
\end{equation}
where $*$ is the Hodge star operator taking $\Omega^p(N)\to \Omega^{n-p}(N)$. Explicitly, if 
\begin{equation*}
    \omega=\omega_{a_1a_2\ldots a_p}dx^{a_1}\wedge dx^{a_2}\wedge \ldots \wedge dx^{a_p},
\end{equation*} then $*\omega$ is given by
\begin{equation*}
    *\omega =\frac{\sqrt{g}}{(n-p)!} \epsilon^{a_1\ldots a_p}{}_{b_{p+1} \ldots b_n} \omega_{a_1\ldots a_p} dx^{b_{p+1}}\wedge \ldots \wedge dx^{b_n},
\end{equation*}
with indices raised by the inverse metric. We saw that our SUSY operator $Q$ then has the geometric interpretation of an exterior derivative,
\begin{equation}
    \hat Q = i\hat{\bar \psi}^a \hat p_a \leftrightarrow d,
\end{equation}
and similarly $\hat{\bar Q}$ has the interpretation of the adjoint of the exterior derivative,
\begin{equation}
    \hat {\bar Q} = -i\hat{ \psi}^a \hat p_a \leftrightarrow d^\dagger,
\end{equation}
where $\avg{\alpha,d^\dagger \beta}=\avg{d\alpha,\beta}$.

We can now fix the ordering ambiguity in $|hat H$ by demanding the SUSY algebra
\begin{equation}
2\hat H=\set{\hat Q,\hat{\bar Q}}
\end{equation}
still holds in the quantum theory. This fixes
\begin{equation}
    H=\frac{1}{2} (d^\dagger d+ dd^\dagger)=-\frac{1}{2} \Delta,
\end{equation}
where $\Delta$ is the Laplacian acting on forms. Since $d:\Omega^p \to \Omega^{p+1}, d^\dagger: \Omega^p \to \Omega^{p-1},$ it follows that $-\Delta=d^\dagger d+dd^\dagger: \Omega^p \to \Omega^p$.

To see this concretely, when acting on a function $f\in \Omega^0(N)$ (i.e. a zero-form), $d^\dagger$ simply annihilates the function (since there are no $-1$-forms) so we get
\begin{align*}
    -\Delta f &= d^\dagger d f\\
        &= d^\dagger(\p_a f dx^a)\\
        &= *d(*df)\\
        &=*d \paren{
            \frac{\sqrt{g}}{(n-1)!} g^{ab} \p_a f \epsilon_{bc\ldots d} \underbrace{dx^c \wedge \ldots \wedge dx^d}_{n-1}
        }\\
        &=\frac{*}{(n-1)!} \p_m (\sqrt{g}g^{ab} \p_a f) \epsilon_{bc\ldots d} \underbrace{dx^m \wedge dx^c \wedge \ldots \wedge dx^d}_n.
\end{align*}
But we see that there are now $n$ one-forms being wedged together, which means we must have all the $dx^1$ through $dx^n$ in some order. We can rewrite this as a totally antisymmetric tensor, with a factor of $1/g$ the determinant of the metric. Using this fact, our expression becomes
\begin{align*}
    -\Delta f&=\frac{1}{g} \p_b(g^{ab}\sqrt{g} \p_a f)*(dx^1 \wedge \ldots \wedge dx^n)\\
    &= -\frac{1}{\sqrt{g}} \p_a(\sqrt{g} g^{ab} \p_b f).
\end{align*}
What we learn is that the generalized Laplacian acting on forms reduces to the ordinary Laplacian with respect to the metric when acting on functions.

However, we now observe that acting on any form $\omega$, 
\begin{align*}
    2\bra{\omega} \hat H \ket{\omega} &= \braket{\omega}{dd^\dagger \omega} + \braket{\omega}{d^\dagger d\omega}\\
        &= ||d^\dagger \omega||^2 + ||d\omega||^2 \geq 0.
\end{align*}
A form which has equality here, $\Delta \omega = 0$, is said to be \term{harmonic}. Therefore supersymmetric ground states are in $1:1$ correspondence with $\text{Harm}^\cdot (N) = \oplus_{p=0}^n \text{Harm}^p(N)$, the space of harmonic $0$- through $p$-forms on $N$. Notice that any form $\omega \in \text{Harm}^p$ must be closed ($d\omega=0$) and co-closed ($d^\dagger \omega=0$).

\begin{thm}[Hodge's theorem] The space of harmonic $p$-forms on $N$ is in correspondence with the de Rham $p$-cohomology group,
\begin{equation}
    \text{\emph{Harm}}^p(N) \cong H^p_{dR}(N)
\end{equation}
where
\begin{equation}
    H^p_{dR}(N)=\set{\omega \in \Omega^p(N)\text{ s.t. }d\omega=0}/\set{\omega=d\alpha} = \text{\emph{ker}}(d:\Omega^p\to \Omega^{p+1})/\text{\emph{im}}(d:\Omega^{p-1}\to \Omega^p).
\end{equation}
\end{thm}
In de Rham cohomology, $\omega$ is specified up to $\omega \sim \omega +d\alpha$ (i.e. we only care about $\omega$ up to the addition of some exact $d\alpha$). The role of the co-closure condition, $d^\dagger \omega=0$, is to select a unique representative. If $d\omega = d^\dagger \omega=0$, then we our freedom becomes $\omega \sim \omega +d\alpha$ where $d^\dagger d\alpha =0$, and the only solutions are $\alpha=0$.%
    \footnote{This is equivalent to the procedure in electromagnetism where we have a potential with a gauge symmetry $A\sim A+d\lambda$, and we fix the gauge by requiring that $d^\dagger A = \p^\mu A_\mu =0$.}
Thus the space of SUSY ground states is $\cong H^\cdot_{dR}(N)$.

Thinking back to our discussion of the Witten index, we see that
\begin{equation}
    I_W = \Tr((-1)^F e^{-\beta H}) = n_B -n_F =\sum_{p=0}^n (-1)^p \dim(H^p_{dR}(N)).
\end{equation}
But this is very interesting because this final expression is precisely $\chi(N)$, the \term{Euler character} of $N$. Thus the space of SUSY ground states has a close relation to some topological information about the space our states live in.

To motivate de Rham cohomology a bit more, suppose $C_p$ is a $p$-cycle in $N$ without boundary. Stokes's Theorem in the vector calculus language says that 
\begin{equation*}
    \int_S(\curl \vec A) \cdot d\vec S=\oint_C \vec A \cdot d\vec l.
\end{equation*}
But we can generalize this to $p$-forms:
\begin{equation}
    \int_{D_{p+1}} d\omega = \int_{C_p} \omega
\end{equation}
if $\p D_{p+1}=C_p$. That is, we can relate the integral in some region $D_{p+1}$ to the value of the form integrated over the boundary $C_p$. However, if $\omega\in H^p_{dR},$ then $d\omega=0 \implies \int \omega=0$ if $C_p$ is the boundary of some $D_{p+1}$.

Furthermore,
\begin{equation}
    \int_{C_p} \omega +d\alpha =\int_{D_{p+1}} d\omega + \int_{D_{p+1}} d^2 \alpha,
\end{equation}
where this second term vanishes since $d$ is nilpotent. Thus we arrive at de Rham's theorem:
\begin{thm}[de Rham]
    \begin{equation}
        H_{dR}^p(N) \cong H_p(N),
    \end{equation}
    where $H_p(N)$ denotes the $p$th homology group, the set \{$p$-cycles in $N$ with no boundary$\}/\{p$-cycles that are the boundary of some $(p+1)$-cycle\}.
\end{thm}

For instance, if $N=S^n$, then $\dim(H_{dR}^0(S^n))=1$. We can also find that  $\dim(H^p_{dR}(S^n))=0$ for $p\neq 0,n$ since we can contract any loop (e.g. an $S^1$) to a point on $S^n$. And then we have $\dim(H^n_{dR}(S^n))=1$, i.e. there is one non-trivial  ``wrapping'' of $S^n$ by an $S^n$.

For $n=\Sigma_g$ a handlebody with genus $n$ (i.e. $n$ donuts glued together) we have instead $H^0(\Sigma_g)=\CC$, $H^1(\Sigma_g)=\CC^{2g}$ and $H^2(\Sigma_g)=\CC$ (dimensions $1,2g,$ and $1$).

The Euler character for the $n$-sphere is
\begin{equation}
    \chi(S^n)=\begin{cases}
        2 & \text{if $n$ even}\\
        0 & \text{if $n$ odd},
    \end{cases}
\end{equation}
and for $\Sigma_g$ it is $\chi(\Sigma_g)=2-2g$.

For the path integral, $\chi(N)=\int e^{-S[x,\psi]}\cD x \cD \psi \cD \bar \psi$, where all fields are periodic with period $\beta$. Now if we take the whole action, we see that our whole action is supersymmetrically trivial:
\begin{align}
    S &= \int \frac{1}{2} g_{ab} \dot x^a \dot x^b+\frac{1}{2} g_{ab} \bar \psi^a \nabla_t \psi^b + \frac{1}{4} R_{abcd} \bar \psi^a \psi^b \bar \psi^c \psi^d dt\\
    &= \bar Q \bkt{ \oint \frac{g_{ab} \bar \psi^a}{2}(i\dot x^b +\Gamma^b_{cd} \bar \psi^c \psi^d) dt.}
\end{align}
We therefore learn that the path integral is independent of $\beta$.