Last time, we considered fields in some spacetime and chose Gaussian normal coordinates in order to write (for variations of the fields $x^a=x^a_0+\delta x^a(\tau), \psi^a=\psi^a_0 +\delta \psi^a(\tau)$,
\begin{equation*}
    g_{ab}(x)=\delta_{ab} -\frac{1}{3} R_{acbd} (x_0) \delta x^c \delta x^d + O(\delta x^3)
\end{equation*}
and a connection
\begin{equation*}
    \Gamma^a_{bc}(x) = \p_d \Gamma^a_{bc}(x_0) \delta x^d = -\frac{1}{3} (R^a{}_{bcd}(x_0) +R^a{}_{cbd}(x_0))\delta x^d + O(\delta x^2).
\end{equation*}

So we have the quadratic action
\begin{equation}
    S^{(2)}[x_0,\psi_0,\delta x, \delta \psi]=\oint \paren{-\frac{1}{2} \delta x^a \delta_{ab} \frac{d^2}{d\tau^2}\delta x^b +\frac{1}{2} \delta \psi^a \delta_{ab} \frac{d}{d\tau} \delta \psi^b -\frac{1}{4} R_{abcd} \psi_0^a \psi_0^b \delta x^c \frac{d\delta x^d}{d\tau}} d\tau.
\end{equation}
For any fixed $(x_0^a,\psi_0^a)$, this is a free action, so the path integral over fluctuations gives
\begin{equation}
    \int e^{-S[x_0,\psi_0,\delta x , \delta \psi]}\cD \delta x \cD \delta \psi = \frac{\sqrt{\det'(\p_\tau \delta^b_a)}}{\sqrt{\det'(-\p^2\tau \delta^a_b - \mathcal{R}^a{}_b (x_0,\psi_0)\p_\tau)}}
\end{equation}
where $\mathcal{R}^a{}_b=R^a{}_{bcd}(x_0) \psi_0^c \psi_0^d$ and $\det'$ means without zero modes, i.e. we haven't yet done the integrals over $(x_0,\psi_0).$

We can split up the denominator by pulling out a $\p_\tau$ to find
\begin{equation}
    \int e^{-S[x_0,\psi_0,\delta x , \delta \psi]}\cD \delta x \cD \delta \psi = \frac{\sqrt{\det'(\p_\tau \delta^b_a)}}{\sqrt{\det'(\delta^a{}_b \p_\tau)}\sqrt{\det'(-\delta^a{}_b \p_\tau - \mathcal{R}_a{}^b)}}
    =\frac{1}{\sqrt{\det'(-\delta^a{}_b \p_\tau - \mathcal{R}^a{}_b)}}.
\end{equation}
Notice that the matrix $\mathcal{R}^a{}_b$ is an antisymmetric $n\times n$ matrix (since we contracted over two indices in the original Riemann tensor, and $R^a{}_{bcd}$ was already antisymmetric in the first two indices) and $n=2m$. We therefore decompose the tangent space $TN|_{x_0}$ into $m$ 2-dimensional subspaces on which $\mathcal{R}^a{}_b|_i$ takes the form
\begin{equation}
    \mathcal{R}^a{}_b|_i =\begin{pmatrix} 0 & \omega_i \\ -\omega_i & 0\end{pmatrix}.
\end{equation}
Let $-D_i$ be the restriction of $-\delta^a{}_b \p_\tau -\mathcal{R}^a{}_b$ to this 2D subspace.

We expand
\begin{equation}
    \delta x^a(\tau)=\sum_{k\neq 0} \delta x_k^a e^{2\pi i k \tau}.
\end{equation}
Then the eigenvalues of $-D_i$ on this subspace are $-2\pi i k \pm \omega_i$ for $k\in \ZZ,k\neq 0$ (where the first term comes from acting on a Fourier mode with $\p_\tau$ and the second comes the eigenvalues of $\mathcal{R}^a{}_b|_i$ being $\pm\omega$). Therefore
\begin{align*}
    \det(-D_i)&=\prod_{k\neq 0} (-2\pi i k+\omega_i)(-2\pi i k -\omega_i)\\
    &= \prod_{k\neq 0}(-(2\pi k)^2-\omega_i^2)\\
    &= \prod_{k=1}^\infty (2\pi k)^4 \prod_{k=1}^\infty \paren{1+\frac{\omega_i^2}{(2\pi k)^2}}^2,
\end{align*}
where the rewriting in the last line has come from changing the $k\neq 0$ product to a product over $k=1\to\infty$. 

This is clearly divergent thanks to the first factor. However, we can regularize this, e.g. using zeta-function regularization. We find that
\begin{equation}
    \prod_{k=1}^\infty (2\pi k)^4 = (4\pi^2)^{2\zeta(0)}e^{-2\zeta'(0)}=1.
\end{equation}
The important factor is then
\begin{equation*}
    \prod_{k=1}^\infty \paren{1+\frac{\omega_i^2}{(2\pi k)^2}}^2,
\end{equation*}
and we recall that 
\begin{equation*}
    \sinh(z)=z\prod_{k=1}^\infty \paren{1+\frac{z^2}{\pi^2 k^2}},
\end{equation*}
so after regularization, we have that $z=\omega_i^2/2$ and (by direct comparison with the expansion of $\sinh(z)$) our determinant term can be written as
\begin{equation}
    \sqrt{\det{}'(-D_i)}=\frac{\sinh(\omega_i/2)}{(\omega_i/2)}.
\end{equation}
We now see that
\begin{align}
    I_W&=\text{index}(\slashed\nabla)=\int \prod_{i=1}^\infty \frac{\omega_i/2}{\sinh(\omega_i/2)} d^n x_0 d^n\psi_0\\
        &= \int \det \paren{\frac{\mathcal{R}^a{}_b(x_0,\psi_0)/2}{\sinh(\mathcal{R}^a{}_b}(x_0,\psi_0)/2)}d^n x_0 d^n \psi_0.
\end{align}
where $\slashed{\nabla}$ denotes the Dirac operator on $N$. But by our regular Grassmann tricks, we must have precisely $n$ factors of $\psi_0$ in order for this integral to be non-vanishing. Thus
\begin{equation}
    I_W = \int_N \det \paren{\frac{\mathscr{R}/2}{\sinh \mathscr{R}/2}}.
\end{equation}
where $\mathscr{R}^a{}_b=R^a{}_{bcd}(x) dx^c \wedge dx^d$ is a curvature two-form. This is the Aatiyah-Singer index theorem.

\subsection*{Supersymmetric QFT}
If we had a $d$-dimensional theory that is Lorentz invariant, we must complete the supersymmetry algebra $\set{Q,Q^\dagger}=2H$. The Hamiltonian now comes with nontrivial kinetic terms and is part of the $d$-momentum multiplet $P_\mu$, so we need further supercharges. If we want to preserve $Q^\dagger =(Q)^\dagger$, then these supercharges must have the same spin, and so must each have spin $1/2$.

Specifically, the SUSY algebra in $d$-dimensions is
\begin{equation}
    \set{Q_\alpha,Q_\beta^\dagger}=2\gamma^\mu_{\alpha\beta} P_\mu,
\end{equation}
where $\alpha,\beta$ are spinor indices and $\gamma^\mu$ is a Dirac $\gamma$ matrix. We'll mostly be concerned with $d=2$, where Dirac spinors have $2^{(d/2)}=2$ complex components. Thus we can write $\psi=\begin{pmatrix}\psi_-\\\psi_+\end{pmatrix}$. With coordinates $(t,s)\in \RR^2$ and Minkowski metric $\eta_{\mu\nu}=\text{diag}(+,-)$, we can represent the Dirac $\gamma$s as
\begin{equation}
    \gamma^t = 
    \begin{pmatrix} 
    0 & 1\\
    1& 0
    \end{pmatrix},
    \quad
    \gamma^s = 
    \begin{pmatrix} 
    0 & -1\\
    1 & 0
    \end{pmatrix}.
\end{equation}
These obey the Clifford algebra $\set{\gamma^\mu, \gamma^\nu}=2\eta^{\mu\nu}$. The action for a free, massless Dirac spinor in $d=2$ is then
\begin{equation}
    S[\psi]=\frac{1}{2\pi}\int_{\RR^2} i\bar \psi \slashed{\p} \psi d^2 x
\end{equation}
where $\slashed{\p}=\gamma^\mu \p_\mu$ and $\bar \psi=\psi^\dagger \gamma^t$. We can of course plug in the explicit form of the spinors and $\gamma$ matrices, and we find that
\begin{equation}
    S[\psi]=\frac{1}{2\pi}\int {\RR^2} i\bar \psi_- (\p_t +\p_s)\psi_- +i\bar \psi_+(\p_t-\p_s)\psi_+ dtds,
\end{equation}
so we see that the spinor components decouple. Classically,
\begin{equation}
    (\p_t + \p_s)\psi_-=0\implies \psi_-(t,s)=f(t-s)
\end{equation}
represents a right-moving mode, while
\begin{equation}
    (\p_t - \p_s)\psi_+=0\implies \psi_+(t,s)=f(t+s)
\end{equation}
is a left-moving mode. Under an $SO(1,1)$ transformation, i.e.
\begin{equation}
    \begin{pmatrix}t\\ s\end{pmatrix} \mapsto
    \begin{pmatrix}
    \cosh\gamma & \sinh\gamma\\
    \sinh\gamma & \cosh\gamma
    \end{pmatrix}
    \begin{pmatrix}t\\ s\end{pmatrix}
\end{equation}
with $\gamma$ the usual (real) rapidity, the spinor components transform as
\begin{equation}
    \psi_\pm \mapsto e^{\pm \gamma/2}\psi,\quad \bar\psi_\pm \mapsto e^{\pm \gamma/2}\bar\psi.
\end{equation}