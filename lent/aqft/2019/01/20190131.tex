Today we'll finish our discussion of the zero-dimensional path integral by introducing fermions to our theory. To model fermions, we will introduce Grassmann variables,%
    \footnote{We've seen these in \emph{Supersymmetry} already.
    }
i.e. a set of $n$ elements $\set{\theta_a}_{a=1}^n$ obeying anticommutation relations,
\begin{equation}
    \theta_a \theta_b = -\theta_b \theta_a.
\end{equation}
Note also that for (complex) scalars $\phi_b\in \CC$,
\begin{equation}
    \theta_a \phi_b = \phi_b \theta_a,
\end{equation}
i.e. scalars commute with Grassmann variables. In addition, $\theta^2_a =0$ by the anticommutation relations, which implies that the Taylor expansion of any (analytic) function of $n$ Grassmann variables can be written in a finite number of terms. That is, polynomials in Grassmann variables are forced to terminate since at some point we run out of distinct Grassmann variables to multiply. A general function $F(\theta)$ can be written
\begin{equation}
    F(\theta)=f+\rho_a \theta_a +\frac{1}{2!} g_{ab} \theta_a \theta_b + \ldots + \frac{1}{n!} h_{a_1\ldots a_n} \theta_{a_1}\ldots \theta_{a_n}.
\end{equation}
Note that the coefficients $\rho,g,\ldots,h$ are totally antisymmetric under interchange of indices.

We also want to define differentiation and integration of these guys. Derivatives $\P{}{\theta_a}$ anticommute with Grassmann variables, i.e.
\begin{equation}
    \paren{\P{}{\theta_a}\theta_b + \theta_b \P{}{\theta_a}} (*) = (\delta_{ab})(*)
\end{equation}
where the derivative in the first term acts on everything coming after. This leads us to a modified Leibniz rule.

To define integration, note that for a single Grassmann variable $\theta$, a function takes the form
\begin{equation}
    F(\theta)=f+\rho \theta,
\end{equation}
so we just need to define $\int d\theta$ and $\int d\theta \,\theta$. If we require translational invariance, i.e.
\begin{equation}
    \int d\theta(\theta+\eta)=\int d\theta \,\theta,
\end{equation}
then
\begin{equation}
    \eta \int d\theta = 0 \implies \int d\theta =0.
\end{equation}
We can then choose the normalization so that $\int d\theta \, \theta = 1$. Note the similarity between differentiation and integration (i.e.  $\int d\theta\,\theta =1=\P{}{\theta}\theta$). This process is called \term{Berezin integration}. Using these rules, we also find that
\begin{equation}
    \int d\theta \P{}{\theta} F(\theta)=0,
\end{equation}
since the term linear in $\theta$ will go to a constant by the derivative and be killed by the integral, and any constant terms will be killed by the derivative. Either way the result is zero.%
    \footnote{This is like the worst version of the fundamental theorem of calculus.}

Suppose now we have $n$ Grassmann variables. Then the only nonvanishing integrals involve exactly one power of each integration variable, e.g.
\begin{equation}
    \int d^n \theta\, \theta_1 \theta_2 \ldots \theta_n = \int d\theta_n d\theta_{n-1}\ldots d\theta_1 \, \theta_1 \theta_2 \ldots \theta_n = 1.
\end{equation}
In general we can just anticommute the Grassmann variables until they're in the right order, possibly picking up an overall minus sign for the parity of the permutation. That is,
\begin{equation}
    \int d^n\theta \,\theta_{a_1}\theta_{a_2}\ldots \theta_{a_n} = \epsilon^{a_1 a_2 \ldots a_n},
\end{equation}
where $\epsilon$ is the totally antisymmetric symbol with value $+1$ for even permutations of $1,2,\ldots,n$, $-1$ for odd permutations, and $0$ if any indices are repeated.%
    \footnote{If you've worked with differential forms, this sort of anticommuting construction should seem very familiar.}

What if we now make a change of variables $\theta_a' = A_{ab} \theta_b$? Then
\begin{align}
    \int d^n \theta \theta'_{a_1} \theta'_{a_2} \ldots \theta'_{a_n} &= A_{a_1b_1}\ldots A_{a_nb_n} \underbrace{\int d^n \theta \, \theta_{b_1} \ldots \theta_{b_n}}_{\epsilon^{b_1\ldots b_n}}\\
    &= \det A \,\epsilon^{a_1\ldots a_n}\\
    &= \det A \int d^n \theta' \,\theta'_{a_1} \ldots \theta'_{a_n}
\end{align}
We conclude that under a change of variables, the integration measures are related by
\begin{equation}
    d^n\theta = \det A \,d^n \theta'.
\end{equation}
Note that this is the opposite of the convention for scalars, where
\begin{equation}
    \phi'_a = A_{ab} \phi_b \implies d^n \phi =\frac{1}{|\det A|}d^n \phi'.
\end{equation}

\subsection*{Free fermion field theory} Consider $d=0$, with two fermion fields $\theta_1,\theta_2$. The action must be bosonic (scalar), so the only possible nonconstant action is
\begin{equation}
    S(\theta)=\frac{1}{2}A \theta_1 \theta_2,\quad A\in \RR.
\end{equation}
Then the path integral is
\begin{equation}
    Z_0 = \int d^2 \theta \, e^{-S(\theta)/\hbar}=\int d^2\theta \paren{ 1-\frac{A}{2\hbar}\theta_1\theta_2} = -\frac{A}{2\hbar},
\end{equation}
where the exponential has terminated thanks to our Grassmann variables.%
    \footnote{That is, since any powers of $\theta_i$ greater than one are equal to zero, analytic functions of a finite number of Grassmann variables always terminate at some finite order.}

Suppose now we have $n=2m$ fermion fields $\theta_a$. Then our action might be quadratic in the fields,
\begin{equation}
    S=\frac{1}{2} A_{ab} \theta_a \theta_b
\end{equation}
with $A$ an antisymmetric matrix, and the path integral is then
\begin{align*}
    Z_0 &= \int d^{2m}\theta\, e^{-S(\theta)/\hbar} = \int d^{2m} \theta \sum_{j=0}^{m} \frac{(-1)^j}{(2\hbar)^j j!} (A_{ab}\theta_a \theta_b)^j\\
    &= \frac{(-1)^m}{(2\hbar)^m m!} \int d^{2m}\theta A_{a_1 a_2} A_{a_3 a_4} \ldots A_{a_{2m-1} a_{2m}} \theta_{a_1} \theta_{a_2} \ldots \theta_{a_{2m}}\\
    &= \frac{(-1)^m}{(2\hbar)^m m!} \epsilon^{a_1 a_2 \ldots a_{2m}} A_{a_1 a_2} A_{a_3 a_4} \ldots A_{a_{2m-1} a_{2m}}\\
    &= \frac{(-1)^m}{\hbar^m} \text{Pf}(A),
\end{align*}
where $\text{Pf}(A)$ is the \term{Pfaffian} of the matrix $A$, defined by
\begin{equation}
    \text{Pf}(A)\equiv \frac{1}{2^m} \epsilon^{a_1 a_2 \ldots a_{2m}} A_{a_1 a_2} A_{a_3 a_4} \ldots A_{a_{2m-1} a_{2m}},
\end{equation}
which we will show on the examples sheet is in fact $\pm \sqrt{\det A}.$ For example, $\text{Pf}\begin{pmatrix}0 & -a \\ a & 0 \end{pmatrix} = a$. In the computation of $Z_0$, the sum over $j$ terminates at $m$ since we run out of Grassmann variables, and the only term in the integral that is nonzero is the one which contains all the $2m$ Grassmann variables.

Using this property, we find that for fermionic fields,
\begin{equation}
    Z_0 = \pm \sqrt{\frac{\det A}{\hbar^n}}
\end{equation}
with $A$ antisymmetric, whereas for bosonic fields with some symmetric mass matrix $M$,%
    \footnote{That is, for an action $S=\frac{1}{2}M_{ab}\phi_a \phi_b$.}
we have
\begin{equation}
    Z_0=\sqrt{\frac{(2\pi \hbar)^n}{\det M}}.
\end{equation}

We can now introduce an external source function to our action, a Grassmann-valued $\set{\eta_a}$, such that the new action is
\begin{equation}
    S(\theta,\eta)=\frac{1}{2} A_{ab} \theta_a \theta_b + \eta_a \theta_b.
\end{equation}
Taking care to respect the anticommutation relations and completing the square as before, we can rewrite the action as
\begin{equation}
    S(\theta,\eta)=\frac{1}{2}(\theta_a +\eta_c(A^{-1})_{ca}) A_{ab}(\theta_b +\eta_d(A^{-1})_{db}) +\frac{1}{2} \eta_a (A^{-1})_{ab} \eta_b.
\end{equation}
We can make a change of variables using the translational invariance of $\theta_a$ and pull out the constant factor to find
\begin{equation}
    Z_0(\eta)=\exp\paren{-\frac{1}{2\hbar}\eta^T( A^{-1}) \eta} Z_0(\eta=0).
\end{equation}
This allows us to get propagators by taking derivatives with respect to the source $\eta$, as we are wont to do:
\begin{equation}
    \avg{\theta_a \theta_b}
    = \frac{\hbar^2}{Z_0(0)}\frac{\p^2 Z_0(\eta)}{\p \eta_a \p \eta_b}|_{\eta=0} 
    = \hbar(A^{-1})_{ab}.
\end{equation}
We see that the propagator is proportional to the inverse of the bilinear part of the action for Grassmann variables. This is just like the bosonic case.