Today we'll begin our discussion of renormalization and why infinities might not be so scary after all (cf. Skinner \textsection 5.1). Let us consider $\phi^4$ theory:
\begin{equation}
    S[\phi]=\int d^4x \bkt{
        \frac{1}{2} \p_\mu \phi \p^\mu \phi+\frac{1}{2} m^2 \phi^2 +\frac{\lambda}{4!}\phi^4
    }.
\end{equation}
In momentum space, the full propagator $\tilde{\gv \Delta}(p^2)$ takes the form
\begin{equation}
    \tilde{\gv \Delta}(p^2)=\int d^4x e^{-ip\cdot x}\avg{\phi(x) \phi(0)}_{\text{connected}}.
\end{equation}

Schematically, we can represent the propagator as the following sum of diagrams: (diagram to be inserted)
%insert diagram
or equivalently the following geometric series:
\begin{align*}
    \tilde{\gv \Delta}(p^2) &= 
        \frac{1}{p^2+m^2}+\frac{1}{p^2+m^2} \Pi(p^2) \frac{1}{p^2+m^2} + \frac{1}{p^2+m^2} \Pi(p^2) \frac{1}{p^2+m^2} \Pi(p^2) \frac{1}{p^2+m^2}+\ldots \\
        &=\frac{1}{p^2+^2-\Pi(p)^2}
\end{align*}
where $\Pi(p^2)$ is called the self-energy. Note that $\Pi$ contributes to the quantum effective action $\Gamma$. Perturbatively, we get contributions from diagrams like the following: (diagram here)
%diagrams here
Note that dashed lines are omitted from the computation of the 1PI factor $\Pi(p^2)$ since they are external propagators.

One of the simplest diagrams we can draw is the one-loop diagram, and it corresponds to the amplitude
\begin{equation}
    -\frac{\lambda}{2}\int^\Lambda \frac{d^4k}{(2\pi)^4} \frac{1}{k^2+m^2}.
\end{equation}
This is divergent. To see this, let us introduce an ultraviolet (UV) cutoff so that we integrate over only modes with $|k|<\Lambda$. Since the integral depends only on $k^2$, we get
\begin{equation}
    -\frac{\lambda S_4}{2(2\pi)^4} \int_0^\Lambda \frac{k^3 dk}{k^2+m^2}=-\frac{\lambda S_4 m^2}{4(2\pi)^4} \int_0^{\Lambda^2/m^2}\frac{u du}{1+u} = -\frac{\lambda}{32\pi^2} \bkt{\Lambda^2 - m^2 \log \paren{1+\frac{\Lambda^2}{m^2}}},
\end{equation}
where $d^dk= S_d |k|^{d-1} d|k|$ with $S_d=\frac{2\pi^{d/2}}{\Gamma(d/2)}$ and we've made the substitution $u=k^2/m^2$ to perform the integral. Here, $S_4=2\pi^2$. With all these substitutions, we reach an amplitude that diverges as $\Lambda\to\infty$.

Suppose we allow the coupling to depend on $\Lambda$ by adding ``counterterms'' to the action. That is,
\begin{equation}
    S[\phi] \to S[\phi]+(\hbar)S^{CT}[\phi,\Lambda].
\end{equation}
For instance, we might define a set of counterterms as
\begin{equation}
    S^{CG}[\phi,\Lambda]=\int d^4x \bkt{
        \frac{\delta Z(\Lambda)}{2} \p_\mu \phi \p^\mu \phi +\frac{1}{2} \delta m^2(\Lambda) \phi^2 +\frac{\delta \lambda(\Lambda)}{4!} \phi^4
    }.
\end{equation}
Note that 
These correspond to some new vertices and thus new contributions to $\Pi(p^2)$: see diagram.%diagram here

At 1 loop, the 1PI contribtuion becomes
\begin{equation}
    \Pi^{\text{1-loop}}(p^2)=-p^2 \delta Z -\delta m^2 -\frac{\lambda}{32\pi^2}\bkt{\Lambda^2-m^2\log \paren{1+\frac{\Lambda^2}{m^2}}}.
\end{equation}
At two loops, the counterterm diagrams
%diagrams here
must also be included. Now we can tune the parameters $\delta Z, \delta m^2, \delta \lambda$ in our counterterms to cancel the divergences. In other words, we \term{renormalize} $\phi,m^2,\lambda$.

\subsection*{On-shell renormalization scheme}
The need to ``regulate'' the theory by cancelling divergences does not uniquely determine the counterterms, so we impose additional renormalization conditions, which we call a \term{scheme}. It will turn out that physical observables do not depend on our choice of scheme.

The on-shell scheme is as follows. We fix $\delta Z, \delta m^2, \delta \lambda$ by requiring that
\begin{enumerate}
    \item[1.] $\tilde {\gv \Delta}(p^2)$ has a simple pole at some experimentally observable mass, i.e. $-p^2=m^2_{\text{phys}}$ (in Euclidean signature)
    \item[2.] The residue of this pole is equal to $1$.
\end{enumerate}
Therefore since
\begin{equation}
    \tilde {\gv \Delta}(p^2)=\frac{1}{p^2+m^2-\Pi(p^2)},
\end{equation}
our conditions say that
\begin{enumerate}
    \item[1.] $\Pi(-m^2_{\text{phys}}=m^2-m^2_{\text{phys}}$, which is zero if we want the mass in $\cL$ to equal $m_{\text{phys}}$ at this order of counterterm.
    \item[2.] $\P{\Pi}{p^2}|_{p^2}=-m^2_{\text{phys}}=0$ (by L'H\^opital's rule).
\end{enumerate}
These tell us that 
\begin{gather}
    2. \implies \delta Z=0,\\
    1. \implies \delta m^2= -\frac{\lambda}{32\pi^2} \bkt{\Lambda^2-m^2\log \paren{1+\frac{\Lambda^2}{m^2}}}.
\end{gather}
Note that $\delta Z=0$ and $\pi(p^2)=0 \forall p^2$ is due to the one-loop diagram not depending on $p^2$. If we instead tried to construct the counterterms to the two-loop diagram we would get $\delta Z\neq 0$ since the integral depends on $p$.

UV divergences are not too hard to spot-- we saw that
\begin{equation}
    \int^\Lambda \frac{d^4k}{(2\pi)^4} \frac{1}{k^2 +m^2} \sim \Lambda^2,
\end{equation}
and generically
\begin{equation}
    \int^\Lambda \frac{d^nk}{k^m}\sim \begin{cases}
        \Lambda^{n-m}, & n\neq m\\
        \log \Lambda, & n = m
    \end{cases}.
\end{equation}