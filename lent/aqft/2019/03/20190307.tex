\subsection*{Gauge fixing, cont.}
We had a general gauge-fixing expression
\begin{equation*}
    \int \cD A e^{-S[A]}=\paren{\int \cD \alpha} \int \cD A e^{-S[A]} \delta(G[A]) \det \paren{\frac{\delta G[A^\alpha]}{\delta \alpha}},
\end{equation*}
and we can choose a particular gauge-fixing function $G[A]=\p^\mu A_\mu^a(x) -\omega^a(x)$, just as in QED. We then integrate over an extra parameter $\omega^a(x)$ with Gaussian weight $1/2\xi$,
\begin{equation}
    \int \cD A e^{-S[A]}\det \paren{\frac{\delta G[A^\alpha]}{\delta \alpha}} \exp \bkt{\int d^4x \frac{1}{2\xi} (\p^\mu A_\mu)^2}
\end{equation}
The gauge-transformed field
is then
\begin{equation}
    (A^\alpha)^a_\mu = A^a_\mu + \frac{1}{g} D_\mu \alpha^a
\end{equation}
such that the variation is%
    \footnote{See Skinner to make sense of this expression. We've left off a delta-function.}
\begin{equation}
    \frac{\delta G[A^\mu]}{\delta \alpha}=\frac{1}{g} \p^\mu D_\mu.
\end{equation}
The Faddeev-Popov method tells us to express the functional determinant as a path integral over new Grassmann fields $c,\bar c$ which transform in the adjoint representation of the gauge group. These new fields anticommute, but otherwise transform as Lorentz scalars (spin zero). These $c,\bar c$ fields do not represent physical particles (i.e. valid in/out states) and should be thought of as constraints. They are called \term{Faddeev-Popov ghosts.}%
    \footnote{You might be concerned about spin-statistics here. Shouldn't anticommuting variables be half-integer spin? Normally, yes, but this is only true when our variables represent physical observables. The ghosts are really constraints in our theory and cannot be observed. Probably why we call them ghosts!}
One can then write a ``ghost Lagrangian,''
\begin{align}
    \cL_\text{gh}&=\bar c^a( \p^2 \delta^{ac} +g f^{abc} \p^\mu A_\mu^b) c^c\\
        &= \bar c^a \p^\mu D_\mu^{ac} c^c,
\end{align}
in terms of a new covariant derivative
\begin{equation}
    D_\mu^{ac}\equiv \p_\mu \delta^{ac} +g f^{abc} \p^\mu A_\mu^b.
\end{equation}
Note that this one has upper (generator) indices, like the ghosts themselves. From the ghost Lagrangian, we can write down a ghost propagator,
\begin{equation}
    \avg{c^a(x) \bar c^b(y)} =\int \frac{d^4k}{(2\pi)^4} \frac{\delta^{ab}}{k^2} e^{-ik\cdot (x-y)}
\end{equation}
so that a ghost propagator is associated with a $\delta^{ab}/k^2$ and we also get a three-point vertex (involving $c,\bar c$, and a gauge field) with amplitude $-g f^{abc}p^\mu$.

If these ghost fields do not represent physical particles, why do we need them in our theory?
\begin{itemize}
    \item The Feynman rules for fermions and gauge bosons only (no ghosts) lead to unphysical gauge field polarizations, e.g. the following diagrams.
    %diagram
    At tree level we can neglect unphysical polarizations by focusing on physcial in/out states, but loops are a problem.
    \item Ghosts cancel the unphysical contributions, e.g. the sum
    %diagram
    is free from contributions of unphysical polarizations.
    \item It is possible to avoid the ghosts by using a Lorentz non-invariant gauge-fixing condition, e.g. axial gauge like $n\cdot A^a=0 \forall a$ with $n$ a unit vector. Thus $G[A^\alpha]=n\cdot (A^\alpha)^a$. The gauge transformation of such a field is then
    \begin{equation}
        n\cdot (A^\alpha)^a = n\cdot A^a +\frac{1}{g} n \cdot \p \alpha^a +f^{abc} +f^{abc} n\cdot A^b \alpha^c =\frac{1}{g} n\cdot \p \alpha^a,
    \end{equation}
    which is independent of $A$ as in QED. (We have simplified by the original gauge condition $n\cdot A=0$.)  Thus $\frac{\delta G[A^\alpha]}{\delta \alpha}$ is independent of $A$. However, the downside of doing this is that we get a more complicated propagator for the gauge field. We see that breaking Lorentz invariance has its costs.
\end{itemize}

\subsection*{BRST symmetry}
This symmetry is a constraint on physical states, named for Becchi-Rouet-Storz-Tyutin. In our general gauge theory, we have a Lagrangian in Euclidean space which takes the form
\begin{equation}
    \cL=\frac{1}{4}(F^a_{\mu\nu})^2 +\bar \psi(\slashed{D}+m)\psi +\frac{1}{2\xi}(\p^\mu A^a_\mu)^2 + \bar c \p^\mu D_\mu c,
\end{equation}
now including the ghosts. This theory has a term that looks like the $F_{\mu\nu}$ field strength tensor of QED, a fermion coupling, an extra kinetic term for the gauge field (useful in gauge fixing), and the ghost term.
Without loss of generality, we may then introduce an auxiliary (i.e. non-dynamical) field $B^a$, modifying our Lagrangian to
\begin{equation}
    \cL = \frac{1}{4}(F_{\mu\nu}^a)^2 + \bar \psi(\slashed{D} + m) \psi +\frac{1}{2\xi} (B^a)^2 -B^a \p^\mu A_\mu^a  +\bar c^a \p^\mu D_\mu^{ac} c^c,
\end{equation}
% \begin{equation}
%     \cL = \frac{1}{4}(F_{\mu\nu}^a)^2 + \bar \psi(\slashed{D} + m) \psi +\frac{1}{2\xi} (B^a)^2 -B^a \p^\mu A_\mu^a  +\bar c \p^\mu D_\mu c,
% \end{equation}
where completing the square and integrating out $B^a$ yields the original Lagrangian.
%this didn't have the indices on the covariant derivative term-- I have added them

This new Lagrangian is then invariant under the global BRST transformation, written in infinitesimal form as
\begin{align}
    \delta A_\mu^a &= \epsilon D_\mu^{ac} c^c\\
    \delta \psi &= ig \epsilon c^a t^a \psi\\
    \delta c^a &= -\frac{1}{2} g\epsilon f^{abc} c^b c^c\\
    \delta \bar c^a &= \epsilon B^a\\
    \delta B^a &= 0,
\end{align}
where $\epsilon$ must be an infinitesimal Grassmann (anticommuting) quantity. Here, the $t^a$s are the generators of the gauge group as before and $f^{abc}$ are the corresponding structure constants.

This transformation may seem a bit ad hoc, but in fact it will help us to make sense of the ghosts. Notice that the transformation of $\bar c^a$ is related to the auxiliary field $B^a$.
The transformations of the fermion field $\psi$ and the gauge field $A_\mu^a$ are just local gauge transformations with $\alpha^a(x)=g\epsilon c^a(x)$, so the invariance of $(F_{\mu\nu}^a)^2$ and $\bar \psi(\slashed{D}+m)\psi$ is clear. The $B^a$ term is also trivially invariant.

The transformation of $\bar c$ cancels the transformation of $A_\mu$ when combining the last two terms of $\cL$, so what remains is the covariant derivative term.
\begin{align*}
    \delta(D_\mu^{ac} c^c) &= \delta(\p_\mu c^a + gf^{abc} A_\mu^b c^c)\\
        &=\p_\mu (\delta c^a) + g f^{abc} (\delta A_\mu^b) c^c +g f^{abc} A_\mu^b \delta(c^c)\\
        &= -\frac{1}{2} g\epsilon \p_\mu (f^{abc} c^b c^c) +gf^{abc} (\epsilon D^{bd}_\mu c^d)c^c -\frac{1}{2}g^2 \epsilon f^{abc} A^b_\mu f^{cde} c^d c^e\\
        &= -\frac{1}{2} g\epsilon \p_\mu (f^{abc} c^b c^c) +gf^{abc} \epsilon (\p_\mu c^b) c^c + g^2 \epsilon f^{abc} f^{bde} A_\mu^d c^e c^c -\frac{1}{2} g^2 \epsilon f^{abc} f^{bde} A_\mu^b c^d c^e.
\end{align*}
Looking at the $O(g)$ terms, we have
\begin{equation}
     -\frac{f^{abc}}{2}\p_\mu (c^b c^c) +f^{abc}(\p_\mu c^b) c^c =-\frac{f^{abc}}{2} \bkt{(\p_\mu c^b) c^c +c^b(\p_\mu c^c)} +f^{abc}(\p_\mu c^b)c^c.
\end{equation}
Using the fact that the ghosts anticommute and $f^{abc}$ is totally antisymmetric,
\begin{equation}
    f^{abc} c^b(\p_\mu c^c) = -f^{abc} (\p_\mu c^c) c^b = + f^{abc}(\p_\mu c^b)c^c,
\end{equation}
the first two terms double up and we see that the whole $O(g)$ term is zero. Similar manipulations show that the $O(g^2)$ term is also zero, using the Jacobi identity:
\begin{equation}
    f^{ade}f^{bcd} + f^{bde} f^{cad} +f^{cde} f^{abd}=0.
\end{equation}
We conclude that the entire Lagrangian $\cL$ is BRST-invariant.