Last time, we introduced conformal field theory. We found that for our two-dimensional worldsheet, we can construct a map
\begin{equation*}
    z=e^{\tau +i\sigma},\quad \bar z=e^{\tau-i\sigma}.
\end{equation*}
We saw that the coordinates $z\to z'=f(z)$ were more generally holomorphic, i.e. we get some functions which satisfy the Cauchy-Riemann equations.

\subsection*{Conformal fields} Here are some definitions common in the literature for conformal field theory.
\begin{defn}
    A \term{chiral field} is a field $\Phi$ that depends on $z$ only, i.e. $\Phi=\Phi(z)$. Similarly, an \term{anti-chiral field} is a field that depends only on $\bar z$.
\end{defn}
\begin{defn}
    The \term{conformal dimension} refers to how a field transforms under scalings $z\to z'=\lambda z, \bar z \to \bar z'=\bar \lambda \bar z$ ($\lambda \in \CC$).
    \begin{equation}
        \Phi(z,\bar z)\to \Phi'(z',\bar z')=\lambda^h \bar \lambda^{\bar h}\Phi(\lambda z, \bar \lambda \bar z).
    \end{equation}
    We shall call $h$ and $\bar h$ the dimension of $\Phi(z,\bar z)$.%
        \footnote{This is a lot like what we did in \emph{Statistical Field Theory}. In looking at the RG flows of different fields and couplings, we saw that they scaled in different ways with some scaling dimension.}
    Sometimes $h+\bar h$ is referred to as the dimension and $h-\bar h$ as the ``conformal spin.''
\end{defn}
\begin{defn}
    Under the conformal transformation $z\to z'=f(z)$, a \term{primary field} with dimension $(h,\bar h)$ transforms as
    \begin{equation}
        \Phi(z,\bar z)\to \paren{\P{f}{z}}^h \paren{\P{\bar f}{\bar z}}^{\bar h}\Phi(f(z),\bar f(\bar z)).
    \end{equation}
    That is, a primary field transforms like a tensor (with the appropriate exponents of $h,\bar h$).
\end{defn}
\begin{exm}
    Consider an infinitesimal transformation
    \begin{equation}
        z\to z'=z+v(z)+\ldots = f(z).
    \end{equation}
    Thus
    \begin{gather}
        \paren{\P{f}{z}}^h=(1+\p v)^h\\ 
        \phi(f(z))=\phi(z)+v(z) \p \phi(z)+\ldots.
    \end{gather}
    So for a field with $(h,\bar h)=(h,0)$ we get
    \begin{equation}
        \delta \Phi(z)=(h\p v(z)+v(z)\p)\Phi(z)
    \end{equation}
    where we have taken only the term to leading order in $h$.
\end{exm}

\subsection*{Symmetries and the stress tensor}
For our classical theory, let us start with the action
\begin{equation}
    S[X]=-\frac{1}{4\pi \alpha'}\int_\Sigma d^2 z \p_a X^\mu \p^a X^\nu \eta_{\mu\nu}.
\end{equation}
Let us note that in going from $\tau,\sigma$ coordinates to $z,\bar z$, we pick up an $i$ as the Jacobian factor, meaning that $e^{iS}\to_{(z,\bar z)} e^{-S}$.

Consider the (conformal) transformation
\begin{equation}
    \delta_v X^\mu = v^a \p_a X^\mu.
\end{equation}
The variation of the action is now
\begin{equation}
    \delta_v S[X]=-\frac{1}{2\pi \alpha'} \int_\Sigma d^2 z\paren{ (\p_a v^b)\p_b X^\mu \p^a X_\mu + v^b \p_a (\p_b X^\mu)\p^a X_\mu}
\end{equation}
where all indices are raised and lowered with the Minkowski metric. After an integration by parts, this transformation becomes
\begin{equation}
    \delta_v S[X]=\frac{1}{2\pi} \int_\Sigma d^2z (\p^a v^b) T_{ab},
\end{equation}
with $T_{ab}$ our old buddy the stress tensor. This tells us that $\delta S[X]=0$ requires that
\begin{equation}
    \p^a T_{ab}=0,
\end{equation}
which is just Noether's theorem. That is, if the action is invariant under conformal transformations, then the stress tensor is conserved.

We could define a conserved charge
\begin{equation}
    Q = Q_+ + Q_-
\end{equation}
where
\begin{equation}
    Q_\pm =\frac{1}{2\pi} \int_0^{2\pi} d\sigma T_{\pm\pm}(\sigma)
\end{equation}
at $\tau=0.$ Classically, the symmetry transformations are generated by the charge $Q$:
\begin{equation}
    \delta X^\mu=\set{Q,X^\mu}_{PB}.
\end{equation}
What's the analogue of this in the quantum theory? Let's find out.
\subsection*{Conformal transformations and Ward identities}
For the following discussion, we stay in $d=2$ but rather than focusing on our embedding fields $X$, we will work with more general fields $\phi(z,\bar z)$. We shall be interested in the quantum analogue of Noether's theorem.

Consider a transformation
\begin{equation}
    \phi \to \phi'=\phi+\delta \phi, \quad S[\phi']=S[\phi]+\delta S[\phi].
\end{equation}
In the classical picture, we would say that if $\delta S=0$, we've got a symmetry and that gives us some conserved quantity. But a classical action doesn't always uniquely specify a quantum action, and conversely there are some quantum actions we don't know the classical versions of. However, what we can say is that a symmetry of a quantum theory should preserve important features of that theory, and in particular it must preserve \emph{correlation functions}.

Let us consider the correlation function
\begin{equation*}
    \avg{\phi_1(z_1)\ldots \phi_n(z_n)} \equiv \avg{\phi_1\ldots \phi_n}.
\end{equation*}
Here, the $\bar z_i$ dependence is implicit. Under a transformation, our correlation functions become
\begin{align*}
    \avg{\phi_1\ldots \phi_n}\to \avg{\phi_1'\ldots \phi_n'} &=\int \cD \phi' e^{-S[\phi']}\phi_1' \ldots \phi_n'\\
        &= \int \cD \phi e^{-S[\phi]}(1-\delta S[\phi]+\ldots)(\phi_1 +\delta\phi_1 +\ldots)\ldots (\phi_n+\delta\phi_n+\ldots)\\
        &= \avg{\phi_1 \ldots \phi_n}-\int \cD \phi e^{-S[\phi]}\delta S[\phi] \phi_1\ldots \phi_n + \sum_{k=1}^n \int \cD \phi e^{-S[\phi]}\phi_1 \ldots \delta \phi_k \ldots \phi_n.
\end{align*}
where we have assumed that $\cD\phi'=\cD \phi$, i.e. the transformations are such that the integration measure is unchanged. If we require that the new correlations are the same as the old, i.e. $\avg{\phi_1\ldots \phi_n} = \avg{\phi_1'\ldots \phi_n'}$, then
\begin{equation}
    \avg{\delta S[\phi]\phi_1\ldots \phi_n}=\sum_{k=1}^n \avg{\phi_1 \ldots \delta \phi_k \ldots \phi_n}.
\end{equation}
We would like to draw an analogue to the classical current, so we write $\delta S[\phi]$ as
\begin{equation}
    \delta S[\phi]=\frac{1}{2\pi i}\int_\Sigma d^2 z (\p_a v(z))j^a(z),
\end{equation}
where $v$ is the parameter of the transformation and $j$ is the classical Noether current. Remember, our aim here is to see how classical symmetries can be promoted to quantum conservation laws. Thus
\begin{equation}
    \frac{1}{2\pi i} \int_\Sigma d^2 z \p_a v(z)\avg{j^a(z)\phi_1 \ldots \phi_n}=\sum_{k=1}^n \avg{\phi_1 \ldots \delta \phi_k \ldots \phi_n}.
\end{equation}
We also choose $\Sigma$ and $v(z)$ to isolate a particular $\delta \phi_k$. We define $\omega =z_k$ (thus $\phi_k(z_k)=\phi(\omega)$ and two curves $C_1,C_2$ such that $\p \Sigma=C_1 \cup C_2$.%see diagram
We choose $v(z)$ to be constant within $C_1$, zero outside of $C_2$, and arbitrary on $\Sigma$. We also require $C_1,C_2$ to encircle $\omega =z_k$ only so that $v=0$ at all other points $z_{j\neq k}$, which implies that all the other $\delta \phi_j, j\neq k$ vanish. In this way, we have
\begin{equation}
    \frac{1}{2\pi i}\int_\Sigma d^2 z \p_a v(z) \avg{j^a (z) \phi_1 \ldots \phi_n} =\avg{\phi_1 \ldots \phi(\omega) \ldots \phi_n}.
\end{equation}