We've started our lightning tour of the theory of Riemann surfaces. Soon, we'll see the emergence of our first scattering amplitudes.

\subsection*{Conformal Killing vectors} Recall from \emph{General Relativity} that Killing vectors are very special objects which represent symmetries of the metric. In the language of Lie derivatives, a vector $K$ is a Killing vector if the Lie derivative of the metric with respect to $K$ is trivial, $\cL_K g =0$.%
    \footnote{In terms of covariant derivatives, $\nabla_a K_b + \nabla_b K_a = 0$.
    }
\term{Conformal Killing vectors} (CKV) generalize this idea. A conformal Killing vector generates diffeomorphisms that preserve the metric up to Weyl transformations.

Our gauge transformations are
\begin{gather}
    \delta_V h_{ab} = \nabla_a V_b +\nabla_b V_a\\
    \delta_\omega h_{ab} = 2\omega h_{ab}.
\end{gather}
We are interested in $V^a$ such that
\begin{equation}
    \delta_{CK} h_{ab}=\nabla_a V_b + \nabla_b V_a + 2\omega h_{ab}=0.
\end{equation}
Note the covariant derivatives are taken with respect to the metric $h_{ab}$. Taking the trace, we have equivalently
\begin{equation}
    2(\nabla_a V^a) +4\omega = 0 \implies \omega = -\frac{1}{2} (\nabla_a V^a),
\end{equation}
so $V^a$ is a conformal Killing vector if
\begin{equation}
    \delta h_{ab} = \nabla_a V_b + \nabla_b V_a - h_{ab}(\nabla_c V^c)=0.
\end{equation}
We define
\begin{equation}
    (Pv)_{ab}\equiv \nabla_a V_b+ \nabla_b V_a -h_{ab}(\nabla_c V^c)
\end{equation}
so that $V^a$ is a conformal Killing vector if $V^a\in \text{Ker}P$.

Why have we introduced these? For closed Riemann surfaces of genus $g$, the (real) dimension of the conformal Killing group (CKG), i.e. the subgroup of diffeomorphisms generated by the conformal Killing vectors, is known: it is
\begin{equation}
    \kappa =|\text{CKG}|=\begin{cases}
        6, & g=0\\
        2, & g=1\\
        0, & g \geq 2.
    \end{cases}
\end{equation}
On the sphere (think of this as $\CC$ with the point at $\infty$), the CKVs generate the transformations
\begin{equation}
    z \to \frac{az+b}{cz+d}
\end{equation}
and similarly for $\bar z$, where $a,b,c,d\in \CC$ and $ad-bc=1$. This is in fact the \href{https://en.wikipedia.org/wiki/M\%C3\%B6bius_transformation}{M\"obius group} from complex analysis. We have four parameters and one algebraic constraint on complex values (hence two real constraints). Therefore we shall fix the conformal Killing symmetry by requiring that the $V^a$ vanish at three distinct points on $\Sigma$ (i.e. imposing six real constraints, since each point on $\Sigma$ comes with two coordinates).

We'll need one more mathematical preliminary before moving forward. This is the \term{modular group.} First, observe that the diffeomorphism group on the Riemann surface $\Sigma_g$ is in general not connected. Let us therefore define something useful-- call the connected set of diffeomorphisms that includes the identity $\text{Diff}_0$. The modular group $\cM_g$ is then
\begin{equation}
    \cM_g =\frac{\text{Diff}}{\text{Diff}_0}.
\end{equation}
For example, for the torus we have $\cM_1=\text{SL}(2:\ZZ)$.

Then the moduli space $M_g$ can be written schematically as
\begin{equation}
    M_g = \frac{\set{\text{metrics}}}{\set{\text{Diff}}\times \set{\text{Weyl}}}=\frac{\set{\text{metrics}}}{\set{\text{Diff}_0}\times \set{\text{Weyl}}}/\cM_g.
\end{equation}
We often call the space
\begin{equation}
    \mathcal{T}_g= \frac{\set{\text{metrics}}}{\set{\text{Diff}_0}\times \set{\text{Weyl}}}
\end{equation}
the Teichm\"uller space. In this notation, $M_g= \mathcal{T}_g/\cM_g$.

\subsection*{The Faddeev-Popov determinant} When we do path integrals, it's usually desirable to check our answer by other means, since path integrals have a way of hiding divergences which we as self-respecting physicists ought to care about. Happily, this will be possible for the following quantity we are about to define.

The idea is to choose a ``gauge slice'' through the space of metrics on $\Sigma_g$. That is, we choose a gauge such that the metric on the worldsheet $h_{ab}$ takes some nice form, $h_{ab}=\hat h_{ab}$ (often diagonal), such that $\text{Diff}_0\times\text{Weyl}$ orbits then take us everywhere else in our space of metrics. We formally define the \term{Fadeev-Popov determinant} as
\begin{equation}
    1=\Delta_{FG}(\hat h)\int_{\text{Diff}_0 \times \text{Weyl}} \cD(\delta h) \delta[h-\hat h] \prod_i \delta(v(\hat \sigma_i)),
\end{equation}
where $\delta[h-\hat h]$ can be thought of as a ``delta functional'' and $\sigma_i$ indicates points on our worldsheet $\Sigma_g$ where the CKVs vanish (in order to fix the CKG). We can think of this determinant in analogy to how $\delta(f(x))\sim \frac{\delta x}{|f'(x_i)|}$ where $f(x_i)=0.$

In more detail, we may write
\begin{equation}
    1=\Delta_{FG}(\hat h) \int_{\mathcal{T}} d^s t \int \cD \omega \cD v \delta[h_{ab}-\hat h_{ab}] \prod_i \delta(v(\hat \sigma_i)),
\end{equation}
where the $d^s t$ integral is taken in Teichm\"uller space and our path integral is now written explicitly over the space of variations of $h$.

We will now write the delta functions and delta functions as integrals and functional integrals. Let us introduce numbers $\zeta^i_a$ and fields $\beta^{ab}(\sigma,\tau)$ such that
\begin{equation}
    1=\Delta_{FG}(\hat h) \int_{\mathcal{T}} d^s t \int \cD \omega \cD v \paren{d^\kappa \zeta^i_a \cD \beta \exp(i(\beta|h-\hat h)+i\zeta^i_a v^a (\hat \sigma_i)},
\end{equation}
where the inner product $(\beta|h-\hat h)$ is defined to be
\begin{equation}
    (\beta|h-\hat h) = \int_\Sigma d^2 \sigma \sqrt{|h|}\beta^{ab}(h_{ab}-\hat h_{ab})
\end{equation}

We can write $h_{ab}-\hat h_{ab}=\delta_{ab}$ as
\begin{align*}
    \delta h_{ab}&=\underbrace{\nabla_a v_b + \nabla_b v_a}_{\text{Diffeos}} +\underbrace{2\omega h_{ab}}_{\text{Weyl}}+\underbrace{t^I \p_I h_{ab}}_{\text{moduli}}\\
        &= (Pv)_{ab} +2(\omega+\nabla_c v^c)h_{ab} +t^I \p_I h_{ab}\\
        &= (Pv)_{ab}+2\bar \omega h_{ab} +t^I \mu_{Iab}
\end{align*}
where $(Pv)_{ab}$ is as defined before, $\mu_{Iab}=\p_Ih_{ab}-\text{trace}$, and $\bar \omega$ contains the residual trace terms.