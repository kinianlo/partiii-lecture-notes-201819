The official course notes and the first example sheet are online now. Note that David Tong's notes may also supplement the notes for this course. In addition, note that problems 4 and 5 are eligible for marking, while problem 6 has a typo and therefore the instructor asks that we ignore problem 6 entirely.

Last time, we introduced the Faddeev-Popov determinant. We found that
\begin{equation}
    \Delta_{FP}^{-1}(\hat h) = \int_{\mathcal{T}} d^s t \int \cD \omega \cD v \paren{d^\kappa \zeta^i_a \cD \beta \exp(i(\beta|Pv+2\bar \omega h + t^I \mu_I))+i\sum_{i=1}^k \zeta^i_a v^a (\hat \sigma_i)}
\end{equation}

\subsection*{Grassmann quantities} If you're keeping up with my \emph{AQFT} and \emph{Supersymmetry} notes, this will be your third time seeing Grassmann quantities/variables. These are a set of quantities $\theta$ such that any two of them anticommute,
\begin{equation*}
    \theta_1 \theta_2 = -\theta_2 \theta_1.
\end{equation*}
Equivalently their anticommutator vanishes,
\begin{equation*}
    \set{\theta_1,\theta_2}=0.
\end{equation*}
Objects (such as wavefunctions) that obey Fermi statistics can naturally be described by Grassmann numbers. One bit of motivation for this is the fact that for any $\theta,$ we have $\theta^2=-\theta^2=0$, which is reminiscent of the Pauli exclusion principle. This anticommuting property also holds for integration measures,
\begin{equation*}
    d\theta_1d\theta_2 = -d\theta_2 d\theta_1.
\end{equation*}

One can show (e.g. by considering $(\int d\theta)^2$) that
\begin{equation*}
    \int d\theta=0,
\end{equation*}
and we can consistently define
\begin{equation*}
    \int d\theta\, \theta = 1.
\end{equation*}
The Dirac delta function for Grassman quantities is then $\delta(\theta)=\theta$, which leads to the somewhat unusual conclusion that integration and differentiation of Grassmann variables are essentially the same process.

Note that Taylor expansions are very easy for Grassman variables, since we cannot have anything of higher degree than 1 because $\theta^2=0$. Thus we can write some function $f(x,\theta)$ as
\begin{equation*}
    f(x,\theta)=f_0(x)+\theta f_1(x),
\end{equation*}
and so an integral can be written
\begin{equation}
    \int d\theta\, f(x,\theta)=f_1(x)=\P{f(x,\theta)}{\theta}.
\end{equation}
\begin{exm}
    Let $\theta^a= \begin{pmatrix}\theta_1 \\ \theta_2 \end{pmatrix}$ and $\bar \theta^a =(\bar \theta_1,\bar \theta_2)$. Consider the integral
    \begin{equation}
        \int d^2 \theta d^2 \bar \theta \exp \paren{-\bar \theta^a M_{ab} \theta^b},
    \end{equation}
    where $M_{ab}$ is some normal $2\times 2$ matrix. This exponential has a few terms but not too many. The first term is just $1$, while the last term has $4$ thetas and two $M$s. Recalling that integration is like differentiation, we write the integral as
    \begin{equation}
        \frac{\p^4}{\p \theta_1 \p\theta_2 \p \bar \theta_1 \p \bar \theta_2} \set{(\bar \theta_1 M_{11} \theta_1)(\bar \theta_2 M_{22}\theta_2)+(\bar \theta_1 M_{12}\theta_2)(\bar \theta_2 M_{21} \theta_1},
    \end{equation}
    noting that the only nonzero term must have all four of $\theta_1,\theta_2,$ and their barred versions.
    
    Equivalently this integral is
    \begin{equation}
        \int d^2 \theta d^2 \bar \theta \exp \paren{-\bar \theta^a M_{ab} \theta^b} =(M_{11}M_{22}-M_{12}M_{21})=\det (M_{ab}).
    \end{equation}
    
    This result generalizes-- the equivalent of a Gaussian for Grassmann variables is
    \begin{equation}
        \int d^n\theta d^n\bar\theta \exp(-\bar \theta^a M_{ab} \theta^b)=\det(M_{ab}).
    \end{equation}
    Note that this is a bit different from the result for $z,\bar z$ real, where
    \begin{equation}
        \int d^2 z d^2 \bar z \exp(-\bar z Mz) =\frac{1}{\det (M_{ab})}.
    \end{equation}
\end{exm}

This effect of inverting the determinant when we replace commuting (bosonic) variables with Grassmann (fermionic) variables carries over to the functional case, which we will just state but not prove.

With our ``new'' Grassmann variables in hand, we will now rewrite the Faddeev-Popov determinant in terms of Grassmann quantities to perform these crazy path integrals. That is, promote
\begin{equation}
    v^a \to c^a, \beta^{ab}\to b^{ab}, t^I \to \xi^I, \zeta^i_a \to \eta^i_a,
\end{equation}
where $c^a$ and $b^{ab}=b^{ba}$ are Grassmann fields on $\Sigma$.

Note also that we can apparently get rid of the $\cD\omega$ integral by writing
\begin{align*}
    \Delta^{-1}_{FP}&\sim \int \cD \omega \exp\bkt{ i(\beta|2\bar \omega h)}\\
    &\sim \int \cD \bar \omega \exp\bkt{ i \int_\Sigma d^2 \sigma \sqrt{h}\beta^{ab} 2\bar \omega h_{ab}}.
\end{align*}
But we can do this $\bar \omega$ integral-- it looks like a delta function, and fixes $\beta^{ab}h_{ab}=0$. Thus $\beta^{ab}$ is traceless.

Thus we rewrite the Fadeev-Popov determinant in terms of our shiny new Grassmann variables as
\begin{equation}
    \Delta_{FP}(\hat h)=\int d^s \xi \int \cD c \cD b d^k\eta \exp(i(b|Pc + \xi^I \mu_I)+i\sum_{i=1}^k \eta_a^i c^a(\hat \sigma_i)),
\end{equation}
having done the $\omega$ integral as above. Note that this is really just the Faddeev-Popov determinant and not its inverse, since we have promoted everything to Grassmann variables. We can also do the $\eta_a^i$ and $\xi^I$ integrals to get
\begin{align*}
    \Delta_{FP} (\hat h) &= \int \cD c \cD b\, e^{i(b|Pc)} \prod_{I=1}^S \delta [(b|\mu_I)] \prod_{i=1}^K \delta(c^a(\hat \sigma_i))\\
        &= \int \cD c \cD b \, e^{i(b|Pc)} \prod (b|\mu_I) \prod_{i=1}^k c^a (\hat \sigma_i).
\end{align*}
After all this computation, we therefore have
\begin{equation}
    \Delta_{FP}(\hat h) = \int \cD c \cD b \, e^{iS[b,c]} \prod (b|\mu_I) \prod_{i=1,a=1,2}^k c^a (\hat \sigma_i)
\end{equation}
where we have something that looks like an action,
\begin{equation}
    S[b,c]=\int_\Sigma d^2\sigma \,\sqrt{h} b^{ab}(Pc)_{ab}=2\int_\Sigma d^2\sigma\,\sqrt{h} b^{ab} (\nabla_a c_b).
\end{equation}
Note that $c^a,b_{ab}$ are Grassmann fields and therefore obey Fermi statistics. However, it turns out they also have integer ``spin'' (for some notion of spin we have not defined precisely yet). Fortunately, this is allowed because these quantities are not observables. We should think of them a bit like constraints on the observable variables of our theory, and we call them \term{Faddeev-Popov ghosts}.