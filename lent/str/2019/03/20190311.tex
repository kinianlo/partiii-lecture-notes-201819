Today we'll continue our discussion of the scattering of massless states. Recall that we had a trick of writing vertex operators as ``source terms'' in the path integral, which we could differentiate and then set sources to zero in, giving us the desired scattering amplitude.

Recall that we could write the vertex operator
\begin{equation}
    i\p X^\mu(z_j) e^{ik_j \cdot X(z_j)} =
        \P{}{\rho_{\mu j}}\set*{\exp\bkt{
        i\int_\Sigma d^2 z (k_{\mu j} +\rho_{\mu j} \P{}{z}) X^\mu(z,\bar z) \delta^2(z-z_j)
    }}_{\rho_j=0}.
\end{equation}
It is useful to split $X$ into a center-of-mass part and its holomorphic and antiholomorphic parts,
\begin{equation*}
    X^\mu(z,\bar z) = x^\mu +\tilde X^\mu(z) + \bar X^\mu(\bar z),
\end{equation*}
and similarly we split
\begin{equation}
    J^\mu(z,\bar z) =j^\mu (z) +\bar j^\mu (\bar z).
\end{equation}
The Green's function also splits:
\begin{equation}
    G(z,\omega)=-\frac{\alpha'}{2} \ln|z-\omega|^2 = -\frac{\alpha'}{2} \ln(z-\omega) -\frac{\alpha'}{2} \ln(\bar z-\bar \omega).
\end{equation}
Explicitly, our source terms splits as
\begin{equation}
    j_\mu(z) = i\sum_{j=1}^n \delta^2( z-z_j) \paren{ \frac{1}{2} k_{j\mu} + \rho_{j\mu} \P{}{z}}.
\end{equation}
As usual, our theory comes with two basically decoupled sectors. The amplitude may be written as
\begin{align*}
    A_n ={}& g_c^{n-2} |z_1-z_2|^2 |z_2-z_3|^2 |z_3-z_1|^2 
        \delta^{26} \paren{ \sum_{j=1}^{n} k_{\mu j}} 
        \int d^2 z_4 \ldots d^2 z_n \, \epsilon_{\mu_1 \nu_1}^{(1)} \ldots \epsilon_{\mu_n \nu_n}^{(n)}\\
        &\times \avg{\prod_{j=1}^{n} \p \tilde X^{\mu_j} (z_j) e^{ik_j \cdot \tilde X(z_j)}} \avg{\prod_{j=1}^n \bar \p \bar X^{\mu_j} (\bar z_j) e^{ik_j \cdot \bar X(\bar z_j)}}
\end{align*}
where
\begin{equation}
    \avg{\prod_{j=1}^n \p \tilde X^{\mu_j} (z_j) e^{ik_j \cdot X(z_j)}}= \frac{1}{i^n} \frac{\p^n}{\p \rho_{1\mu_1} \ldots \p \rho_{n\mu_n}} W[j]|_{\rho_1 = \rho_2 =\ldots =0}
\end{equation}
and
\begin{equation}
    W[j]= \exp \paren{ \frac{1}{2} \int_{\Sigma \times \Sigma} d^2 z d^2\omega\, j(z) j(\omega) G(z,\omega)}
\end{equation}
is an action with sources. We can think of $W$ as a generating functional, just like in QFT. Here, $G(z,\omega)=-\frac{\alpha'}{2} \ln(z-\omega)$. If we do the integral, we find that
\begin{equation}
    W[j]= \prod_{i<j} |z_i -z_j|^{\alpha' k_i \cdot k_j /2} \times \exp \paren{\frac{\alpha'}{2} \sum_{i<j} \frac{\rho_i \cdot \rho_j}{(z_i-z_j)^2}+ \frac{\alpha'}{2} \sum_{i\neq j} \frac{k_i \cdot \rho_j}{z_i-z_j}}.
\end{equation}
Actually taking the derivatives is a bit of a pain for many-particle scattering amplitudes, but in principle it is tractable. We can also compute amplitudes using a Wick contraction method-- it's a matter of taste which way is preferable.

\begin{exm}
    Let's consider three-point graviton scattering. Here, the polarization vectors $\epsilon_{\mu\nu}$ are symmetric and traceless to represent gravitons. Recall that three-point interactions are nice since momentum conservation and the ghosts simplify our problem a bit. For three gravitons,
    \begin{equation}
        \alpha' k_1 \cdot k_2 = \frac{\alpha'}{2}(k_1+k_2)^2 = \frac{\alpha'}{2} k_3^2=0.
    \end{equation}
     We therefore have
    \begin{equation}
        |z_1-z_j|^{\alpha' k_1 \cdot}=1,
    \end{equation}
    and so
    \begin{equation}
        \avg{\prod_{j=1}^3 \p \tilde X^{\mu_j} (z-J) e^{ik_j \cdot X(z_j)}} =\paren{\frac{\alpha'}{2}}^2 \frac{T^{\mu_1 \mu_2 \mu_3}}{(z_1-z_2)(z_2-z_3)(z_3-z_1)}
    \end{equation}
    with
    \begin{equation}
         T^{\mu_1 \mu_2 \mu_3} =\eta^{\mu_1 \mu_2} k_2^{\mu_3} + \eta^{\mu_2 \mu_3} k_3^{\mu_1} +\eta^{\mu_3 \mu_1} k_1^{\mu_2} + \frac{\alpha'}{2} k_3^{\mu_1} k_1^{\mu_2} k_2^{\mu_2},
    \end{equation}
    where we used identities like $k^{\mu_1} \epsilon_{\mu_1 \nu_1}^{(1)}=0$ to simplify. Thus the three-point graviton scattering amplitude comes out to
    \begin{equation}
        A_3 = g_c (2\pi)^{26} \delta^{26} \paren{\sum_{j=1}^3 k_{\mu j}} \epsilon_{\mu_1 \nu_1} \epsilon_{\mu_2 \nu_2} \epsilon_{\mu_3 \nu_3} T^{\mu_1 \mu_2 \mu_3} T^{\nu_1 \nu_2 \nu_3}.
    \end{equation}
    One can show that this agrees with (tree-level) perturbation theory in general relativity, where there's some identification between $g_c$ and the gravitational constant $G_N$ in 26 dimensions. However, there might also be interactions at tree level in string theory, higher derivative corrections to the Einstein equations.
\end{exm}

\subsection*{One loop}
This is non-examinable material (no exam questions on loop corrections) but it's certainly the next natural step after discussing tree-level amplitudes. We integrate over the moduli space $\cM_1$, i.e. the moduli space of the torus (genus 1 Riemann surfaces). Thus
\begin{equation}
    \cM_1 =\set{\tau = \tau_1 +i \tau_2| \tau_2 >0; -\frac{1}{2} \leq \tau_1 \leq \frac{1}{2}, |\tau| \geq 1}
\end{equation}
where $z\sim z + \tau$. At one loop, the amplitude is given by
\begin{equation}
    A_n = \frac{1}{|U(1) \times U(1)|} \int_{\cM_1} \frac{d^2 \tau}{\tau_2} \int \cD b \cD \bar b \cD c \cD \bar c (\mu_\tau|b) (\bar \mu_\tau |\bar b) e^{-S_\text{gh}[b,c]} \avg{V_1 \ldots V_n}_X c(z_1) \bar c(\bar z_1).
\end{equation}
That is, this prefactor is equivalent to dividing out by the surface area of the torus. The weird inner products are then
\begin{equation}
    (\mu_\tau | b) =\frac{1}{2\pi} \int_\Sigma d^2 z\paren{\P{h_{\bar z \bar z}}{\tau}} b_{zz}.
\end{equation}
We can compute this explicitly. Consider the worldsheet metric
\begin{equation}
    h_{ab} =\begin{pmatrix}
        0 & \frac{1}{2}\\
        \frac{1}{2} & 0
    \end{pmatrix}
\end{equation}
under the deformation $h_{\bar z \bar z}=0\to \epsilon$. We find that
\begin{equation}
    ds^2 = dz d \bar z  \to (1+\epsilon +\bar \epsilon) dz' d\bar z' +O(\epsilon^2),
\end{equation}
with $z'=z+\epsilon(\bar z-\bar z)+O(\epsilon^2)$.
Looking at $z'\sim z'+\tau',$ we see that 
\begin{equation}
    \tau'=\tau+\epsilon(\bar \tau_\tau)+O(\epsilon^2).
\end{equation}
We know $\P{h_{\bar z \bar z}}{\epsilon}$ and $\P{\tau}{\epsilon}$ and we have
\begin{equation}
    \p_\tau h_{\bar z \bar z} =\frac{i}{2\tau_2}.
\end{equation}
Compared with tree level, we have b-ghost insertions in the ghost path integral.

However, observe that the Green's function also changes now that the topology of our Riemann surface has changed. $G(z,\omega)$ is more complicated due to periodicity requirements. Such requirements will lead us to introduce ``theta functions'' from the study of partial differential equations on manifolds of interesting topology.

We may ask whether our theory has divergences after we introduce loops. It turns out that it does, but these aren't UV divergences-- they are associated to a tachyon, which is not present in the superstring theory. It seems that this theory is perturbatively finite to all orders.