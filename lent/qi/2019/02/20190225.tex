\subsection*{Distance measures}
Given two states $\rho,\sigma$, how well can we distinguish between them? We have a few measures to test this.

\subsection*{Trace distance}
We define the \term{trace distance} $D(\rho,\sigma)$ between two density matrices $\rho,\sigma$ as follows:
\begin{gather}
    D(\rho,\sigma)=\frac{1}{2}||\rho-\sigma||_1\\
    ||A||_1 = \Tr\abs{A}=\Tr\sqrt{A^\dagger A}
\end{gather}
Thus define $A:=\rho-\sigma$. We take $A$ to be self-adjoint, $A^\dagger =A$. Then we can decompose $A$ into its eigenvalues,
\begin{align*}
    A&=\sum a_i \dyad{\phi_i}\\
        &=\sum_{a_i\geq 0} a_i \phi_i + \sum_{a_i <0} a_i \phi_i\\
        &= Q-R
\end{align*}
where $Q,R$ are now positive semi-definite. Thus by the linearity of the trace,
\begin{align*}
    D(\rho,\sigma)&= \frac{1}{2}\Tr\abs{A}\\
        &= \frac{1}{2} \bkt{\sum_{a_i \geq 0} -\sum_{a_i \geq 0} a_i -\sum_{a_i <0} a_i}\\
        &=\frac{1}{2}(\Tr Q + \Tr R),.
\end{align*}
However, note that $A=\rho-\sigma =Q-R$. Since $A$ is traceless,%
    \footnote{This follows since $\rho$ and $\sigma$ are density matrices and have trace 1, so by the linearity of the trace, $\Tr(A)=\Tr(\rho)-\Tr(\sigma)=1-1=0.$}
it follows that $\Tr Q = \Tr R$, which implies that
\begin{equation}
    D(\rho,\sigma)=\Tr Q =\Tr R.
\end{equation}

\begin{lem}\label{tracedistanceproj}
\begin{equation}
    D(\rho,\sigma)=\max \Tr(P(\rho-\sigma)), \quad 0 \leq P \leq I.
\end{equation}
\end{lem}
Here, we use $P\leq I$ to indicate that $X\equiv (I-P)\geq 0 $ is positive semi-definite as an operator.
\begin{proof}
Since $D(\rho,\sigma)=\Tr Q$, let $P$ be the projector onto the support of $Q$. Then
\begin{align*}
    \Tr(P(\rho-\sigma))&=\Tr(P(Q-R))\\
        &=\Tr PQ - \Tr PR\\
        &= \Tr Q\\
        &= D(\rho,\sigma),
\end{align*}
since the supports of $Q$ and $R$ are orthogonal.
Conversely, $\forall 0\leq P \leq I$, we have
\begin{align*}
    \Tr P(\rho-\sigma) &= \Tr(P(Q-R))\\
        &\leq \Tr PQ\\
        &\leq \Tr Q\\
        &= D(\rho,\sigma)
\end{align*}
since $P\leq I$.%
    \footnote{The second line follows since $R$ is positive semi-definite, and $PR$ is also positive semi-definite. The third line follows because $\Tr PQ = \Tr P Q^{1/2} Q^{1/2}=\Tr Q^{1/2} P Q^{1/2} \leq \Tr Q.$}
Combining these, we see that
\begin{equation*}
    D(\rho,\sigma) = \max_{0\leq P \leq I} \Tr P(\rho-\sigma).
\end{equation*}
\end{proof}
This trace distance has some nice properties. It forms a metric on $\cD(\cH)$, as it is
\begin{itemize}
    \item Symmetric, $D(\rho,\sigma)=D(\sigma,\rho)$%
        \footnote{
            Follows since $D(\rho,\sigma)=\frac{1}{2}||A||_1$, and $||A||_1=\Tr\sqrt{A^\dagger A}=\Tr\sqrt{(-A)^\dagger (-A)}=||-A||_1$, as it should be. Thus $D(\rho,\sigma)=\frac{1}{2}||A||_1
            =\frac{1}{2}||-A||_1=D(\sigma,\rho).$
        }
    \item $D(\rho,\sigma)=0$ iff $\rho=\sigma$
    \item Triangle inequality, $D(\rho,\omega) \leq D9\rho,\sigma)+D(\sigma,\omega)$.
\end{itemize}
The second property follows from $D(\rho,\sigma)=\Tr Q =\Tr R=0\implies a_i=0\forall i \implies A=0 \implies \rho=\sigma$. The final follows from using Lemma \ref{tracedistanceproj} and noting that
\begin{align*}
    D(\rho,\sigma) &= \Tr(P(\rho-\sigma))\\
        &= \Tr P(\rho-\omega)+\Tr P(\omega-\sigma)\\
        &\leq D(\rho,\omega) +D(\omega,\sigma).
\end{align*}

\begin{lem}
    Monotonicity under quantum operations $\Lambda$:
    \begin{equation}
        D(\Lambda(\rho),\Lambda(\sigma))\leq D(\rho,\sigma).
    \end{equation}
\end{lem}
\begin{proof}
    We know that
    \begin{equation*}
        D(\Lambda(\rho),\Lambda(\sigma))=\Tr (P(\lambda(\rho)-\Lambda(\sigma)),
    \end{equation*}
    where equality is reached for some $P$.
    We saw that $\Tr Q = \Tr R\implies \Tr (\Lambda(Q)) =\Tr (\Lambda(R))$ since $\Lambda$ is CPTP. Now
    \begin{align*}
        D(\rho,\sigma) = \Tr Q = \Tr \lambda(Q) &\geq \Tr(P \Lambda (Q))\\
        &\geq \Tr (P(\Lambda(Q)-\Lambda(R)))\\
        &= D(\Lambda(\rho, \Lambda(\sigma)),
    \end{align*}
    using the fact that $0\leq P \leq I.$
\end{proof}
\subsection*{Fidelity}
Let us define
\begin{align}
    F(\rho,\sigma) &= \Tr \sqrt{\sqrt{\rho}\sigma \sqrt{\rho}}\\
        &= ||\sqrt{\sigma}\sqrt{\rho}||_1.
\end{align}
This quantity $F(\rho,\sigma),$ called the fidelity, has some nice properties.
\begin{itemize}
    \item $F(\rho,\sigma)=F(\sigma,\rho)$.
    \item $0\leq F(\rho,\sigma)\leq 1$ with equality on the right if $\rho=\sigma$.
\end{itemize}
\textit{Case 1}: Note that if $[\rho,\sigma]=0$, then $\rho$ and $\sigma$ admit a simultaneous eigenbasis,
\begin{equation*}
    \rho=\sum_i \lambda_i \dyad{e_i},\quad \sigma=\sum_i \mu_i \dyad{e_i}.
\end{equation*}
Thus since $\sqrt{\dyad{\psi}}=\dyad{\psi},$
\begin{align}
    F(\rho,\sigma)&=\Tr (\sum_i \sqrt{\lambda_i \mu_i} \dyad{e_i}\\
    &=\sum_i \sqrt{\lambda_i \mu_i} \equiv F_{\text{cl}}(\lambda,\mu),
\end{align}
where $F_{\text{cl}}$ is now the classical fidelity and we've moved the trace inside the sum by linearity.

\textit{Case 2}: $\rho=\dyad{\phi},\sigma$. Exercise: show that
\begin{equation}
    F(\dyad{\phi},\sigma)=\sqrt{\bra{\phi}\sigma \ket{\phi}}.
\end{equation}
\textit{Case 3}: $\rho=\dyad{\phi},\sigma = \dyad{\psi}$.
\begin{equation}
    F(\rho,\sigma)=\sqrt{\abs{\braket{\phi}{\psi}}^2} =\abs{\braket{\phi}{\psi}}.
\end{equation}
For instance, this tells us that $F(U\rho U^\dagger, U\sigma U^\dagger)=F(\rho,\sigma)$.

\begin{thm}[Uhlmann]
    Let $\rho_A, \sigma_A \in \cD(\cH_A)$. Then the fidelity $F(\rho_A,\sigma_A)$ is
    \begin{equation}
        F(\rho_A,\sigma_A) =\max_{\ket{\psi_\rho^{AB}},\ket{\psi_\sigma^{AB}}} \abs{\braket{\psi_\rho^{AB}}{\psi_\sigma^{AB}}}
    \end{equation}
    where these $\ket{\psi_{\rho/\sigma}^{AB}}$ are purifications of the original density matrices.
    \begin{proof}
        We will need the following lemma:
        \begin{lem}
            $||A||_1 = \max_{U\text{ unitary}} \abs{\Tr(UA)} \quad\forall A \in \cB(\cH)$.
        \end{lem}
        \begin{proof}
            We use the polar decomposition, so $\exists V$ a unitary such that $A=|A|V$. If we now choose $U=V^\dagger$, then the inequality will be saturated:
            \begin{equation*} 
                \abs{\Tr(UA)}=\Tr(V^\dagger |A|V)=\Tr(|A|).
            \end{equation*}
            Hence $\forall U$ unitary,
            \begin{align*}
                |\Tr(AU)| &= |\Tr(|A|V U)|\\
                    &\leq \sqrt{(\Tr|A|)\Tr(U^\dagger V^\dagger |A| VU)}\\
                    &=\Tr|A|\equiv ||A||_1
            \end{align*}
            where we recognize this trace in the first line as a Hilbert-Schmidt inner product $(x,y)_{HS}$ with $x=|A|^{1/2},y=|A|^{1/2}VU$ and apply Cauchy-Schwartz. In the last line, we use the cyclic property of the trace to get $\sqrt{(\Tr|A|)^2}=\Tr|A|.$
            
            This proves the lemma.
        \end{proof}
        Now we can use the lemma to prove the theorem. We want to prove that 
        \begin{equation*}
            F(\rho_A,\sigma_A)=\max_{\ket{\psi_\rho^{RA}},\ket{\psi_\sigma^{RA}}} \abs{\braket{\psi_\rho^{RA}}{\psi_\sigma^{RA}}},
        \end{equation*}
        where $RA$ indicates the purified system.
        
        Recall that there is a unitary freedom of purifications-- any two purifications of the same $\rho$ are related by a unitary transformation acting only on the reference system. Equivalently, we can maximize over unitaries. If we fix $\ket{\psi_\sigma}_{RA}$, then
        \begin{equation}
            F(\rho_A,\sigma_A)=\max_{U^\rho, U^\sigma} \abs{\bra{\psi_\rho} (U_\rho^R{}^\dagger \otimes I_A) (U_\sigma^R{}^\dagger \otimes I_A)\ket{\psi_\sigma}.
            }
        \end{equation}
        Since the set of unitaries is closed, take $U\equiv U_\rho^R{}^\dagger U_\sigma^R$, and we maximize over $U$:
        \begin{equation*}
            \max_U \bra{\psi_\rho}U\otimes I \ket{\psi_\sigma}.
        \end{equation*}
        Now we know that we can generically write purified states as
        \begin{equation}
            \ket{\psi_\sigma^{RA}}=\sqrt{d}(I_R \otimes \sqrt{\sigma_A} \ket{\Omega},
        \end{equation}
        and so
        \begin{align*}
            \max_U \bra{\psi_\rho}U\otimes I \ket{\psi_\sigma} &= d\max_U \abs{\bra{\Omega}(I_R \otimes \sqrt{\rho_A})(U\otimes I)(I_R \otimes \sqrt{\sigma_A})\ket{\Omega}
            }\\
                &= d\max_U \abs{\bra{\Omega}(I_R \otimes \sqrt{\rho_A}\sqrt{\sigma_A})(U\otimes I)\ket{\Omega}}\\
                &=d\max_U \abs{\bra{\Omega}(I_R \otimes \sqrt{\rho_A}\sqrt{\sigma_A}U^T\ket{\Omega}}.
        \end{align*}
        Hence this becomes
        \begin{align*}
            \max_U \bra{\psi_\rho}U\otimes I \ket{\psi_\sigma}&=\max_U \abs{\sum_i \bra{i} \sqrt{\rho_A}\sqrt{\sigma_A} U^T \ket{i}}\\
            &= \max_U \abs{\Tr(\sqrt{\rho_A}\sqrt{\sigma_A} U^T)}\\
            &=\max_V \abs{\Tr (\sqrt{\rho_A}\sqrt{\sigma_A} V} = ||\sqrt{\rho_A}\sqrt{\sigma_A}||_1 =F(\rho,\sigma).
        \end{align*}
    \end{proof}
\end{thm}