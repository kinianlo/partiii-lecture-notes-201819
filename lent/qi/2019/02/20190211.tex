Admin note: there was no lecture (and hence no notes) for Friday, February 8, as Prof. Datta sustained some injury which prevented her from giving the lecture.

\subsection*{Quantum operations and CPTP maps}
To recap from last time, any allowed physical process on a quantum system is given by a quantum operation. The map must be completely positive (CP) in order to allow us to properly couple an ancilla (environment) to our system, and it must be linear and trace-preserving in order to take density matrices to other density matrices.

Consider a map $\Lambda:B(\cH)\to B(\mathcal{K}),$ where $\cH\simeq \CC^m,\mathcal{K}\simeq \CC^n$. Let $\cM_m, \cM_m^+$ bu $m\times m$ complex positive semi-definite matrices. The set of density matrices on $\CC^n$ is given by
\begin{equation}
    \cD(\CC^m)=\set{\rho \in \cM_m^+; \Tr \rho = 1}.
\end{equation}

\begin{defn}
    A map $\Lambda:\cM_m\to \cM_n$ is positive if
    \begin{equation}
        \Lambda(A)\in \cM_n^+ \forall A \in \cM_m^+.
    \end{equation}
\end{defn}
\begin{defn}
    For a given positive integer $k$, $\Lambda$ is $k$-positive if
    $(\Lambda \otimes \text{id}_k)$ is positive, where $\text{id}_k$ is the identity (super)operator, $\text{id}_k:\cM_k \to \cM_k$ such that $\text{id}_k(Q)=Q \forall Q\in \cM_k$.
\end{defn}
\begin{defn}
    The map $\Lambda$ is completely positive (CP) if it is $k$-positive $\forall k\in \ZZ^+$, positive integers.
\end{defn}
\begin{thm}[Necessary and sufficient condition for CP]
    A linear map $\Lambda:\cB(\CC^d)\to \cB(\CC^{d'})$ is completly positive $\iff (\Lambda \otimes \text{id}_d)(\kb{\Omega}{\Omega})\geq 0$, where
    \begin{equation}
        \ket{\Omega}=\frac{1}{\sqrt{d}}\sum_{i=1}^d \ket{i}\ket{i} \in \CC^d \otimes \CC^d.
    \end{equation}
\end{thm}
That is, it suffices to check positivity on the density matrix corresponding to the maximally entangled $d$-dimensional state.
\begin{proof}
    Necessity follows immediately from the definition of CP. To show sufficiency, consider an arbitrary $k\geq 1$. For a state $\rho \in \cD(\CC^d \otimes s\CC^k$, we have a spectral decomposition
    \begin{equation}
        \rho =\sum \lambda_i \kb{\phi_i}{\phi_i}
    \end{equation}
    where $\ket{\phi_i}\in \CC^d \otimes \CC^k.$
    Now we have
    \begin{align}
        (\Lambda \otimes \text{id}_k)\rho \geq 0 &\implies \sum_i \lambda_i(\Lambda \otimes \id_k)(\kb{\phi_i}{\phi_i})\geq 0\\
        &\implies \forall i, (\Lambda \otimes \id_k)\kb{\phi_i}{\phi_i} \geq 0.
    \end{align}
    We saw that for each of the basis states $\ket{\phi_i}$, we could write it as
    \begin{equation}
        \ket{\phi_i}=(I\otimes R_i)\ket{\Omega}
    \end{equation}
    for some $R_i\in \cB(\CC^d, \CC^k)$. Thus we can rewrite the basis states in our inequality to get
    \begin{equation}
        (\Lambda \otimes \id_k)(I\otimes R_i)\kb{\Omega}{\Omega} (I\otimes R_i^\dagger)\geq 0.
    \end{equation}
    Note that with the following definition
    \begin{equation}
        (\id_d \otimes f)(\omega := (I\otimes R_i)(\omega(I\otimes R_i^\dagger),
    \end{equation}
    our inequality becomes
    \begin{equation}
        (\Lambda \otimes \id_k)(\id_d \otimes f_i)(\kb{\Omega]{\Omega}}) \geq 0.
    \end{equation}
    Rewriting, this expression becomes
    \begin{equation}
        (\id_{d'}\otimes f_i)(\Lambda \otimes \id_d)\kb{\Omega}{\Omega} = \underbrace{(I_{d'}\otimes R_i)}_A\underbrace{(\Lambda \otimes \id_d)(\kb{\Omega}{\Omega})}_{B}\underbrace{(I_{d'}\otimes R_i^\dagger)}_{A^\dagger}.
    \end{equation}
    This is equivalent to the condition on matrices that $ABA^\dagger \geq 0$, and it turns out that for $ABA^\dagger \geq 0$, it suffices to have $B\geq 0$.%
        \footnote{Exercise.}
    Thus
    \begin{equation}
        (\Lambda \otimes \id)\kb{\Omega}{\Omega} \geq 0. \qedhere
    \end{equation}
\end{proof}

This construction we have defined is known as the \term{Choi matrix} (a Choi state of $\Omega$), i.e.
\begin{equation}
    J\equiv J(\Lambda) = (\Lambda \otimes \id)\kb{\Omega}{\Omega}).
\end{equation}

\begin{thm}[Stinespring's dilation theorem]
    Let $\Lambda:\cB(\cH)\to \cB(\cH)$ be a quantum operator. Then there exists a Hilber space $\cH'$ and a unitary operator $U\in \cB(\cH\otimes \cH')$ such that $\forall \rho \in \cD(\cH)$,
    \begin{equation}
        \Lambda(\rho)=\Tr_{\cH'}(U(\rho\otimes \phi)U^\dagger)
    \end{equation}
    where $\phi$ is some fixed (pure) state in $\cH'$.
\end{thm}
That is, to perform a quantum operation we can couple to an ancilla, perform the unitary operation, and trace over the degrees of freedom in the ancilla $\cH'$.

Stinespring's dilation theorem is a result from operator theory, but we'll see shortly that there are two more equivalent and relevant formulations, known as the Kraus Representation Theorem and the C-J isomorphism. We'll discuss this first one today.

\begin{thm}[Kraus Rep'n Theorem]
    A linear map $\Lambda:\cM(\cH)\to \cB(\mathcal{K})$ is CP if
    \begin{equation}\label{krausdecomp}
        \Lambda(\rho)=\sum_{k=1}^r A_k \rho A_k^\dagger
    \end{equation}
    where $\set{A_k}_{k=1}^r$ is a finite set of linear operators in $\cB(\cH,\mathcal{K}).$ Moreover it is TP if
    \begin{equation}\label{krausunitaryish}
        \sum_{k=1}^r A_k^\dagger A_k = I_{\cH}.
    \end{equation}
\end{thm}
\begin{proof}
    We start by proving that the latter holds if the map is trace preserving and \ref{krausdecomp} holds. That is, trace preserving tells us that
    \begin{align*}
        1 &= \Tr \Lambda(\rho)\\
            &= \Tr \sum_k A_k \rho A_k^\dagger\\
            &= \sum_k \Tr(A_k \rho A_k^\dagger)\\
            &= \sum \Tr(A_k^\dagger A_k \rho)\\
            &= \Tr\paren{(\sum_k A_k^\dagger A_k)\rho} \,\forall \rho\\
            &\implies \sum_k A_k^\dagger A_k= I_{\cH}.
    \end{align*}
    Here, we have done nothing other than use definitions and the linearity and cyclic property of the trace.
\end{proof}
\subsection*{Kraus Rep'n Thm $\equiv$ restatement of Stin. D. Thm.}
WLOG assume $\phi\equiv \kb{\phi}{\phi} \in \cD(\cH')$. Let $\set{\ket{e_k}}_k$ be an onb for $\cH'$. By Kraus, we can write
\begin{equation}
    \Lambda(\rho)=\sum_k \bra{e_k}U(\rho\otimes \phi) U^\dagger \ket{e_k}=\sum_k A_k \rho A_k^\dagger.
\end{equation}
with $\phi$ defined as above. That is, $\Lambda(\rho)=\Tr_{\cH'}(U(\rho \otimes \phi)U^\dagger)$. We define
\begin{equation}
    A_k:= \bra{e_k}U \ket{\phi}
\end{equation}
where $U\in \cB(\cH \otimes \cH')$ and it follows that
\begin{align*}
    \sum_k A_k^\dagger A_k &= \sum_k \bra{\phi}U^\dagger \kb{e_k}{e_k} U \ket{\phi}\\
    &= \braket{\phi}{\phi}I_{\cH} = I_{\cH}.
\end{align*}
We call the $A_k$ Kraus operators. Some of the details are an exercise to fill in later.

\subsection*{Choi-Jamilkowski (C-J) isomorphism} We saw that $\Lambda:\cB(\cH)\to \cB(\mathcal{K})$ where $\cH\simeq \CC^d, \mathcal{K}\simeq \CC^{d'}$ is CP iff $J(\Lambda)=(\Lambda \otimes \id_d)\kb{\Omega}{\Omega} \geq 0$. In fact, it turns out that $\exists$ an isomorphism between linear maps and positive operators. This is a great result, since positive operators are much nicer to work with.
\begin{thm}
    The following equation provides a bijection between linear maps $\Lambda:\cM_d \to \cM_{d'}$ and operators $J \in \cB(\CC^{d'}\otimes \CC^d)$, with $J$ defined as follows:
    \begin{equation}\label{cjforwards}
        J\equiv (\Lambda \otimes \id_d)\kb{\Omega}{\Omega}
    \end{equation}
    and
    \begin{equation}\label{cjbackwards}
        \Tr(A\Lambda(B))=d \Tr (J(A\otimes B^T))
    \end{equation}
    $\forall A \in \cM_{d'},B\in \cM_d$.
    The maps $\Lambda \to J \to \Lambda$ defined by \ref{cjforwards} and \ref{cjbackwards} are mutual inverses and lead to the following correspondence:
    \begin{enumerate}
        \item $\Lambda$ is CP $\iff J \geq 0$.
        \item $\Lambda$ is TP $\iff \Tr_A J = I_d/d$.
    \end{enumerate}
\end{thm}
\begin{proof}
    We'll first prove that \ref{cjforwards}$\to$\ref{cjbackwards}. The RHS of \ref{cjbackwards} is
    \begin{align*}
        \text{RHS}&=d\Tr (J(A\otimes B^T))\\
        &= d\Tr((\Lambda\otimes \id)(\Omega)(A\otimes B^T)).
    \end{align*}
    Note we will need the \term{adjoint} $\Lambda^*$ of a map $\Lambda$ w.r.t. the Hilbert-Schmidt inner product. That is, if $\Lambda:\cB(\cH)\to \cB(\mathcal{K}),$ then $\Lambda^*(\cB(\mathcal{K})\to \cB(\cH)$ where
    \begin{equation}
        \Tr(A\Lambda(B)) = \Tr (\Lambda^*(A) B).
    \end{equation}
    Thus with the adjoint we have
    \begin{align*}
        \text{RHS}&=d\Tr((\Lambda\otimes \id)(\Omega)(A\otimes B^T))\\
        &= d\Tr(\Omega(\Lambda^*\otimes \id)(A\otimes B^T))\\
        &= d\Tr(\Omega(\Lambda^*(A)\otimes B^T))\\
        &= d\Tr \Lambda^*(A)\otimes B^T (\kb{\Omega}{\Omega}).
    \end{align*}
    Of course, we can split up the tensor product as
    \begin{align*}
        (\Lambda^*(A)\otimes B^T)\ket{\Omega} &= (\Lambda^*(A) \otimes I)(I\otimes B^T)\ket{\Omega}\\
        &= (\Lambda^*(A)\otimes I)(B\otimes I)\ket{\Omega}\\
        &=(\Lambda^*(A)B \otimes I)\ket{\Omega}.
    \end{align*}
    We get
    \begin{equation}
        d\bkt{\Tr(\Lambda^*(A)B\otimes I)\kb{\Omega}{\Omega}}=\Tr(A\Lambda(B)),
    \end{equation}
    where we trace over the second system and use the adjoint to get back $\Lambda$.
\end{proof}