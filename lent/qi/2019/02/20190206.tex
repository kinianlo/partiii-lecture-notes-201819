Last time, we introduced maximally entangled states. As it turns out, these states have a few interesting properties. Recall that such states are defined on composite Hilbert spaces such that for
\begin{equation}
    \cH_A \otimes \cH_B \simeq \CC^d \otimes \CC^d
\end{equation}
equipped with a (fixed) onb for $\CC^d$ given by $\set{\ket{i}}_{i=1}^d$, a maximally entangled state is then a state which is written
\begin{equation}
    \ket{\Omega}=\frac{1}{\sqrt{d}} \sum_{i=1}^d \ket{i}\ket{i}.
\end{equation}
\begin{itemize}
    \item Every MES $\ket{\Phi}\in \CC^d \otimes \CC^d$ can be written in the form
    \begin{equation}
        \ket{\Phi}=(I_d \otimes U) \ket{\Omega}
    \end{equation}
    for some unitary $U$. One should check explicitly%
        \footnote{The proof is quick.
        \begin{align*}
            \Tr_2(\kb{\Phi}{\Phi})&= \Tr_2 \paren{(I\otimes U) \frac{1}{\sqrt{d}}\sum_i \ket{i}\ket{i}}\paren{\frac{1}{\sqrt{d}} \sum_j \bra{j} \bra{j} (I\otimes U^\dagger)}\\
                &= \frac{1}{d}\sum_{i,j} \kb{i}{j} \Tr(U\kb{i}{j} U^\dagger)\\
                &= \frac{1}{d}\sum_{i,j} \kb{i}{j} \Tr(\kb{i}{j} U^\dagger U)\\
                &=\frac{1}{d} \sum_{i,j} \kb{i}{j} \delta_{ij}\\
                &= \frac{I}{d}.
        \end{align*}
        The proof for tracing over the first subsystem is almost the same. Strictly, what this shows is that every state of this form is maximally entangled. We haven't shown that every maximally entangled state admits this form.
        }
    that
    \begin{equation}
        \Tr_2 \ket{\Phi}\bra{\Phi}=\frac{I}{d}\text{ and }\Tr_1 \ket{\Phi}\bra{\Phi}=\frac{I}{d}.
    \end{equation}
    \item Lemma: for any $A,B\in B(\CC^d),$
    \begin{itemize}
        \item $\bra{\Omega}A\otimes B\ket{\Omega}=\frac{1}{d}\Tr(A^T B)$, where transposition is done in the basis $\set{\ket{i}}_{i=1}^d$.
        \item $(A\otimes I)\ket{\Omega}=(I\otimes A^T)\ket{\Omega}$, a property we shall call ``ricochet.'' The proofs of these lemmas are an exercise, and are done at the end of this lecture's notes.
        %We can do this by writing
        %\begin{gather*}
        %    A=\sum_{i,j} a_{ij}\ket{i}\bra{j}\\
        %    B=\sum_{k,l} a_{kl}\ket{k}\bra{l}\\
        %    A^T=\sum_{i,j} a_{ji}\ket{i}\bra{j}.
        %\end{gather*}
        %and computing explicitly.
    \end{itemize}
    \item We can write down a purification of a state $\rho$ in terms of $\ket{\Omega}$: we claim it is
    \begin{equation}
        \ket{\psi}=\sqrt{d}(\sqrt{\rho}\otimes I) \ket{\Omega}.
    \end{equation}
    Let us check:%
        \footnote{Note that $\sqrt{\rho}^\dagger$ is $\sqrt{\rho}$, since $\rho$ is positive and therefore Hermitian. Therefore $\bra{\psi}=\bra{\Omega} (\sqrt{\rho}\otimes I).$}
    \begin{align*}
        \ket{\psi}\bra{\psi}&=d(\sqrt{\rho} \otimes I) \ket{\Omega}\bra{\Omega} (\sqrt{\rho}\otimes I)\\
            &= \sum_{i,j}\sqrt{\rho}\ket{i}\bra{j}\sqrt{\rho} \otimes \ket{i}\bra{j}.
    \end{align*}
    Tracing over the second system, $\Tr(\ket{i}\bra{j})=\delta_{ij}$, so the partial trace is then
    \begin{equation}
        \Tr_2 \ket{\psi}\bra{\psi} = \sqrt{\rho} \sum_i \ket{i}\bra{i} \sqrt{\rho} = \rho.
    \end{equation}
    \item Every bipartite pure state $\ket{\psi}\in \cH_A \otimes \cH_B$ can be written in the form
    \begin{equation}\label{bipartitedecomp}
        \ket{\psi}=(I\otimes R) \ket{\Omega}
    \end{equation}
    for some operator $R$.
    \begin{proof}
        Let $\ket{\psi}=\ket{\psi_{AB}}=\sum \lambda_i \ket{i_A}\ket{i_B},$ by the Schmidt decomposition. Let $V,W$ be isometries such that
        \begin{gather}
            V\ket{i}=\ket{i_A}\forall i ; \quad V:\CC^d \to \cH_A\\
            W\ket{i}=\ket{i_B}\forall i; \quad W:\CC^d \to \cH_B.
        \end{gather}
        The proof is constructive. Choose $R \equiv W Q V^T$, where $Q$ is defined in terms of the Schmidt coefficients,
        \begin{equation}
            Q=\sum \sqrt{d} \lambda_j \ket{j}\bra{j}.
        \end{equation}
        Let us look at the RHS of \ref{bipartitedecomp}. For this choice of $R$, it is
        \begin{align*}
            &=(I\otimes WQV^T) \ket{\Omega}\\
            &=(I\otimes W)(I\otimes Q)(I\otimes V^T)\ket{\Omega}\\
            &=(I\otimes W)(I\otimes Q)(V\otimes I)\ket{\Omega}\\
            &= (V\otimes W)(I\otimes Q) \ket{\Omega}\\
            &= (V\otimes W) \frac{1}{\sqrt{d}} \sum_i \ket{ i}\otimes Q\ket{i}\\
            &= (V\otimes W) \frac{1}{\sqrt{d}} \sum_i \ket{ i}\otimes \paren{\sum_j \sqrt{d} \lambda_j \ket{j}\bra{j}}\ket{i}\\
            &=(V\otimes W) \sum_i \lambda_i \ket{i}\otimes \ket{i}\\
            &= \sum \lambda_i \ket{i_A} \ket{i_B}.\qedhere
        \end{align*}
        Here, we have used the ``ricochet'' property to interchange $I\otimes V^T$ to $V\otimes I$, and moved $V\otimes I$ through $I\otimes Q$ since they act on independent parts of the composite system.%
            \footnote{Notice also that if $\ket{\psi}$ is a maximally entangled state, then $Q$ is just the identity matrix. $W$ and $V$ are isometries which rotate basis elements into the Schmidt basis, hence unitary. Thus $V^T$ is also unitary and $WV^T$ is unitary by closure. This proves the earlier claim that every maximally entangled state $\ket{\Phi}$ can be written as $\ket{\Phi}=(I \otimes U)\ket{\Omega}$ for $U$ a unitary.}
    \end{proof}
\end{itemize}

\subsection*{Time evolution of open systems}
Question: what is the most general description of the dynamics of an \emph{open} quantum system? Answer: it is given by a linear \term{completely positive trace-preserving} (CPTP) map. The advantage of such a map is that it gives us a description of the effect of \emph{any} allowed physical process on your system, including operations like measurement. In particular, it also allows us to describe discrete state changes.

As all reasonable evolution operators should be linear, we will usually omit this from the description and just speak of a CPTP map. We can also reasonably call this a \term{quantum operator} or a \term{quantum channel}. That is, we have a map $\Lambda:\cD(\cH)\to \cD(\cH)$, e.g. it takes the density matrix $\rho$ from $\rho\mapsto \Lambda(\rho)=\rho'$. We call this a \term{superoperator} because it is a map from operators to operators.

\begin{exm}
    We've constructed a general description of open quantum systems, but it should include our previous description of closed quantum systems as a special case. We can do this by taking $\Lambda$ to be a unitary transformation, such that
    \begin{equation}
        \rho'=\Lambda(\rho)=U\rho U^\dagger.
    \end{equation}
\end{exm}

Let us now unpack some of the properties of CPTP maps.
\begin{itemize}
    \item This map satisfies linearity:
    \begin{equation*}
        \Lambda(a\rho_1 +b\rho_2) = a\Lambda(\rho_1) + b\Lambda(\rho_2).
    \end{equation*}
    We want our CPTP maps to be linear so that we can interpret mixed state density matrices in a probabilistic way. That is, if we have some distribution of density matrices $\rho_i$ given with some probabilities $p_i$ (i.e. a set $\set{p_i,\rho_i}_{i=1}^m$, then we can describe the system as a new density matrix
    \begin{equation*}
        \sigma=\sum_{i=1}^m p_i \rho_i,
    \end{equation*}
    and thus the map $\Lambda$ should also represent a valid map on the full system $\sigma$:
    \begin{equation*}
        \Lambda(\sigma)=\sum_{i=1}^m p_i \Lambda(\rho_i).
    \end{equation*}
    \item Positivity: for $\rho\geq 0, \rho' = \Lambda(\rho) \geq 0$. We say $\Lambda$ is a positive (or positivity-preserving) map if
    \begin{equation}
        \Lambda(A)\geq 0 \forall A \geq 0.
    \end{equation}
    \item $\Lambda$ must be trace-preserving, i.e. for $\rho$ with $\Tr \rho =1$, we want 
    \begin{equation}
        \Tr (\Lambda \rho)=\Tr \rho'=1.
    \end{equation}
\end{itemize}
These conditions are necessary, but not sufficient. In fact we, require $\Lambda$ to be \term{completely positive}, as we'll define now. 
\begin{defn}
    Let $\Lambda:\cD(\cH_A)\to \cD(\cH_A')$, where $\cH_A$ is the Hilbert space of this system. Consider an extension of $\cH_A$ to the bigger space $\cH_A \otimes \cH_B$. That is, we add another system $B,$ called the ancilla or (for obvious reasons) the environment. 

    Note that $I_B$ is the identity operator on $B$, i.e. $I_B\in \cB(\cH_B)$, whereas $\id_B$ is the superoperator $\cB(\cH_B)\to \cB(\cH_B)$ such that $\id_B Q=Q \,\forall Q\in B(\cH_B).$ We then say that $\Lambda$ is \term{completely positive} if $\Lambda \otimes \id_B$ is positive for all such extensions.
\end{defn}
For instance, suppose the composite system $AB$ is initially in a state $\rho_A\otimes \omega_B$. Thus a completely positive map yields a state
\begin{equation}
    (\Lambda \otimes id_B)(\rho_A \otimes \omega_B)=\sigma_{AB},
\end{equation}
where $\sigma_{AB}$ is guaranteed to be a legitimate state of the composite system $AB$.

\begin{exm}
    Let $\Lambda$ be the transposition map. This is certainly positive:
    \begin{equation}
        \Lambda \equiv T: \rho \to \rho^T.
    \end{equation}
    That is, if $\rho$ had no negative eigenvalues, then transposition will preserve the eigenvalues and therefore preserve positivity.
    
    We will now show that there exists a composite state which is positive, but not positive after the application of $\Lambda \otimes id_B$. Let the composite system $\cH_A\otimes \cH_B$ with $\cH_A,\cH_B\simeq \CC^d$ be described by the density matrix
    \begin{equation}
        \rho_{AB}=\ket{\Omega}\bra{\Omega}
    \end{equation}
    where
    \begin{equation}
        \ket{\Omega}=\frac{1}{\sqrt{d}} \sum_{i=1}^d \ket{i}\ket{i}
    \end{equation}
    is a MES. Now we hit the first part with the transpose:
    \begin{align*}
        (\Lambda \otimes id_B)\ket{\Omega}\bra{\Omega} &= \frac{1}{d} \sum T(\ket{i}\bra{j}\otimes \ket{i}\bra{j}\\
        &= \frac{1}{d} \sum \ket{j}\bra{i} \otimes \ket{i}\bra{j} \equiv \tilde \rho.
    \end{align*}
    Now we ask whether $\tilde \rho \geq 0.$ The factor $d$ certainly doesn't change the positivity of the state, so take $Q\equiv d\tilde \rho$ and consider its action on some states $\ket{\phi}=\sum_k a_k \ket{k},\ket{\psi}=\sum_l b_l \ket{l}.$ Then
    \begin{align*}
        Q(\ket{\phi}\otimes \ket{\psi}) &= \paren{\sum \ket{j}\bra{i} \otimes \ket{i}\bra{j}}\paren{\sum a_k \ket{k} \otimes \sum b_l \ket{l}}\\
        &= \sum_{i,j} a_i \ket{j} \otimes b_j \ket{i}\\
        &=\sum_j b_j \ket{j} \otimes \sum_i a_i \ket{i} = \ket{\psi}\otimes \ket{\phi}.
    \end{align*}
    What we see is that $Q$ has swapped the states between the Hilbert spaces,
    \begin{equation}
        Q(\ket{\phi}\otimes \ket{\psi}) =\ket{\psi}\otimes \ket{\phi} \implies Q^2=I.
    \end{equation}
    This tells us that the eigenvalues of $Q$ are $\pm 1$, which means that we have constructed an operator which is positive but not completely positive.
\end{exm}

\subsection*{Non-lectured aside: extra proofs}
These proofs were originally footnotes, but I thought it might be useful to collect them here at the end of the lecture to avoid clutter.

\begin{proof}
    Trace and $\ket{\Omega}$: we show that $\bra{\Omega}A\otimes B\ket{\Omega}=\frac{1}{d}\Tr(A^T B)$.
    
    Note that by the usual laws of matrix multiplication, if $A=a_{ij}\ket{i}\bra{j}$ and similarly $B=b_{ij}\ket{i}\bra{j}$, then $A^TB=a_{ji}B_{jl}\ket{i}\bra{l}$ and so 
    \begin{equation}
        \Tr(A^T B)=a_{ji} b_{jl} \braket{l}{i}=A_{ji}b_{ji}.
    \end{equation}
    
    Now by explicit computation, we see that
    \begin{align*}
        \bra{\Omega}A\otimes B\ket{\Omega}&= \frac{1}{\sqrt{d}}\bra{\Omega}(a_{ij} \ket{i}\braket{j}{k} \otimes b_{lm} \ket{l} \braket{m}{k})\\
        &=\frac{1}{\sqrt{d}} \bra{\Omega} (a_{ik} \ket{i} \otimes b_{lk} \ket{l})\\
        &= \frac{1}{d} a_{ik} \braket{n}{i} b_{lk} \braket{n}{l}\\
        &= \frac{1}{d} a_{nk} b_{nk}\\
        &= \frac{1}{d} \Tr(A^T B),
    \end{align*}
    where we have swapped $\ket{\Omega}$s freely for their expressions in terms of an orthonormal basis and evaluated the Kronecker deltas implicitly rather than writing them out.
\end{proof}

\begin{proof}
    Ricochet property: we wish to prove that
    \begin{equation*}
        (A\otimes I)\ket{\Omega}=(I\otimes A^T)(\ket\Omega).
    \end{equation*}
    For brevity, I'm suppressing the sums in the following expressions. All sums are taken over $1$ to $d$. Let $A=a_{ij}\ket{i}\bra{j}$, and thus $A^T=a_{ji}\ket{i}\bra{j}$. Then
    \begin{align*}
        (A\otimes I)\ket{\Omega} &= a_{ij}\ket{i} \braket{j}{k}\otimes \ket{k}\\
            &= a_{ij} \ket{i} \delta_{jk} \otimes \ket{k}\\
            &= a_{ik} \ket{i} \otimes \ket{k}\\
            &= a_{ki} \ket{k} \otimes \ket{i}\\
            &= \ket{k} \otimes a_{ji}\ket{i} \delta_{jk}\\
            &= \ket{k} \otimes a_{ji}\ket{i}\braket{j}{k}\\
            &= (I\otimes A^T)\ket{\Omega},
    \end{align*}
    where we have simply relabeled $i$ and $k$ in the fourth line since both sums run from $1$ to $d$.
\end{proof}