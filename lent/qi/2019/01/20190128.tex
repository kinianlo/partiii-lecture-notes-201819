Last time, we introduced the setup of Shannon's second key theorem, the noisy channel theorem.

Recall our problem-- Alice has a message she wants to send to Bob, but she only has access to a noisy channel (defined by a stochastic map) which has some probability of corrupting her message when she sends it. Therefore Alice selects a codeword, translating her message $m\in [M]$ to a codeword $x^{(n)}$ which she then sends through the noisy channel $\mathcal{N}$.

The channel then outputs a transmitted (still encoded) message $y^{(n)}$ with probability
\begin{equation}
    p(y^{(n)}|x^{(n)}) \equiv \prod_{i=1}^n p(y_i| x_i),
\end{equation}
and Bob then decodes this transmission to get a decoded message $m'$.

We say that a code $(\mathcal{E}_n,\mathcal{D}_n)$ (i.e. an encoding-decoding scheme) has rate $R$ if $\ceil{M}\approx 2^{nR}$. Thus $R =\frac{\log |M|}{n}$. A rate is \term{achievable} if there exists a code with that rate such that the probability of error after decryption goes to zero in the limit as $n\to \infty$.

Shannon's theorem tells us that the capacity $C(\mathcal{N})$ (i.e. the supremum of all achievable rates) is precisely related to the mutual information between the inputs and outputs of the noisy channel:
\begin{equation}
    C(\mathcal{N})=\max_{\set{p(x)}_{x\in J_X}} I(X: Y).
\end{equation}

\begin{exm}
    Consider our m.b.s.c. from last time. Recall the mutual information is defined
    \begin{equation}
        I(X:Y)=H(Y)-H(Y|X),
    \end{equation}
    where $H(Y|X)=\sum_{x\in J_X} p(x) H(Y|X=x)$, with $H(Y|X=x)=-\sum_{y\in J_Y} p(y|x) \log p(y|x).$
    But of course we can explicitly compute these entropies%
        \footnote{$p(y|x)$ is given to us in the channel matrix, so for example
        \begin{align*}
            H(Y|X=0) &= -\sum_{y\in J_Y} p(y|0) \log p(y|0)\\
            &= -\bkt{(1-p)\log (1-p) +p \log p}\\
            &= h(p),
        \end{align*}
        and $H(Y|X=1)$ is the same by the symmetry of the channel matrix.
        }
    and we can check that
    \begin{equation}
        H(Y|X=x)=h(p) \forall x\in \set{0,1}.
    \end{equation}
    Thus
    \begin{align*}
        I(X:Y)&=H(Y)-\sum p(x) h(p)\\
        &= H(Y) - h(p) \leq \log |J_Y|-h(p),
    \end{align*}
    so
    \begin{align*}
        C(\mathcal{N})&=\max_{\set{ p(x)}} I(X:Y)\\
        &= H(Y)-h(p)\\
        &\leq \log |J_Y|-h(p).
    \end{align*}
    Could we have equality? That is, $\set{p(y)}$ such that $H(Y)=\log|J_Y|$. This happens if we have outcomes $y$ in a uniform distribution. What are the initial probabilities $\set{p(x)}$? Well,
    \begin{equation}
        p(y)=\sum_x p(y|x) \cdot p(x),
    \end{equation}
    and we find that $p(x)=1/2$ for $x=0$ and $1/2$ for $x=1$, with $p(y) = 1/2$ for $y=0$ and $1/2$ for $y=1$.
    
    We therefore find that the capacity of an m.b.s.c is
    \begin{equation}
        C(\mathcal{N}_{mbsc})=\log|J_Y|-h(p) = 1 - h(p).
    \end{equation}
\end{exm}

As a quick note, the input and output alphabets need not be of equal size. Consider the \term{binary erasure channel}, which transmits the input with probability $1-p$ and erases the input with probability $p$. Thus $J_X=\set{0,1}$ and $J_Y=\set{0,1,e}$.

%encoding-decoding Tikz figure
\begin{center}
    \begin{tikzpicture}[scale=2, node distance = 2cm]
        %make nodes
        \node [] (startzero) {$0$};
        \node [right of=startzero] (endzero) {$0$};
        \node [below of=endzero, node distance = 1cm] (enderase) {$e$};
        \node [below of=enderase, node distance = 1cm] (endone) {$1$};
        \node [left of=endone] (startone) {$1$};
        
        %make lines
        \path [line, dashed] (startzero) -- node[above] {$1-p$} (endzero);
        \path [line, dashed] (startone) -- node[below] {$1-p$} (endone);
        \path [line] (startzero) -- node[below, near start] {$p$} (enderase);
        \path [line] (startone) -- node[above, near start] {$p$} (enderase);
        %\path [line] (start) -- node[below] {$m\in M$} (encoding);
    \end{tikzpicture}
\end{center}

Recall the intuitive picture of the theorem. Shannon realized that for certain codewords, their images after applying the channel map $\mathcal{N}^n$ will represent disjoint subsets in $J_Y^n$ in the asymptotic limit. The maximal rate is then the number of codewords with this property we can choose divided by $n$ the codeword length, or equivalently the number of disjoint subsets we can pack into $J_Y^n$.

Now for each input sequence $\ul x^{(n)}$ of length $n$, how many typical $Y$ sequences will we get? Recall that there are $|T_n| \approx 2^{nH(X)}$ typical sequences for our variable $X\sim p(x)$. So translating this formula through our channel, we expect to get
\begin{equation}
    |T_n|\approx 2^{nH(Y|X)}
\end{equation}
typical sequences in $J_Y^n$. The total number of possible typical $Y$ sequences is $2^{nH(Y)}$ using the induced distribution $\set{p(y)}$. Therefore we expect to be able to partition the set of typical $Y$ sequences into a number of disjoint typical sets given by
\begin{align}
    \frac{2^{nH(Y)}}{2^{nH(Y|X)}} &\approx 2^{n(H(Y)-H(Y|X))}\\
    &= 2^{nI(X:Y)},
\end{align}
so heuristically, $C(\mathcal{N})=\max_{\set{p(x)}} I(X:Y).$

Note that this theorem \emph{does not} translate directly to the quantum case. The classical proof relies on a notion of joint probability of two typical sequences, which has no analogue in QI.%
    \footnote{I suspect this is due to entanglement.}
    
\subsection*{QIT preliminaries} Consider a quantum system $A$. Its states are described by a Hilbert space $\cH_A$, where we will take $\dim H$ to be finite. That is, a finite-dimensional Hilbert space is a complex inner product space, i.e. a set with a vector space structure over the field $\CC$ equipped with an inner product $(\cdot,\cdot):\cH \times \cH \to \CC$.

\begin{defn}
    An \term{inner product} is a bilinear function obeying the following properties:
    \begin{itemize}
        \item $(v,v')=(v',v)^* \,\forall v,v' \in \cH$
        \item $(v,av')=a(v,v')$ and $(v,v_1+v_2)=(v,v_1)+(v,v_2)$.
        \item $(v,v)\geq 0$  (positive semidefinite), with equality when $v=0.$
    \end{itemize}
\end{defn}
The inner product induces a norm on $\cH$, defined
\begin{equation}
    ||v||=\sqrt{(v,v)}, \cH \to \RR.
\end{equation}
The norm defines a distance between two vectors,
\begin{equation}
    d(v,v')=||v-v'||,
\end{equation}
which has the properties of being symmetric, with $d(v,v')=0$ iff $v=v'$, and obeying the triangle inequality, 
\begin{equation}
    d(u,v) \leq d(u,v') + d(v,v').
\end{equation}

The Cauchy-Schwarz inequality also holds, i.e.
\begin{equation}
    \forall v, v' \in \cH, |(v,v')|\leq \sqrt{(v,v) (v',v')}.
\end{equation}

\subsection*{Linear maps/operators on $\cH$}
\begin{itemize}
    \item We call a map $A:\cH \to \cH'$ a homomorphism, with the set $A\in \cB(\cH, \cH')=\text{Homo}(\cH, \cH').$
    \item When $\cH' = \cH$, we call such a map an endomorphism and denote the set of such maps $\text{End}(\cH)$.
    \item The simplest operator we can define is the identity map, $1\in \cB(\cH)$ such that $1v = v \forall v\in \cH$.
    \item We may also define the adjoint of a homomorphism, $A^\dagger: \cH' \to \cH$. Thus if $A\in \cB(\cH, \cH'),$ then $A^\dagger \in \cB(\cH',\cH).$ Thus $A^\dagger$ is defined to be the unique operator satisfying
    \begin{equation}
        (v',Av) = (A^\dagger v',v)
    \end{equation}
    with $(A^\dagger)^\dagger = A$, where $v\in \cH, v' \in \cH'.$ Note that the set of homomorphisms and endomorphisms can be promoted to Hilbert spaces by defining an inner product, the \term{Hilbert-Schmidt inner product}, defined as
    \begin{equation}
        (A,B)_{HS}=\Tr(A^\dagger B).
    \end{equation}
\end{itemize}

\subsection*{Matrix representation} Since $\cH$ is finite-dimensional by assumption, it can be given a basis $\set{v_i}^d_{i=1}$ where $d=\dim \cH$. Thus an element $A\in B(\cH)$ can be represented by a $d\times d$ matrix $A_{ij}$ with elements
\begin{equation}
    A_{ij}=(v_i,A v_j).
\end{equation}
If $\cH=\CC^d$, for instance, then $\cB(\cH)=\cB(\CC^d) \equiv \cM_d$, the set of $d\times d$ complex matrices. This is the space of linear maps on length-$d$ complex vectors. In $d=2,$ we would have
\begin{equation}
    A=\begin{pmatrix}
        a & b\\
        c & d
    \end{pmatrix}, A^\dagger = \begin{pmatrix}
        a^* & c^* \\
        b^* & d^*
    \end{pmatrix}.
\end{equation}

Now maps $A$ have the property that if $A=A^\dagger$, then $A\geq 0$, i.e. $A$ is positive semidefinite, so that $\forall v\in \cH, (v,Av)\geq 0.$ %If $A\geq 0, A^2=A$.
