Today, we shall begin our discussion of quantum information theory. First, a quick review of Dirac's bra-ket notation-- we denote a vector in Hilbert space $\cH =\CC^d$ by
\begin{equation}
    \ket{v} = \begin{pmatrix}
    v_1\\ \vdots \\ v_d
    \end{pmatrix},
\end{equation}
and call this a \term{ket}. We also have the dual vectors (row vectors, if you like), called \term{bras}. such that
\begin{equation}
    \bra{v}=(v_1^*, \ldots, v_d^*).
\end{equation}
The braket notation provides us with a natural inner product:
\begin{equation}
    (u,v)=\braket{u}{v}=\sum_{i=1}^d u_i^* v_i.
\end{equation}
This space also comes equipped with an outer product, $\ket{u}\bra{v}$, which is the matrix
\begin{equation}
    \ket{u}\bra{v} =\begin{pmatrix}
    u_1 v_1^* &\ldots&\\
    \vdots\\
    u_d v_1^* &\ldots &u_d v_d^*
    \end{pmatrix}.
\end{equation}
We can then take an orthonormal basis (onb) for $\cH$, which we denote by $\set{\ket{e_i}}$ with $\braket{e_i}{e_j}=\delta_{ij}$. Note that for any basis of $\cH$, we can write the identity matrix as
\begin{equation}
    I=\sum_{i=1}^d \ket{e_i}\bra{e_i}.
\end{equation}
There is a nice basis $\ket{e_1}=\begin{pmatrix}1\\0\\\vdots\\0\end{pmatrix}$ we can write down,
so that for a general basis $\set{\ket{f_i}}_{i=1}^d$ related to the original by a unitary $U$, we find that
\begin{equation}
    \sum_{i=1}^d \ket{f_i}\bra{f_i}=\sum_{i=1}^d U \ket{e_i} \bra{e_i} U^\dagger = U I U^\dagger =UU^\dagger = I.
\end{equation}

Now in classical information, our simplest system was a binary bit, a system taking values $0$ and $1$. For quantum information theory, we have a \term{qubit}, a two-level system represented by a Hilbert space with $\cH=\CC^2$ and basis vectors $\set{\ket{0},\ket{1}}$ or equivalently $\set{\ket{\uparrow},\ket{\downarrow}}.$ Physically, these could be the spin states of an electron or perhaps the polarizations of a photon.

Now, it is obvious that any state in Hilbert space can be decomposed in the basis of our choice, i.e.
\begin{equation}
    \ket{\psi}=a\ket{0}+b\ket{1},
\end{equation}
with $a,b\in \CC$. We shall require that our states are normalized under this inner product, so that
\begin{equation}
    1=\braket{\psi}{\psi}=|a|^2+|b|^2,
\end{equation}
which means that $|a|^2$ and $|b|^2$ have the interpretation of probabilities.

We also have some important operators on Hilbert space. These are the Pauli matrices
\begin{gather*}
    \sigma_0 = \begin{pmatrix}
        1 & 0\\
        0 & 1
    \end{pmatrix},\quad
    \sigma_x = \begin{pmatrix}
        0 & 1\\
        1 & 0
    \end{pmatrix},\\
    \sigma_y = \begin{pmatrix}
        0 & i\\
        -i & 0
    \end{pmatrix},\quad
    \sigma_z = \begin{pmatrix}
        1 & 0\\
        0 & -1
    \end{pmatrix}.
\end{gather*}
As it turns out, these operators form a basis for hermitian operators on $\cH$. Note that we have a set of self-adjoint $2\times 2$ complex matrices
\begin{equation}
    \cB_{SA}(\CC^2)=\set{A\in \cB(\cH): A=A^\dagger},
\end{equation}
and we can write a general matrix $M\in M_2 /M_{sa}$ in terms of the Pauli matrices,
\begin{equation}
    M=\frac{1}{2}(x_0 \sigma_0 +\vec x \cdot \gv \sigma),
\end{equation}
where $\vec x=(x_1,x_2,x_3)\in \RR^3$.

\subsection*{Spectral decomposition}
The spectral decomposition says that we can write a matrix in terms of its eigenvalues $\lambda_i$ and eigenvectors $\ket{e_i}$,
\begin{equation}
    A=\sum_{i=1}^d \lambda_i \ket{e_i}\bra{e_i},
\end{equation}
such that $A\ket{e_i}=\lambda_i \ket{e_i}$. Sometimes we say that the eigenvalue decomposition is written in terms of projectors instead,
\begin{equation}
    A=\sum_{i=1}^m \lambda_i \Pi_i
\end{equation}
where $\Pi_i=\dyad{e_i}$ projects onto the basis of eigenvectors.

Given a self-adjoint operator $A=A^\dagger$ and a nice function $f$, what is the value $f(A)$? Note that $A$, being self-adjoint, can be diagonalized by a unitary. Thus
\begin{equation}
    A_d = UAU^\dagger \implies A = U^\dagger A_d U,
\end{equation}
so that
\begin{equation}
    f(A)=U^\dagger \begin{pmatrix}
    f(\lambda_1) && \\
    & \ddots & \\
    && f(\lambda_d)
    \end{pmatrix}.
\end{equation}
Thus for example
\begin{equation*}
    f(A)=e^{iA}=I + iA +\frac{i^2}{2!}+\ldots.
\end{equation*}

\subsection*{QM postulates}
We consider the following postulates of quantum mechanics, which will in fact be qualified by the fact we are working in an open system.
\begin{enumerate}
    \item The state of a (closed) system is given by a ray in $\cH$, i.e. a vector defined up to a global phase. Thus we cannot distinguish a state $\ket{\psi}$ and $e^{i\phi}\ket{\psi}$ by any physical measurement. We traditionally take a representative of this equivalence class, $\ket{\psi}.$
\end{enumerate}
For an open system $A$, consider a system which is in states $\ket{\psi_i}$ with some coefficients $p_i, i=1,\ldots, m$. The state is characterized by an ensemble
\begin{equation}
    \set{p_i,\ket{\psi_i}}_{i=1}^m.
\end{equation}
Note that these $\ket{\psi_i}$s need not be mutually orthogonal,
\begin{equation}
    \braket{\psi_i}{\psi_j}\neq \delta_{ij},
\end{equation}
and moreover this is \emph{not} a superposition but a statistical mixture. A superposition is a \emph{pure state} where the state is normalized and can be written as 
\begin{equation}
    \ket{\Psi}=\sum_{i=1}^d a_i \ket{\phi_i}.
\end{equation}

So a statistical mixture is instead described by a \term{density matrix} (or density operator). We could write our ensemble as
\begin{equation}
    \rho \equiv \sum_{i=1}^m p_i \ket{\psi_i} \bra{\psi_i},
\end{equation}
noting that the $\ket{\psi_i}$s in general \emph{need not be orthogonal}.
\begin{defn}
    A \term{density matrix} on $\cH$ ($\dim \cH = d$) is an operator $\rho$ with the following properties:
    \begin{itemize}
        \item $\rho \geq 0$, i.e. $\rho$ is positive semi-definite, $\bra{\phi}\rho \ket{\phi}\geq 0,$ which implies that $\rho=\rho^\dagger$.
        \item $\Tr \rho =1$ (which gives it a probabilistic interpretation).
    \end{itemize}
\end{defn}
Let us remark that $\rho$ is hermitian and therefore admits a spectral decomposition, i.e.
\begin{equation}
    \rho = \sum_{j=1}^d \lambda_j \ket{e_j}\bra{e_j}
\end{equation}
in terms of an orthonormal basis. Thus
\begin{equation}
    \rho=\sum_{i=1}^m p_i \ket{\psi_i}\bra{\psi_i} = \sum_{j=1}^d \lambda_j \ket{e_j} \bra{e_j}.
\end{equation}
We will prove on Examples Sheet 2 that the set
$\mathcal{D}(\cH)$ of density matrices is a convex set.

\subsection*{Pure and mixed states} 
Consider a density matrix
\begin{equation}
    \rho = \sum p_i \ket{\psi_i}\bra{\psi_i},
\end{equation}
and suppose for example that $p_2=1, p_i=0 \forall i\neq 2$. Then
\begin{equation}
    \rho=\ket{\psi_2}\bra{\psi_2}.
\end{equation}
This is very nice, because we know precisely the state of the system (or equivalently the outcome of applying the operator $\rho$). We call this a \term{pure state}, referring either to the vector $\ket{\psi_2}$ or the operator $\ket{\psi_2}\bra{\psi_2}$. Otherwise, $\rho$ is a \term{mixed state}.

A pure state will have $\rho^2 = \rho$, so we can define the \term{purity} of a state by $\Tr \rho^2$. Conversely, we can define a completely mixed state by
\begin{equation}
    \rho = I/d=\frac{1}{d} \sum_{i=1}^d \ket{e_i} \bra{e_i},
\end{equation}
such that a completely mixed state has purity $1/d$ (where we get a factor of $d$ from taking the trace of $I$).%
    \footnote{Explicitly, the trace is $\Tr \rho^2 = \sum_{i=1}^d \sum_{j=1}^d \frac{1}{d^2} \braket{e_j}{e_i}\braket{e_i}{e_j}=\frac{1}{d^2}\sum_{i=1}^d \sum_{j=1}^d \delta_{ij} \delta_{ij} = \frac{1}{d^2}\sum_{i=1}^d 1 = 1/d.$
    }

In classical probability, we remark that the convex set of probability distributions forms a \term{simplex}.

Now let's briefly discuss the expectation value of an observable (self-adjoint operator) in $B(\cH)$. For a state described by a density matrix $\rho$, we define the expectation value to be
\begin{equation}
    \phi(A) \equiv \avg{A}_\rho =\Tr(A \rho).
\end{equation}
This is a linear normalized functional--
\begin{itemize}
    \item $\phi(aA +b B) = a\phi(A) + b\phi(B)$
    \item $\phi(A)\geq 0$ with equality when $A=I$.
\end{itemize}