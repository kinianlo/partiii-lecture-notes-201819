Last time, we discussed the HSW theorem, stating that
\begin{equation}
    C^{(1)}(\Lambda)=\chi^*(\Lambda)=\max_{\set{p_x,\rho_x}} \chi(\set{p_x,\Lambda(\rho_x)}).
\end{equation}
\begin{lem}
    Any quantum channel can transmit classical information as long as it is not a constant channel, i.e. $\Lambda(\rho)$ not identical for all $\rho$.
\end{lem}
\begin{proof}
    If $\Lambda$ is not a constant channel, $\exists\ket{\psi},\ket{\phi}$ s.t. 
    \begin{equation}
        \Lambda(\dyad{\psi})\neq \Lambda(\dyad{\phi}).
    \end{equation}
    Hence for the ensemble $\mathcal{E}=\set{p_1=p_2=1/2, \rho_1= \dyad{\psi},\rho_2=\dyad{\phi}}$, we have a Holevo $\chi$ quantity
    \begin{equation}
        \chi(\mathcal{E})=S(\frac{1}{2}\Lambda(\psi) +\frac{1}{2} \Lambda(\phi)) -\bkt{\frac{1}{2}S(\Lambda(\psi))+\frac{1}{2} S(\Lambda(\phi))} >0
    \end{equation}
    since equality is achieved in concavity of $S$ only for $\Lambda(\psi)=\Lambda(\phi)$, which we required was false. Hence $\chi^*(\Lambda)>0$ since it is the maximum over all ensembles, so we can send classical information using product states.
\end{proof}

$\chi^*(\Lambda)$ has the following property, known as  \term{superadditivity}:
\begin{equation}
    \chi^*(\Lambda_1 \otimes \Lambda_2) \geq \chi^*(\Lambda_1) +\chi^*(\Lambda_2).
\end{equation}
The proof is an exercise on Examples Sheet 4. It follows by iteration that
\begin{align*}
    \chi^*(\Lambda^{\otimes n}) &= \chi^*(\Lambda \otimes \Lambda^{\otimes n-1})\\
        &\geq \chi^*(\Lambda) +\chi^*(\Lambda^{\otimes n-1})\\
        &\geq n \chi^* (\Lambda)
\end{align*}

Question: can one increase the classical capacity of a quantum channel by using entangled input states? To address this question, we have the \term{additivity conjecture of $\chi^*$} (since resolved):
\begin{equation}
    \chi^*(\Lambda_1 \otimes \Lambda_2)=\chi(\Lambda_1) +\chi^*(\Lambda_2).
\end{equation}
%proving it is mightily difficult
We have a formal expression for the classical capacity for product state inputs:
\begin{equation}
    C_{cl}(\Lambda)=\lim_{n\to \infty} \frac{1}{n} \chi^*(\Lambda^{\otimes n}).
\end{equation}
Notice that applying superadditivity, we have
\begin{equation}
    C_{cl}(\Lambda) \geq \lim_{n\to \infty}\frac{n}{n} \chi^*(\Lambda),
\end{equation}
so that if \emph{additivity} holds,
\begin{equation}
    C_{cl}(\Lambda)=\chi^*(\Lambda).
\end{equation}
Therefore if additivity is true, then entangled inputs do not improve the capacity of the channel. However, an insightful proof by Matt Hastings established a counterexample to the additivity conjecture. Finding a counterexample is very difficult-- it's more like an existence proof. But many examples of additivity do exist, e.g.
\begin{equation}
    \chi^*(\Lambda) \text{ for }\Lambda(\rho)=(1-p) \rho+p\frac{I}{2}
\end{equation}
the depolarizing channel. Hastings used a powerful theorem by Peter Shor linking four different quantities, $\chi^*, S_{\min}(\Lambda)=\min_\rho S(\Lambda(\rho))$, and two others-- which says that if one is additive, then all are, and conversely if one is not additive, then all are not. What Hastings found was a counterexample to the additivity of $S_{\min}$. (This is non-examinable.)

\subsection*{Quantum capacity}
In transmitting a quantum state, we're not interested in the probability of error but rather the fidelity between the initial and final states, $F(\rho,\sigma)$. Let us first define the \term{coherent information}. 

Suppose we start with an arbitrary (possibly mixed) state
\begin{equation}
    \rho_Q \in \cD(\cH_Q).
\end{equation}
We can purify $\rho_Q$ by coupling to a reference system $R$ to get a pure state
\begin{equation}
    \ket{\psi^\rho_{RQ}}\in \cH_Q\otimes \cH_R.
\end{equation}
Suppose we now want to send $\rho_Q$ through a quantum channel $\Lambda:Q\to Q'$. Stinespring says we can implement this quantum channel by coupling an environment $E$ and acting on the $QE$ part of the system with a unitary $U_\Lambda$, so we have a tripartite initial state
\begin{equation}
    \ket{\psi_{RQE}}=\ket{\psi^\rho_{RQ}}\otimes \ket{0_E} \in \cH_R \otimes \cH_Q \otimes \cH_E,
\end{equation}
where we may choose $\ket{0_E}$ to be a pure state of the environment $E$. By Stinespring, we run our quantum channel on the $QE$ part to get an output state
\begin{equation}
    \ket{\psi_{RQE}'}=(I_R \otimes U_\Lambda)\ket{\psi_{RQE}}.
\end{equation}
Notice that because $U_\Lambda$ is unitary, this output state is also pure.
\begin{defn}
    The \term{coherent information} of a tripartite pure state $\ket{\psi}_{ABE}$ is defined to be
    \begin{equation}
        I_c^{A>B}(\psi)=-S(A|B)_\psi.
    \end{equation}
    For a state $\rho_Q$, a channel $\Lambda$, and the output state $\ket{\psi_{RQE}'}$ as defined above, the coherent information of a quantum channel is defined to be
    \begin{equation}
        I_c(\Lambda,\rho_Q)=I_c^{R>Q'}(\psi_{RQE}')=-S(R|Q').
    \end{equation}
\end{defn}

\begin{lem}
    The coherent information is bounded above by the von Neumann entropy of the input state,
    \begin{equation}
        I_c(\Lambda,\rho) \leq S(Q).
    \end{equation}
\end{lem}
\begin{proof}
    Note that
    \begin{align*}
        I_c(\Lambda,\rho)&:= -S(R|Q')_{\rho'}\\
            &=-\bkt{S(\rho_{RQ'}')-S(\rho'_{Q'})}\\
            &= -\bkt{S(\rho'_E)-S(\rho'_{RE})}\\
            &= S(R|E)_{\rho'},
    \end{align*}
    where we have used the fact that $\rho'_{RQ'E}$ is pure to swap out subsystem entropies. Now we can write
    \begin{align*}
        I_c(\Lambda,\rho) &= S(R|E)_{\rho'}\\
            &= S(\rho'_{RE})-S(\rho_E')\\
            &\leq S(\rho'_R)+S(\rho'_E) -S(\rho'_E) \quad \text{by subadditivity}\\
                &= S(\rho_R) = S(\rho_Q) \qquad \text{ since $R'=R$ and $\rho_{RQ}$ is pure}.
    \end{align*}
    We conclude that
    \begin{equation}
        I_c(\Lambda,\rho)\leq S(\rho).\qedhere
    \end{equation}
\end{proof}

The coherent information is an interesting quantity because of its relationship to the quantum capacity of a channel, i.e. the maximum rate at which \emph{quantum information} can be reliably transmitted using that channel. This relationship is summarized in the following theorem, which we will not prove.
\begin{thm}[LSD]
    Let $\Lambda$ be a memoryless quantum channel. Then
    \begin{equation}
        Q(\Lambda)=\lim_{n\to \infty} \frac{1}{n} \max_{\rho^{(n)} \in \cD(\cH^{\otimes n})} I_c(\Lambda^{\otimes n}, \rho^{(n)})
    \end{equation}
    where $I_c(\Lambda,\rho)$ is the coherent information with respect to input $\rho$, and $Q$ is the quantum capacity of the channel $\Lambda$.
\end{thm}
%Let $\Lambda:Q\to Q'$. It has a Stinespring implementation $U_\Lambda:QE \to Q'E$, where $E$ is an ancilla we can take WLOG to be in a pure state $\ket{0_E}$. We can also purify $Q$ by introducing a reference system $R$. Hence the initial state is $\ket{\psi_{RQE}}=\ket{\psi_{RQ}^\rho}\otimes \ket{0_E}$. After the operation,

%But now

%useful fact: for a pure state, the entropies of the subsystems are equal by Schmidt

\subsection*{DPI for quantum systems}
Classically, we talked about the data processing inequality, by which a system $X\to Y\to Z$ obeyed $I(X:Y)\geq I(X:Z)$. That is, data processing cannot improve the mutual information between two systems.

Here, the quantum analogue is
\begin{equation}
    S(\rho)\geq I_c(\Lambda_1,\rho) \geq I_c(\Lambda_2 \circ \Lambda_1,\rho).
\end{equation}
The proof is by strong subadditivity.
\begin{proof}
    We have already proved the upper bound on $I_c(\Lambda_1,\rho)$. To show the seocond inequality, we can describe applying the two maps $\Lambda_1,\Lambda_2$ with their Stinespring implementations. By strong subadditivity, we know that for a tripartite system $ABC$, $S(A|BC)\leq S(A|B)$. Hence with the final state $\rho_{RE_1E_2}''$, we have
    \begin{equation}
        S(R|E_1E_2)_{\rho''}\leq S(R|E_1)_{\rho''},
    \end{equation}
    where the LHS is just $I_c(\Lambda_2 \circ \Lambda_1, \rho)$. What is the right side? It is
    \begin{align*}
        S(R|E_1)_{\rho''} &= S(\rho_{RE_1}'') -S(\rho_{E_1}'')\\
            &=S(\rho_{RE_1}')-S(\rho_{E_1}')\\
            &=S(R|E_1)_{\rho'}\\
            &= I_c(\Lambda_1,\rho).
    \end{align*}
    Hence the DPI holds:
    \begin{equation}
        S(\rho) \geq I_c(\Lambda_1,\rho) \geq I_c(\Lambda_2 \circ \Lambda_1,\rho).\qedhere
    \end{equation}
\end{proof}
\begin{note}
Some quick notes on (non-)examinable material: the entanglement fidelity is non-examinable. The converse of the Schumacher compression theorem is non-examinable (the one with four inequalities). Justify manipulations when using e.g. SSA.
\end{note}