Today we will complete the proof of the Holevo bound. We said that Alice's message, the encoded state, and Bob's measurement device were written as a tripartite initial state
\begin{equation}
    \rho_{AQB}=\sum_x p_x \dyad{x} \otimes \rho_x \otimes \dyad{\phi_B},
\end{equation}
which became
\begin{equation}
    \rho_{A'Q'B'} = \sum_{x,y} p_x \dyad{x} \otimes \sqrt{E_y} \rho_x \sqrt{E_y} \otimes \dyad{y_B}.
\end{equation}
We know that SSA has the following consequences:
\begin{enumerate}
    \item[(1)] $I(A:B) \leq I(A:BC)$
    \item [(2)] $I(A':B')\leq I(A:B)$
\end{enumerate}
where $\rho_{A'B'}=(\id_A \otimes \Lambda) \rho_{AB}$. Thus notice that
\begin{equation}
    I(A:Q) = I(A:QB)
\end{equation}
since $B$ is uncorrelated initially.
\begin{equation}
    I(A:QB) \geq I(A': Q'B')
\end{equation}
by (2) above, and 
\begin{equation}
    I(A':Q'B') \geq I(A': B')
\end{equation}
by (1). Hence
\begin{equation}
    I(A':B')\leq I(A:Q).
\end{equation}
In fact, this is the Holevo bound. For notice that the RHS is $I(A:Q)$, such that
\begin{align*}
    \rho_{AQ}&= \sum_x p_x \dyad{x} \otimes \rho_x\\
    \rho_A &= \sum p_x \dyad{x}\\
    \rho_Q &= \sum p_x \rho_x = \rho.
\end{align*}
The entropies of these states are
\begin{equation}
    S(\rho_A) = H(\set{p_x}),\quad S(\rho_Q)=S(\sum p_x \rho_x).
\end{equation}
Recall that on Examples Sheet 3, we showed that for $\omega_x$ with mutually orthogonal supports,
\begin{equation}
    S(\sum p_x \omega_x) = H(\set{p_x}) + \sum p_x S(\omega_x).
\end{equation}
Thus for the state $\rho_{AQ}=\sum p_x \omega_x$ wth $\omega_x=\dyad{x} \otimes \rho_x$, we have
\begin{equation}
    S(\rho_{AQ})=H(\set{p_x}) +\sum p_x S(\rho_x)
\end{equation}
since $S(\dyad{x}\otimes \rho_x)=S(\dyad{x})+S(\rho_x)$ and $S(\dyad{x})=0$.

It follows that the mutual information $I(A:Q)$ is
\begin{align*}
    I(A:Q) &= S(A) + S(Q) -S(AQ)\\
    &= H(\set{p_x}) + S(\sum p_x \rho_x) - H(\set{p_x}) -\sum p_x S(\rho_x)\\
    &= \chi(\set{p_x,\rho_x}).
\end{align*}
That is, $I(A:Q)$ is the Holevo $\chi$ quantity. We're nearly done. All that remains to to evaluate the LHS,
\begin{equation}
    I(A':B')=S(A')+S(B')-S(A'B').
\end{equation}
Returning to our expression for the post-measurement state $A'Q'B'$, we can trace out the $Q'$ system to get
\begin{align}
    \rho_{A'B'}&=\sum_{x,y} p_x \Tr(\sqrt{E_y} \rho_x\sqrt{E_y}) \dyad{x} \otimes \dyad{y_B}\\
        &=\sum_{x,y} p_x \Tr(E_y \rho_x) \dyad{x} \otimes \dyad{y_B}
\end{align}
by the cyclicity of the trace. Notice that $\Tr(E_y \rho_x) = p(y|x)$, the probability of measuring outcome $y$ given that the state was $\rho_x$, so we have
\begin{equation}
    \rho_{A'B'}= \sum_{x,y} p(x) p(y|x) \dyad{xy}_{A'B'} = \sum_{x,y} p(x,y) \dyad{xy}_{A'B'}.
\end{equation}
Since these are mutually orthogonal states, we find that
\begin{equation}
    S(A'B')=H(\set{p(x,y)}) = H(XY),
\end{equation}
the joint Shannon entropy. Now
\begin{equation}
    \rho_{A'}=\sum p(x) \dyad{x} \implies S(A')= H(\set{p_x})=H(X)
\end{equation}
and similarly
\begin{equation}
    \rho_{B'}=\sum_{x.y} p(x,y)\dyad{y} = \sum p(y) \dyad{y} \implies S(B')=H(Y).
\end{equation}
Thus we find that
\begin{equation}
    I(A':B')=H(X)+H(Y)-H(XY)\equiv I(X:Y),
\end{equation}
i.e. the mutual information between $A'$ and $B'$ is the classical mutual information between the sources $X$ and $Y$. We conclude that
\begin{equation}
    I(X:Y) \leq \chi(\set{p_x,\rho_x}). 
\end{equation}
This is the Holevo bound. \qed

\subsection*{Properties of the Holevo $\chi$ quantity}
Let us denote the average state $\rho=\sum p_x \rho_x$.
\begin{enumerate}
    \item $\chi(\set{p_x,\rho_x}) \geq 0$ (follows from concavity of $S(\rho)$).
    \item $\chi(\mathcal{E})\to S(\rho)$ when the $\rho_x$ are pure.
    \item $\chi(\mathcal{E})=\sum p_x D(\rho_x ||\rho)$. This follows from explicit evaluation:
    \begin{align*}
        \sum p_x D(\rho_x || \rho)
        &= \sum p_x \bkt{-S(\rho_x) - \Tr \rho_x \log \rho}\\
        &= -\sum p_x S(\rho_x) +S(\sum p_x \rho_x).
    \end{align*}
    Notice that $D(\rho_x ||\rho)\geq 0$.
    \item By the data processing inequality, the relative entropy is non-increasing under quantum operations,
    \begin{equation}
        D(\Lambda(\rho_x) || \Lambda(\rho)) \leq D(\rho_x ||\rho).
    \end{equation}
    Taking $\mathcal{E} =\set{p_x,\rho_x}, \mathcal{E}'=\set{p_x,\Lambda(\rho_x)}$, we find that
    \begin{equation}
        \chi(\mathcal{E}')\leq \chi(\mathcal{E}).
    \end{equation}
    \item For an ensemble $\set{p_x,\rho_x}$, we can embed the state in terms of the classical labels as
    \begin{equation}
        \rho_{XA} = \sum p_x \dyad{x} \otimes \rho_x,
    \end{equation}
    known as a classical-quantum (c-q) state.%
        \footnote{We saw this on the examples sheet as well.}
    Then the Holevo quantity of such an ensemble is
    \begin{equation}
        \chi(\set{p_x,\rho_x})=I(X:A)_\rho.
    \end{equation}
    Moreover, we could run the quantum part $A$ through a quantum channel $\Lambda:\cB(\cH_A) \to \cB(\cH_B)$ to get a new state
    \begin{equation}
         \tilde \rho_{XB}=\sum p_x \dyad{x} \otimes \Lambda(\rho_x),
    \end{equation}
    and then it follows from (d) that
    \begin{equation}
        I(X:B)_{\tilde \rho} \leq I(X:A)_\rho.
    \end{equation}
    This shows that the mutual information is non-increasing under a quantum operation which acts only on one subsystem.
\end{enumerate}

\subsection*{Noisy quantum channels}
What happens if our noiseless quantum channel is replaced by a noisy quantum channel $\Lambda$? Clearly, Bob no longer receives $\rho_x$ but $\Lambda(\rho_x)$, and must decode this new state. The maximum information Bob can retrieve for a \emph{single use} of the channel is then bounded by
\begin{equation}
    \chi(\set{p_x,\Lambda(\rho_x)}),
\end{equation}
the Holevo $\chi$ quantity. In fact, we can do better by using the channel multiple times. Preskill has an argument that more uses of the channel always improves the outcome.

Now what is the classical capacity of the channel $\Lambda$? Suppose we have a memoryless quantum channel, i.e.
\begin{equation}
    \Lambda^{(n)} \equiv \Lambda^{\otimes n}.
\end{equation}
That is, the noise is uncorrelated between uses.

\begin{defn}
    The \term{classical capacity} is the maximum rate of reliable transmission of classical information evaluated in the asymptotic limit, i.e. in the limit of channel uses $n\to \infty$.
\end{defn}

Here's something strange about quantum channels as opposed to classical channels. For a classical memoryless channel $\cN$, we model this channel by a set of conditional probabilities $\set{p(y|x)}$, where the capacity was
\begin{equation}
    C(\cN)=\max_{p(x)} I(X;Y).
\end{equation}
This was a unique value. On the other hand, consider a quantum channel $\Lambda$. We have some options which will affect the capacity.
\begin{enumerate}
    \item Information sent-- classical or quantum
    \item The encoded state $\rho^{(n)}$ (input to $n$ uses of the channel)-- entangled or product state. Notice that if $\rho^{(n)} \equiv \rho_1 \otimes \ldots \otimes \rho_n$ is a product state, then the outcome state $\Lambda(\rho_1) \otimes \ldots \otimes \Lambda(\rho_n)$ is also a product state.
    \item The measurement/decoding protocol could act on the $n$ uses individually or collectively.
    \item Presence of auxiliary resources, e.g. Alice and Bob share an entangled state, and Alice uses her half in the encoding of her classical message.
\end{enumerate}

We'll restrict ourselves to the following scenario. Alice has a classical information source and she wants to send a message through a (noisy) memoryless quantum channel $\Lambda$. Thus Alice has some messages
\begin{equation}
    \cM = \set{1,2,\ldots, |\cM|}
\end{equation}
and these messages get encoded as
\begin{equation}
    M \in \cM \mapsto \rho_M^{(n)},
\end{equation}
a quantum state which is then transmitted through the channel as
\begin{equation}
     \sigma_M^{(n)}=\Lambda^{\otimes n}(\rho_M^{(n)}).
\end{equation}
Bob performs a POVM $\set{E_M^{(n)}}_{M\in \cM}$. Call our encoding scheme $\mathcal{E}^{(n)}$ and the decoding scheme $\cD^{(n)}$.

If the message $M$ was sent, the probability of error is then
\begin{equation}
    p_\text{err} = 1-\Tr(E_M^{(n)} \sigma_M^{(n)}).
\end{equation}
The maximum probability of error of $\cC^{(n)}=(\mathcal{E}^{(n)},\cD^{(n)})$ is then
\begin{equation}
    p_\text{max}(\cC^{(n)}) = \max_{m\in \cM} \bkt{1-\Tr(E_M^{(n)} \sigma_M^{(n)}))}.
\end{equation}
We say the information transmssion is reliable if
\begin{equation}
    \lim_{n\to \infty} p_\text{max}^{(n)} =0.
\end{equation}
That is, the maximum probability of error tends to zero in the asymptotic limit.
We then say that the rate of the channel is
\begin{equation}
    R= \frac{\log|\cM|}{n}.
\end{equation}