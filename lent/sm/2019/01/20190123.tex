Today we'll discuss the consequences of discrete symmetries (CPT).
\subsection*{Symmetry operators}
We will start by quoting a result proven by Wigner. 
\begin{thm}
If physics is invariant under some transformation $\Psi\to \Psi'$ (with $\Psi,\Psi'\in$ some Hilbert space), then there is an operator $W$ such that $\Psi'=W\Psi$ and where either $W$ is linear and unitary, or \emph{anti}linear and \emph{anti}-unitary.
\end{thm}
That is, writing the inner product on the hilbert space as $(\cdot,\cdot)$, we have either
\begin{itemize}
    \item $W$ is unitary and linear,
    \begin{equation}
        (W\Phi,W\Psi)=(\Phi,\Psi) \text{ and }W(\alpha \Phi+\beta \Psi)=\alpha W\Phi + \beta W \Psi \, (\alpha,\beta \in \CC)
    \end{equation}
    \item or $W$ is antiunitary and antilinear, 
    \begin{equation}
        (W\Phi,W\Psi)=(\Phi,\Psi)^*\text{ and } W(\alpha \Phi+\beta \Psi)=\alpha^* W\Phi + \beta^* W \Psi.
    \end{equation}
\end{itemize}
Note that $W$ being antiunitary as an operator is not the same as $W$ being an antiunitary matrix ($W^{-1} = -W^\dagger$).

Now, let us recall the Poincar\'e transformations, which take
\begin{equation}
    x^\mu \to \Lambda^\mu{}_\nu x^\nu + a^\mu.
\end{equation}
In particular we have some improper Lorentz transformations (not of $\det = +1$) which are of special importance. There's the parity transformation,
\begin{equation}
    \Lambda^\mu{}_\nu=\mathbb{P}^\mu{}_\nu = \begin{pmatrix}
        1 &&& 0 \\
        & -1 && \\
        && -1 &\\
        0 &&& -1
    \end{pmatrix}
\end{equation}
and also time reversal,
\begin{equation}
    \mathbb{T}^\mu{}_\nu = \begin{pmatrix}
        -1 &&& 0 \\
        & 1 && \\
        && 1 &\\
        0 &&& 1
    \end{pmatrix}.
\end{equation}
Consider an infinitesimal transformation
\begin{equation}
    \Lambda^\mu{}_\nu + \delta^\mu{}_\nu + \omega^\mu{}_\nu, a_\mu = \epsilon_\mu.
\end{equation}
Then the corresponding operator can be expanded as
\begin{equation}\label{wexpansion}
    W(\Lambda,a) = W(1+\omega, \epsilon) = 1+\frac{i}{2} \omega_{\mu\nu} J^{\mu\nu} -i\epsilon_\mu P^\mu,
\end{equation}
where $J^{\mu\nu}$ is the generator of boosts and rotations and $P^\mu$ is a four-momentum operator with $P^0=H$ the Hamiltonian and $p^i=$ the three-momentum operator.

Thus we can write the parity and time reversal operators as
\begin{align*}
    \hat P &= W(\mathbb{P},0)\\
    \hat T &= W(\mathbb{T},0).
\end{align*}
From the general composition rule, we can write
\begin{equation}
    \hat P W(\Lambda, a) \hat P^{-1} = W(\mathbb{P} \Lambda \mathbb{P}^{-1}, \mathbb{P} a).
\end{equation}
If we now insert the expansion of $W$ \ref{wexpansion} on both sides of the equation and compare the coefficients of $\epsilon_0$, we find that
\begin{equation}\label{p-hathamiltonian}
    \hat P iH \hat P^{-1} = iH,
\end{equation}
where we recall that $H=P^0$.
Similarly,
\begin{equation}
    \hat T W(\Lambda, a)\hat T^{-1} = W(\mathbb{T} \Lambda \mathbb{T}^{-1},\mathbb{T} a),
\end{equation}
which implies that
\begin{equation}\label{t-hathamiltonian}
    \hat T i H \hat T^{-1}= - iH.
\end{equation}
We've been careful not to move the $i$ through the operator $\hat T$, since we don't yet know whether the operator is unitary or anti-unitary.

Suppose now $\Psi$ is an energy eigenstate,
\begin{equation*}
    (\Psi, iH \Psi)=i E.
\end{equation*}
If $\hat P$ and $\hat T$ are symmetries, then $\hat P \Psi$ and $\hat T \Psi$ should also be energy eigenstates with energy $E.$

Suppose $\hat P$ is linear. Then we have
\begin{equation}
    (\hat P \Psi, iH \hat P \Psi)=(\hat P \Psi, \hat P iH \Psi) = (\hat P \Psi, \hat P i E \Psi) = iE (\hat P \Psi, \hat P \Psi) = iE,
\end{equation}
by \ref{p-hathamiltonian} and linearity. We could have also run this argument with unitarity instead.

Similarly, suppose $\hat T$ is linear. Then
\begin{equation}
    (\hat T \Psi, i H \hat T \Psi) = -(\hat T \Psi, \hat T iH \Psi) = -iE
\end{equation}
by an equivalent argument using \ref{t-hathamiltonian}. But this tells us that $\hat T$ has produced an energy eigenstate with energy $-iE$, which is wrong.

Therefore, suppose $\hat T$ is anti-linear. Then
\begin{equation}
    (\hat T \Psi, iH \hat T \Psi)=-(\hat T \Psi, \hat T i H \Psi) = -(\hat T \Psi, \hat T iE \Psi)= + iE(\hat T \Psi, \hat T \Psi)=+iE.
\end{equation}
Therefore $\hat T$ must be anti-linear and anti-unitary.

To sum up, the parity operator $\hat P$ is unitary and linear, while the time reversal operator $\hat T$ is antiunitary and antilinear.

\subsection*{Parity} Now that we've defined some basic properties of these symmetries, let's consider what parity does to different fields.

For a complex scalar field, 
\begin{equation}
    \phi(x) = \sum_p \bkt{a(p)e^{-ip\cdot x} + c^\dagger(p) e^{+ip\cdot x}},
\end{equation}
where the operator $a$ annihilates a particle and $c^\dagger$ creates an antiparticle.

The operator $\hat P$ maps momentum eigenstates $\ket{p}\mapsto \eta^{a*}\ket{p_P}$ where
\begin{align}
    p_P &= (p^0,-\vec p)\\
    X_P &= (x^0, - \vec x)
\end{align}
and $\eta^{a*}$ is a complex phase.

Thus
\begin{equation}
    \hat P a^\dagger(p)\ket{0} = \eta^{a*} a^\dagger (p_P) \ket{0}.
\end{equation}
Since $\hat P \hat P^{-1}= I$ and assuming $\hat P\ket{0}=\ket{0}$ (the vacuum is invariant under $\hat P$), we conclude that
\begin{equation}
    \hat P a^\dagger(p) \hat P^{-1} = \eta^{a*} a^\dagger (p_P).
\end{equation}
To preserve the normalization, we must have
\begin{equation}
    \hat P a(p) \hat P^{-1} = \eta^a a(p_P).
\end{equation}
Similarly, we can work out that
\begin{equation}
    \hat P c^\dagger(p) \hat P^{-1} = \eta^{c*}c^\dagger (p_P).
\end{equation}
Now since $\hat P$ is linear and unitary, we can write $\hat P \phi(x) \hat P^{-1}$ as follows:
\begin{align*}
    \hat P \phi(x) \hat P^{-1} &= \sum_p \bkt{ \hat P a(p) \hat P^{-1} e^{-ip\cdot x}+ \hat P c^\dagger (p) \hat P^{-1} e^{+ip\cdot x}}\\
    &= \sum_p \bkt{ \eta^a a(p_P) e^{-ip \cdot x} + \eta^{c*} c^\dagger(p_P) e^{+ip\cdot x}}\\
    &= \sum_{p_P} \bkt{ \eta^a a(p) e^{-ip_P \cdot x} + \eta^{c*} c^\dagger(p) e^{+i p_P \cdot x}}\text{ relabeling }p \leftrightarrow p_P\\
    &= \sum_{p_P} \bkt{ \eta^a a(p) e^{-ip \cdot x_P} + \eta^{c*} c^\dagger(p) e^{+i p \cdot x_P}}\text{ using } p_P\cdot x = p\cdot x_P\\
    &= \sum_{p} \bkt{ \eta^a a(p) e^{-ip \cdot x_P} + \eta^{c*} c^\dagger(p) e^{+i p \cdot x_P}}\text{ relabeling }\sum_p = \sum_{p_P.}
\end{align*}
Note that this does not ``look like'' $\phi(x_P)$ unless $\eta^a = \eta^{c*} \equiv \eta_p$ (if you like, we're matching the coefficients of Fourier modes). If you're not convinced by this, notice that we would not in general find the commutator $[\phi(x),\hat P \phi^\dagger(y) \hat P^{-1}]$ vanishes for spacelike $x-y$.
