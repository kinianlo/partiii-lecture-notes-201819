\begin{quote}
    \textit{``Wald proved a version of the first law [of black hole mechanics] which is the most badass, hard version to prove.'' --Jorge Santos}
\end{quote}

In our last class, we talked about Killing horizons. We have seen that if $\cH^+$ is a Killing horizon with Killing vector field $\xi^a$, then
\begin{equation}
    \xi^a \nabla_a \xi^b|_{\cH^+} = \kappa_0 \xi^b.
\end{equation}
For the RN and Kerr-Newman solutions, the surface gravity $\kappa_0$ (which a priori could have been a function) is in fact a constant.

Hawking thought that perhaps we could prove that in general, the surface gravity was a constant and not dependent on where on the horizon we looked. This turns out to be true, and it leads us to the zeroth law of black hole mechanics.

\subsection*{Zeroth law}
\begin{prop}
    Consider a null geodesic congruence that contains the generators of a Killing horizon $\cN$. Then $\theta=\hat \sigma =\hat \omega=0$.
\end{prop}
\begin{proof}
    We know that $\hat \omega=0$ since our congruence is hypersurface orthogonal.%check Reall notes
    Let $\xi^a$ be a Killing field normal to $\cN$. On $\cN$, we can write $\xi^a= hU^a$, where $U^a$ is affinely parametrized and $h$ is a function on $\cN$.
    Let the horizon $\cN$ be specified by an equation $f=0$. Then we can write $U^a=h^{-1} \xi^a + f V^a$, where $V^a$ is a smooth vector field. This lets us extend $U^a$ off of $\cN$ a bit.
    
    We can calculate the optical matrix
    \begin{equation}
        B_{ab}=\nabla_b U_a = (\nabla_a h^{-1})\xi_a + h^{-1} \nabla_b \xi_a + (\nabla_b f)V_a +f \nabla_b V_a.
    \end{equation}
    Evaluating on $\cN$ and using the fact that $\xi$ is Killing,
    \begin{equation}
        B_{ab}|_\cN = \xi_{(a}\nabla_{b)} h^{-1} + V_{(a}\nabla_{b)}f|_{\cN},
    \end{equation}
    where we only consider the symmetric part since the antisymmetric part was $\omega$ and vanishes. But both $\xi_a,\nabla_a f$ are parallel to $U_a$ on $\cN$. Hence when we project onto $T_\perp$, we find that
    \begin{equation}
        \hat B_{(ab)}|_\cN = P_a{}^c B_{(cd)} P_b{}^d = 0,
    \end{equation}
    so indeed $\hat \sigma=0$.
\end{proof}

\begin{thm}[Zeroth law of black hole mechanics]
    $\kappa_0$ is constant on the future event horizon of a stationary black hole spacetime obeying the dominant energy condition.
\end{thm}
\begin{proof}
    Note that Hawking's theorem implies that $\cH^+$ is Killing with respect to some Killing vector field $\xi^a$. We have just seen that $\theta=0$ along the generators of $\cH^+$. Hence $\frac{d\theta}{d\lambda}=0$ (where $\lambda$ parametrizes the integral curves of the generators). We also have $\hat \sigma = \hat \omega =0$. Now the Raychaudhuri equation says that
    \begin{equation}
        0=R_{ab} \xi^a \xi^b|_{\cH^+} = 8\pi T_{ab} \xi^a \xi^b|_{\cH^+}
    \end{equation}
    where we've applied the Einstein equation and dropped the trace term since we're on the horizon.
    
    This implies that
    \begin{equation}
        J\cdot \xi|_{\cH^+} =0
    \end{equation}
    where $J_a \equiv -T_{ab}\xi^b$. Now $\xi^a$ is a future-directed causal vector, so by the dominant energy condition, $J_a$ is also future-directed. Hence the above equation implies that $J^a$ is parallel to $\xi^a$ on $\cH^+$.%revisit this?
    
    Since this is the case, we can look at the expression
    \begin{equation}
        0=\xi_{[a}J_{b]}|_{\cH^+}= -\frac{1}{8\pi} \xi_{[a} R_{b]c} \xi^c |_{\cH^+}.
    \end{equation}
    On Examples Sheet 4, problem 1, we will prove that the previous expression implies the following:
    \begin{equation}
        0=\frac{1}{8\pi} \xi_{[a}\nabla_{b]}\kappa_0|_{\cH^+}.
    \end{equation}
    But this means that $\nabla_a \kappa_0$ is proportional to $\xi_a$, i.e. 
    \begin{equation}
        t \cdot \nabla \kappa_0=0
    \end{equation}
    for any tangent vector to $\cH^+$. Hence $\kappa_0$ is constant on the horizon.
\end{proof}
In fact, one can relax the energy condition from the DEC a bit, but it requires other details about the spacetime (asymptotic flatness and others).

\subsection*{First law of black hole mechanics}
Consider the Kerr spacetime. The horizon area $A$ is the area of the intersection of $\cH^+$ with a partial Cauchy surface of $t={}$constant. It is also the area of the bifurcating Killing horizon. $A$ will be a function of $J$ and $M$, and we can consider the quantity $\frac{\kappa_0}{8\pi}$. This turns out to be
\begin{equation}
    \frac{\kappa_0}{8\pi}\delta A = \delta M -\Omega_H \delta J.
\end{equation}
This leads us to an interesting question. Start with a particular Kerr solution. Take a slice of the Kerr solution, perturb the initial conditions, and solve the perturbed initial value problem. Can we show that the area of the black hole increases in this way? This is hard to do and in fact wasn't completed until the 1990s.%Wald-- see Reall
%This is a version of the first law which is the most badass, hard version to prove.
However, there is a more physical argument we can make about this relationship, made much earlier by Hartle and others.

Suppose we have some matter with a stress-energy tensor of $O(\epsilon)$. That is, it is small compared to the stress-energy of the black hole. Let
\begin{equation}
    J^a = -T^a{}_b K^b,\quad L^a = T^a{}_b m^b
\end{equation}
where $K$ is the stationary Killing vector field and $m$ is the axisymmetric Killing vector field. We would like these currents to still be conserved after we throw some stuff in. In Kerr, $\nabla_a J^a$ is exactly zero. In the new metric, we have instead
\begin{equation}
    \nabla_a J^a = O(\epsilon^2), \nabla_a L^a = O(\epsilon^2)
\end{equation}
That is, the metric is perturbed from Kerr by $g=\bar g_K +\epsilon \hat g$ where $\bar g_K$ is exactly Kerr. Hence we get an $O(\epsilon)$ correction in the covariant derivative from the perturbation to the metric, and we get another factor of $\epsilon$ from the perturbation of the stress-energy tensor. Hence to order epsilon this really is a current.

One may compute (on examples sheet 3) that
\begin{equation}
    \delta M =-\int_\cN * J, \quad \delta J = -\int_\cN * L.
\end{equation}
Note that the $J$ in the first expression is the current corresponding to the Killing vector $K$, while the second $J$ is the angular momentum of the black hole.

Now, one may choose Gaussian null coordinates on the horizon. In these coordinates, $\cH^+$ is the surface $r=0$, and the metric on $\cH^+$ takes the form
\begin{equation}
     ds^2_{\cH^+}= 2drd\lambda +h_{ij}(\lambda,y) dy^i dy^j.
\end{equation}
Using $\sqrt{-g}=\sqrt{h}$, we can write
\begin{equation}
    \eta=\sqrt{h} d\lambda \wedge dr \wedge dy^1 \wedge dy^2
\end{equation}
and fix an orientation (i.e. choose whether or not to put a minus sign on our definition of $\eta$ as a volume element).

The orientation of $\cN$ used in defining $\delta M, \delta J$ is the one used in Stokes's theorem. Viewing $\cN$ as the boundary of the region with $r>0$ (the black hole exterior, if you like), the volume element on $\cN$ is $d\lambda \wedge dy^1 \wedge dy^2$. Hence on $\cN$,
\begin{equation}
    (*J)_{\lambda 12} = \sqrt{h} J^r = \sqrt{h} J_\lambda = \sqrt{h} U\cdot J
\end{equation}
where $U=\P{}{\lambda}$.

We can now evaluate everything on Kerr, since the deviation is $O(\epsilon^2)$. Hence $\cN$ is Killing with a killing vector $\xi=K+\Omega m$ on $\cN$, and we also have $\xi=f U$ for a function $f$. Using $\xi^a \nabla_a \xi^b|_{\cH^+}=\kappa_0 \xi^b$, we find that
\begin{equation}
    \xi \cdot \nabla( \log|f|)=\kappa_0 \implies f= \kappa_0 \lambda + f_0(y)
\end{equation}
(where we can integrate knowing $\kappa_0=0$).

So far, we have
\begin{equation}
     \xi^a = [\kappa_0 \lambda +f_0(y)]U^a,
\end{equation}
where $\lambda=0$ is the bifurcating surface. Hence we find that the integration constant vanishes, $f_0(y)=0$, and so
\begin{equation}
    \xi^a = \kappa_0 \lambda U^a\text{ on } \cN.
\end{equation}

From the definition of $\delta M$, we have (substituting)
\begin{align*}
    \delta M &= \int_\cN d\lambda d^2y \sqrt{h} T_{ab} U^a(\xi^b -\Omega_H m^b)\\
        &= \int_\cN d\lambda dy^2 \lambda \sqrt{h } T_{ab} U^a U^b \kappa_0 \lambda -\Omega_H \int_\cM d\lambda d^2y \sqrt{h} U\cdot L,
\end{align*}
where this second term is precisely $-\delta J$. However, we have not yet used the Einstein equation-- we can turn $T_{ab}$ into a curvature condition,
\begin{equation}
    8\pi T_{ab} U^a U^b = R_{ab} U^a U^b,
\end{equation}
so that
\begin{align*}
    \delta M -\Omega_H \delta J &= \frac{\kappa_0}{8\pi} \int_\cN d\lambda d^2 y \sqrt{h} \lambda R_{ab} U^a U^b\\
    \implies \delta M -\Omega_H \delta J &= -\frac{\kappa_0}{8\pi} \int d^2y \int_0^{+\infty} \sqrt{h} \lambda \frac{d\theta}{d\lambda} d\lambda\\
    &= -\frac{\kappa_0}{8\pi} \int d^2y \set*{ [\sqrt{h}\lambda\theta]_0^{+\infty} -\int_0^{+\infty}(\sqrt{h}+ \underbrace{\lambda \frac{d\sqrt{h}}{d\lambda}}_{O(\epsilon^2)})\theta d\lambda},
\end{align*}
where we applied Raychaudhuri and integrated by parts.

Now recall that
\begin{equation}
    \frac{d\sqrt{h}}{d\lambda}=\theta\sqrt{h}=O(\epsilon).
\end{equation}
We will assume the final states exist in a $\lambda\to +\infty$ limit, $\sqrt{h}$ is finite. Hence
\begin{equation}
    \int_0^{+\infty} \sqrt{h}\theta d\lambda = \int_0^{+\infty} \frac{d\sqrt{h}}{\lambda} d\lambda = \delta(\sqrt{h}).
\end{equation}
Finiteness gives $\theta = o(1/\lambda)$ (decays at least as fast as $1/\lambda$), and hence the boundary terms go away. We learn that
\begin{equation}
    \delta M-\Omega_H \delta J = \frac{\kappa_0}{8\pi} \delta A. \qed
\end{equation}