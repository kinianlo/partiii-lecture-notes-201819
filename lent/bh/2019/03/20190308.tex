\begin{quote}
    \textit{``So far in the course everything has been done in mathematical rigor, at least to my standards. But this is the work of the devil. It smells like sulphur. Nevertheless we are going to roll with it.'' --Jorge Santos}
\end{quote}
\subsection*{Second law of black hole mechanics}
\begin{thm}[Hawking 1972]
    Let $(\cM,g)$ be a strongly asymptotically predictable spacetime satisfying the Einstein equation with the null energy condition. Let $U\subset \cM$ be globally hyperbolic region for which $\bar J^-(\mathcal{I}^+)\subset U$.  Let $|Sigma_1,\Sigma_2$ be spacelike Cauchy surfaces for $U$ with $\Sigma_2 \subset J^+(\Sigma_1)$ and let $H_1=\cH^+ \cap \Sigma_i$. Then $\text{area}(H_2) \geq \text{area}(H_1)$.
\end{thm}
That is, the surface area of the black hole event horizon is monotonically increasing.
\begin{proof}
    We will make the additional assumption that inextendible generators of $\cH^+$ are future complete (i.e $\cH^+$ is non-singular). Recall that the generators for a null congruence of geodesics, and their behavior is in part governed by the expansion $\theta$. So it might be a good start to show that $\theta \geq 0$ on $\cH^+$.
    
    The proof is by contradiction. Suppose $\theta < 0$ at $p\in \cH^+$, and let $\gamma$ be the (inextendible) generator of $\cH^+$ through $p$ and let $q$ be slightly to the future of $p$ along $\gamma$. But then we know that there exists a point $r$ conjugate to $p$ on $\gamma$ using our assumption and the Raychaudhuri equation.
    
    Recall our theorem about conjugate points, however. %add reference
    This tells us that we can deform $\gamma$ to obtain a timelike curve from $p$ to $r$, but this violates achronality of $\cH^+$.
    
    Let $p\in H_1$. The generators of $\cH^+$ through $p$ cannot leave $\cH^+$, so they must intersect $H_2$ as $\Sigma_2$ is a Cauchy surface. This therefore defines a map $\phi:H_1 \to H_2$ (just follow the generators of $\cH^+$). Now $\text{area}(H_2) \geq \text{area}(\phi(H_1)) \geq \text{area}(H_1)$, where the first inequality follows because $\phi(H_1) \subset H_2$ and the second inequality follows from $\theta >0$.
    %I encourage you to look up other proofs of [the second law]. There is a proof which Malcolm used to lecture in his course with way too many indices.
\end{proof}

Suppose we have two black holes at some time in the past with masses $M_1,M_2$ and they merge to form a new black hole with mass $M_3$. The theorem we have just proved told us that
\begin{equation}
    A_3 \geq A_1 +A_2 \implies M_3^2 \geq M_1^2 +M_2^2.
\end{equation}
We define the efficiency of the merger process to be
\begin{equation}
    \frac{M_1+M_2 -M_3}{M_1+M_2} \leq 1-\frac{1}{\sqrt{2}}.
\end{equation}
By looking at a BH merger (e.g. by LIGO), we could have checked the area theorem. We have very sensitive measurements of $M_1$ and $M_2$, but we currently have no way to measure $M_3$. In fact, the $M_3$ that was reported comes from numerics (input $M_1$ and $M_2$ and simulate the merger)-- with a more sensitive detector, we might have seen the quasinormal ringdown (decaying oscillations) of the post-merger black hole, but we weren't able to at the time, and hence Stephen Hawking didn't win a Nobel Prize.

Let us now recap the three laws of black hole mechanics.
\begin{itemize}
    \item 0th law: $\kappa_0$ is constant.
    \item $dM=\frac{\kappa_0}{8\pi} dA + \Omega dJ$
    \item $\Delta A \geq 0$.
\end{itemize}
Notice the first law looks suspiciously like the first low of thermodynamics,
\begin{equation}
    dE = TdS + \mu dJ
\end{equation}
under the identification 
\begin{equation}
    T=\lambda \kappa_0, \quad S=\frac{A}{8\pi \lambda}
\end{equation}
where $\lambda$ is some undetermined number. The second law checks out here as well-- $\delta A \geq 0$ tells us entropy is increasing. It's also reasonable to think that black holes are thermodynamic objects-- there's entropy in the stuff we throw in, so there should still be entropy in the black hole system by the second law of thermodynamics. This argument is due to Bekenstein.

But we just proved that black holes were black-- classically, nothing escapes the black hole. But thermodynamic objects must radiate. What's the solution?

It is the following formula:
\begin{equation}
    T=\frac{\hbar \kappa_0}{2\pi},
\end{equation}
taking $\lambda=\hbar/2\pi$. We see now that this is something quantum. In the classical limit as $\hbar \to 0$, the entropy becomes infinite.
%In 1974, it was not clear that QFT was the right thing to use. THis was Stephen's insight
%So far in the course everything has been done in mathematical rigor, at least to my standards.
%But this is work of the devil. It smells like sulphur. Nevertheless we are going to roll with it.
%If you have not taken QFT, this will look to you like Chinese looks to me. ...I don't understand any Chinese.

\subsection*{Quantization of the free scalar field}
Let $(\cM,g)$ be a globally hyperbolic spacetime with metric
\begin{equation}
    ds^2 =-N^2 dt^2 + h_{ij}(dx^i + N^i dt)(dx^j + n^j dt).
\end{equation}
Let $\Sigma$ be a Cauchy surface with normal $n_a=-N(dt)_a$. The metric on $\Sigma_t$ is simply $h_{ij}$, with $\sqrt{-g}=N\sqrt{h}$.
Now we can write down the action for a free scalar field,
\begin{equation}
    S=\int_\cM dt d^3x \sqrt{-g} \bkt{-\frac{1}{2} \nabla^a \Phi \nabla_a \Phi -\frac{1}{2} m^2 \Phi^2}.
\end{equation}
We compute the conjugate momentum to the field; it is
\begin{equation}
    \pi(x) =\frac{\delta S}{\delta (\p_t \Phi)} = \sqrt{h} n^\mu \nabla_\mu \Phi.
\end{equation}
Next, to quantize we promote $\Phi,\pi$ to operators (in $\hbar =1$ units) such that they satisfy equal-time commutation relations
\begin{equation}
    [\Phi(t,x),\pi(t,x')]=i\delta^{(3)}(x-x').
\end{equation}

We now introduce the Hilbert space where the field lives. Let $S$ be the space of complex solutions to the Klein-Gordon equation ($\Box \Phi=m^2 \Phi$, where $\Box=\nabla_a \nabla^a$). Global hyperbolicity implies that a point in $S$ is specified uniquely by the initial conditions $\Phi, \p_t \Phi$ on $\Sigma_0$. Note that the Hilbert space comes equipped with an inner product:
\begin{equation}
    (\alpha,\beta)=-\int_{\Sigma_0} d^3x \sqrt{h} n_a j^a(\alpha,\beta),\quad j(\alpha,\beta)\equiv -i (\bar \alpha d\beta -\beta d\bar \alpha).
\end{equation}
In fact, one now finds that this current $j$ we have defined is actually conserved, i.e. from Klein-Gordon it follows that
\begin{equation}
    \nabla^a J_a = 0.
\end{equation}
This implies our norm is actually independent of $t$-- while it may be convenient to calculate on $\Sigma_0$, $(\alpha,\beta)$ does not depend on $t$.

Note that
\begin{equation}
    (\alpha,\beta)=\overline{(\beta,\alpha)},
\end{equation}
which implies that $(\cdot,\cdot)$ is a Hermitian form. It is non-degenerate, i.e. if $(\alpha,\beta)=0$ for all $\beta \in S,$ then $\alpha=0$. However,
\begin{equation}
    (\alpha,\beta)=-(\bar \beta,\bar \alpha),\text{ so }(\alpha,\alpha)=-(\bar \alpha,\bar \alpha),
\end{equation}
which tells us our inner product is not positive-definite. In Minkowski space, the inner product $({},{})$ is positive-definite on the subpsace $S_p$ of $S$ consisting of positive frequency solutions. A basis for $S_p$ is the set of plane waves,
\begin{equation}
    \psi_{\vec p} (x) = \frac{1}{(2\pi)^{3/2}(2p^0)^{1/2}} e^{ip \cdot x},\quad p^0 =\sqrt{\vec p^2 +m^2}. 
\end{equation}
These modes (i.e. solutions of the K-G equation) are positive frequency in the sense that if we take $K=\P{}{t}$ to be our stationary Killing field, then these modes have negative imaginary eigenvalues with respect to the Lie derivative $\cL_K$. That is,
\begin{equation}
    \cL_K \psi_{\vec p} = -i p^0 \psi_{\vec p}
\end{equation}
Now observe that the complex conjugate of $\psi_{\vec p}$ is a negative frequency plane wave. By linearity, we can decompose
\begin{equation}
    S=S_p + \bar S_p.
\end{equation}

However, note that in curved spacetime, we do not have a definition of ``positive frequency'' except if the spacetime is stationary (in which case the Killing field provides us a ``time'' direction). Instead, we simply choose a subspace $S_p$ for which $(\cdot,\cdot)$ is positive-definite and $S$ can be decomposed $S=S_p\oplus \bar S_p$. In general there are many ways to do this, and moreover physics may look very different depending on how we decompose the space.

In the QFT, we define the creation and annihilation operators to be modes $f\in S_p$ of a real scalar field ($\Phi^\dagger = \Phi$) by
\begin{equation}
    a(f)=(f,\Phi),\quad a^\dagger(f) =-(\bar f,\Phi).
\end{equation}
These imply that
\begin{equation}
    [a(f),a(g)^\dagger]=(f,g)
\end{equation}
which in Minkowski space reduces to
\begin{equation}
    [a_{\vec p},a^\dagger_{\vec q}]=\delta^{(3)}(\vec p- \vec q).
\end{equation}
The last ingredient we need is a vacuum state. Once we have a vacuum state, we can say that the Hilbert space is just the Fock space we get by acting on the vacuum repeatedly with the one-particle operators. We therefore define the (normalized) vacuum state $\ket{0}$ by the condition
\begin{equation}
    a(f)\ket{0} = 0\, \forall f\in S_p, \quad \braket{0}{0}=1.
\end{equation}
However, suppose we chose a different positive frequency space, $S_p'$. The for any $f'\in S'_p$, we can decompose this function in the original basis, $f'=f+\bar g$ for $f\in S_p,\bar g\in \bar S_p$. Now the annihilation operator of our $f'$ is
\begin{equation}
    a(f') = (f,\Phi) +(\bar g,\Phi)=a(f) -a(g)^\dagger.
\end{equation}
And hence what we would have called a positive frequency solution relative to $S_p$ now gives us an $a(g)^\dagger$, which creates a particle relative to the original vacuum. We see that something as innocuous-sounding as a decomposition into positive and negative frequency solutions means that we won't agree even on the number of particles in the system.