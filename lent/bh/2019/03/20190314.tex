\begin{quote}
    \textit{``Look! This has G, $\hbar$, and c. That's all the fundamental constants. This is the one statement about quantum gravity we know is true.'' --Jorge Santos}
\end{quote}

Today, we conclude the calculation of Hawking radiation. Using WKB, we took our modes to have the form $\Phi=A(x)e^{i\lambda S(x)},\lambda \gg 1$, and applying $\Box \Phi=0$ we have $\nabla^a S \nabla_a S=0$ to leading order in $\lambda$. Surfaces of constant phase $S$ are null hypersurfaces.

Now, we can introduce a null vector $N$ which is parallel propagated along the generators $U$
\begin{equation}
    U^b \nabla_b N^a =0,\quad N\cdot U=-1.
\end{equation}
We can import some of our results from Kruskal and take
\begin{equation}
    U^a = \paren{\P{}{V}}^a,\quad N = c \P{}{U}
\end{equation}
with some constant $c$. Hence any deviation vector for this congruence can be decomposed into the sum of a part orthogonal $U^a$ and a term $\beta N^a$ which is parallel transported along the geodesics.

Choose $\beta = -\epsilon$ where $\epsilon>0$ and small. A nearby geodsic $\gamma'$ then has tangent
\begin{equation}
    U=-C\epsilon \implies u=-\frac{1}{\kappa_0} \log(-U) = -\frac{1}{\kappa_0} (C\epsilon).
\end{equation}
Let $F(u)$ denote the phase of our wavepacket $p_i$ on $\mathcal{I}^+$. Then the phase everywhere on $\gamma'$ must be
\begin{equation}
    S=F(-\frac{1}{\kappa_0} \log(C\epsilon)),
\end{equation}
so we know what the value of the phase is in the region $v<0$. We want to translate this into a statement in terms of $v$.

In fact, we've already seen the technology to do this calculation and find $\epsilon$ in terms of $v$. This is because on $\mathcal{I}^-$, we have another null hypersurface and we can therefore apply the same reasoning as we did near $\mathcal{I}^+$. Recall that the metric on $\mathcal{I}^-$ is
\begin{equation}
    ds^2 =-du dv +\frac{1}{4}(u-v)^2 d\Omega_2^2,
\end{equation}
so we again introduce $N$ as before,
\begin{equation}
    N=D^{-1} \P{}{v}
\end{equation}
at $\mathcal{I}^-$. We cocnlude that
\begin{equation}
     v=-D^{-1}\epsilon \implies \epsilon = -Dv.
\end{equation}
Hence the phase on $\mathcal{I}^-$ in terms of $v$ is
\begin{equation}
    S=F\bkt{-\frac{1}{\kappa_0} \log(-CDv)}.
\end{equation}
We have
\begin{equation}
    p_i^{(2)} \approx \begin{cases}
        0, & v>0\\
        A_i(v) \exp \bkt{iF \paren{-\frac{1}{\kappa_0} \log(-D(v)}}, & v<0.
    \end{cases}
\end{equation}
Notice that most of the wavepacket is very close to $v=0$, where the argument of the log becomes very large.
%So now I am going to cheat very badly.

We'll now cheat ``very badly'' by converting everything to plane waves.  This is not really valid-- some of the integrals we will write down diverge, breaking some of our original assumptions. For plane waves,
\begin{equation}
    F(u)=-\omega_i u.
\end{equation}
This leads to $p_i$ that are neither normalizable nor localized at late times. This is more of an illustrative calculation-- the integrals for e.g. the modulated cosine can be computed, but they are hard.
\begin{equation}
    p_\omega \simeq \begin{cases}
        0, & v>0\\
        A_\omega(v) \exp \bkt{i\frac{\omega}{\kappa_0} \log(-CDv)}, & v<0.
    \end{cases}
\end{equation}
Similarly, we will use a basis of ``in'' modes $f_\sigma$ such that $v_\sigma$ has frequency $\sigma>0$. THus
\begin{equation}
    f_\sigma = \frac{1}{(2\pi N_\sigma)} e^{-i\sigma v}.
\end{equation}
These are just related by a Fourier transform:
\begin{equation}
    \tilde p_\omega^{(2)} (\sigma) = A\omega \int_{-\infty}^0 dv e^{i\sigma v} \exp \bkt{i\frac{\omega}{\kappa_0}\log(-CDv)}.
\end{equation}
with inverse transform
\begin{align*}
    p_\omega^{(2)} (v) &=\int_{-\infty}^\infty \frac{d\sigma}{2\pi} \tilde p^{(2)}_\omega (v) e^{-i\sigma v}\\
    &= \int_0^{+\infty} d\sigma N_\sigma \tilde p_\omega^{(2)} f_\sigma(v) + \int_0^{+\infty} d\sigma \bar N_\sigma \tilde p_\omega^{(2)} (v) \bar f_{\sigma} (v).
\end{align*}
But we see that $A$ is just related to the positive frequency ($f(v)$) components and $B$ is related to the negative frequency components, hence
\begin{equation}
    A^{(2)} \omega_\sigma = N_\sigma \tilde p_\omega^{(2)} (\sigma), \quad B \omega_\sigma = \bar N_\sigma \tilde p_\omega^{(2)}(-\sigma), \quad \omega,v >0.
\end{equation}

We now have to choose where the branch cut lies for the log in the complex plane. Define
\begin{equation}
    \log z = \log|z| + i\text{ang} z,\quad \text{ang} \in (-\pi/2,3\pi/2).
\end{equation}
Here's the first integral that diverges. We're integrating over $-\infty$ to $0$ in $v$, and we close the contour by adding a contour over $0\to\infty$ and close the contour in the lower half-plane. The log is otherwise analytic, so we assume that $I_1$ the integral $-\infty \to 0$ is just $-I_2$ the integral from $0\to\infty$. Hence
\begin{align*}
    \tilde p_\omega^{(2)} (-\sigma) &=-A\omega \int_0^{+\infty} dv e^{-\sigma v} \exp \bkt{i\frac{\omega}{\kappa_0} \log (-CDv)}\\
    &= -A\omega \int_0^{+\infty} dv e^{-i\sigma v} \exp \set*{i\frac{\omega}{\kappa_0} [\log(CDv)]} e^{2\omega \pi/\kappa_0} = -e^{-\omega \pi/\kappa_0} p_\omega^{(2)} (v).
\end{align*}
We find that the $B$s and $A$s are related:
\begin{equation}
    |B\omega \sigma| = e^{-\omega \pi/\kappa_0} |A_{\omega \sigma}^{(2)}|.
\end{equation}
If we were braver, we might have repeated this calculation with the actual wavepackets and found that
\begin{equation}
    |B_{ij}|=e^{-\omega_i \pi/\kappa_0} |A_{ij}^{(2)}|.
\end{equation}
Then
\begin{equation}
    T_i^2=(p_i^{(2)}, p_i^{(2)}) = \sum_j (|A_{ij}^{(2)}|^2 -|B_{ij}|^2) \implies (BB^\dagger)_{ii} =\frac{T_i^2}{e^{2\omega_i \pi/\kappa_0}-1}.
\end{equation}
This gives (almost) blackbody radiation at the \emph{Hawking temperature}
\begin{equation}
    T_H= \frac{\kappa_0}{2\pi}.
\end{equation}

For a solar mass black hole, we find that
\begin{equation}
    T_H = 6\times 10^{-8} \frac{M_{\cdot}}{M} \text{ Kelvin}.
\end{equation}
This can be generalized to the fermionic case, the massive case, and the gravitational case, amongst others. This is remarkable.

\subsection*{Black hole thermodynamics}
Now that we have a concept of temperature, we can define an entropy! Putting constants back in,
\begin{equation}
    S_{BH}=\frac{c^3 A}{4G \hbar}.
\end{equation}
%Look! This has G, \hbar, and c. That's all the fundamental constants. This is the one statement about quantum gravity we know is true.
This is an incredible result. Any theory of quantum gravity must reproduce the entropy formula.

Notice that for a solar mass black hole, $S_{BH}\sim 10^{77}$, whereas $S_\cdot \sim 10^{58}$. Why isn't everything in the universe just black holes? It's because the initial state of the universe was very special, as postulated by Roger Penrose.

Moreover, if we calculate the energy change from Hawking radation, we find that
\begin{equation}
    \frac{dE}{dt} \approx -\alpha A T^4.
\end{equation}
This leads us to a generalized second law-- because of evaporation, it's not the black hole entropy alone which is nondecreasing but the entropy of the BH-universe system.
%In fact, Stephen lost all bets he made in life.